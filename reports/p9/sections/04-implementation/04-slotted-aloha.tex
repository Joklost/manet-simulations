\section{Slotted ALOHA}\label{sec:saloha}
In this section we will present an implementation of the Slotted ALOHA protocol~\cite{Roberts:1975:APS:1024916.1024920}, using the hardware interface from \autoref{sec:hardwareinterface}. The Slotted ALOHA is a \gls{tdma} protocol, that enable communication between nodes in a wireless network, by randomly selecting a time slot to transmit in, while listening in every other time slot. Pseudocode describing the Slotted ALOHA protocol can be seen in \autoref{algo:slottedaloha}. \medbreak

%An important part of any \gls{tdma} protocol is that only a single node should transmit at a given timeslot, otherwise collisions will occur. Thus, the Slotted ALOHA protocol is a good choice for 

\begin{algorithm}[ht]
    \DontPrintSemicolon
    \SetKwFunction{FSlottedALOHA}{Slotted ALOHA}
    \SetKwProg{Fn}{procedure}{}{}

    \Fn{\FSlottedALOHA{}}{
        \Repeat{\textit{protocol terminates}}{
            selected $\leftarrow$ randomly select a slot $\in$ {1, 2, \dots, S}\;

            \ForEach{current $\in$ {1, 2, \dots, S}}{
                \If{select = current}{
                    \KwBroadcast\;
                }
                \Else{
                    \KwListen\;
                }
            }
        }
    }

    \caption{The Slotted ALOHA protocol.}
    \label{algo:slottedaloha}
\end{algorithm}

\todo[inline]{Should we include cpp code, using the functions from the header file?}