
\section{Slotted ALOHA}\label{sec:saloha}
Slotted ALOHA~\cite{Roberts:1975:APS:1024916.1024920} is a \gls{tdma} protocol, that enable communication between nodes in a wireless network, by randomly selecting a time slot to transmit, while listening in every other time slot. This method of selecting time slots for transmitting will have a high probability of collisions happening, where multiple nodes transmit at the same time, which make this protocol an ideal candidate for simulating collisions and packet loss. 

%An important part of any \gls{tdma} protocol is that only a single node should transmit at a given timeslot, otherwise collisions will occur. Thus, the Slotted ALOHA protocol is a good choice for 
% TODO: Add number of arguments as parameter
\begin{algorithm}[ht]
    \DontPrintSemicolon
    \SetKwFunction{FSlottedALOHA}{SlottedALOHA}
    \SetKwProg{Fn}{procedure}{}{}

    \Fn{\FSlottedALOHA{S}}{
        secret $\leftarrow$ a single node picks a secret message\;

        \Repeat{\textit{1 hour}}{

            selected $\leftarrow$ randomly select a slot $\in$ {1, 2, \dots, S}\;

            \ForEach{current $\in$ {1, 2, \dots, S}}{
                \If{select = current}{
                    \If{node knows secret}{
                        duration $\leftarrow$ \texttt{Broadcast}(select)\;
                        \texttt{Sleep}(10s - duration)\;
                    }\Else{
                        \texttt{Sleep}(10s)\;
                    }
                }
                \Else{
                    \If{node knows secret}{
                        \texttt{Sleep}(10s)\;
                    }\Else{
                        secret $\leftarrow$ \texttt{Listen}(10s)\;
                    }                   
                }
            }
        }
    }

    \caption{The Slotted ALOHA protocol~\cite{Roberts:1975:APS:1024916.1024920}.}
    \label{algo:slottedaloha}
\end{algorithm}

The Slotted ALOHA protocol shown in \autoref{algo:slottedaloha} is a modified version of the protocol, where the goal is to share a secret message between every node a the network. Initially, a single random node picks a secret message, and the protocol begins. The protocol is designed to conserve energy by sleeping as much as possible. If the node knows the secret message, it will broadcast the message when its selected time slot comes up, and sleep in any other time slot. If the node does not know the secret message, it will sleep in its selected time slot, and listen for the message in any other time slot.

\subsection{Experiment}
We ran an experiment with the Slotted ALOHA protocol from \autoref{algo:slottedaloha}. The experiment was conducted on a network topology of 33 nodes generated from \acrshort{gps} location data, from a field test conducted in the Philippines. The topology can be found in \autoref{ch:app:aloha-topologies}. The goal of the experiment is to show the difference between virtual and real time, when running the protocol for an hour with a different number of time slots S. 

\begin{table}[H]
    \begin{tabular}{|l|l|l|l|l|l|l|l|l|}
        \hline
        \diagbox{Time}{Slots (S)} & 5    & 6   & 7    & 8    & 9    & 10    & 11   & 12   \\ \hline
        Real                  & 45 s & 5 s & 90 s & 60 s & 50 s & 105 s & 60 s & 88 s \\ \hline
        %        Virtual               & 6h, 48 m & 5 h, 36 m & 4 h, 48 m & 4 h, 15 m& 3 h, 42 m&    &    & 2 h, 48m   \\ \hline
    \end{tabular}
    \caption{Real time results from running the ALOHA protocol with different total time slots.}
    \label{table:experiments:aloha-results}
\end{table}

\autoref{table:experiments:aloha-results} show the real time duration the simulation took to run the Slotted ALOHA with different parameters for the time slots S. We see that the shortest simulation took 5 seconds with 6 time slots, and the longest took 105 seconds with 10 time slots. This demonstrates the non-deterministic nature of the Slotted ALOHA protocol, where the time required to complete the simulation heavily depends on the number of colliding transmissions and lost packets. The main takeaway from this experiment is that we are able to run the hour long Slotted ALOHA protocol in less than two minutes, with 33 nodes.

%The results show that virtual time gives significant time savings, when running the simulations.

%Sleep real fast, listen real slow. Done.
%