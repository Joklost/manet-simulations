
\section{Slotted ALOHA}\label{sec:saloha}
\todo[inline]{Make sure code is up to date!}
%In this section we will present an implementation of the Slotted ALOHA protocol~\cite{Roberts:1975:APS:1024916.1024920}, using the hardware interface from \autoref{sec:hardwareinterface}. 
Slotted ALOHA~\cite{Roberts:1975:APS:1024916.1024920} is a \gls{tdma} protocol, that enable communication between nodes in a wireless network, by randomly selecting a time slot to transmit, while listening in every other time slot. Pseudo code describing the Slotted ALOHA protocol can be seen in \autoref{algo:slottedaloha}. This method of selecting time slots for transmitting will have a high probability of collisions happening, where multiple nodes transmit at the same time, which make this protocol an ideal candidate for simulating collisions and packet loss.\medbreak

%An important part of any \gls{tdma} protocol is that only a single node should transmit at a given timeslot, otherwise collisions will occur. Thus, the Slotted ALOHA protocol is a good choice for 
% TODO: Add number of arguments as parameter
\begin{algorithm}[ht]
    \DontPrintSemicolon
    \SetKwFunction{FSlottedALOHA}{Slotted ALOHA}
    \SetKwProg{Fn}{procedure}{}{}

    \Fn{\FSlottedALOHA{S}}{
        \Repeat{\textit{protocol terminates}}{
            selected $\leftarrow$ randomly select a slot $\in$ {1, 2, \dots, S}\;

            \ForEach{current $\in$ {1, 2, \dots, S}}{
                \If{select = current}{
                    duration $\leftarrow$ Broadcast(select)\;
                    Sleep(200us - duration)\;
                }
                \Else{
                    packets $\leftarrow$ Listen(200us)\;
                }
            }
        }
    }

    \caption{The Slotted ALOHA protocol~\cite{Roberts:1975:APS:1024916.1024920}.}
    \label{algo:slottedaloha}
\end{algorithm}

\todo[inline]{At some point, we should be able to create a graph to measure received packets and collisions for the Slotted ALOHA protocol.}
