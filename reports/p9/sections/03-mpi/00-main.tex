\chapter{Message Passing Interface}
To be able to simulate radio communication using \gls{mpi}, it is necessary to expose a series of functions, such that \gls{manet} nodes will be able to communicate with each other. This will be done in three parts. The first part is a set of hardware emulating functions to enable transmission and reception of data packets between nodes. This part is introduced in \autoref{sec:mpiprotocol}. The second part is a centralised controller, used to control the networking between the nodes. The controller is presented in \autoref{sec:mpicontroller}. The third part is the \acrfull{lmc}, used to continuously compute the link model. The \acrlong{lmc} is presented in \autoref{sec:mpi:lmc}. The architecture of the \acrlong{mpi} can be seen in \autoref{figure:mpi_architecture}.
%Finally, \autoref{sec:saloha} presents the Slotted ALOHA protocol, as a sample \gls{tdma} protocol that we can simulate using our \gls{mpi}. 

\begin{figure}[ht]
    \centering
    \includegraphics[width=0.8\textwidth]{figures/mpi_architecture.png}
    \caption{The architecture of the \acrlong{mpi}.}
    \label{figure:mpi_architecture}
\end{figure}

% Explain the scenario, why do we need this?, sell the concept that each of the nodes is running in a different process, in real time.
% We simulate transmissions in virtual time, but we emulate each process that we run in its own process.

\section{Hardware Emulator}\label{sec:mpiprotocol}
In this section, we will introduce the hardware functions required to emulate the radio hardware on a wireless sensor node, as well as a sample protocol that we can implement, using these functions. \autoref{sec:saloha} will present the Slotted ALOHA protocol, and \autoref{sec:hwfuncspseudo} will present the three functions, along with pseudocode descriptions of these.

\subsection{Slotted ALOHA}\label{sec:saloha}
%In this section we will present an implementation of the Slotted ALOHA protocol~\cite{Roberts:1975:APS:1024916.1024920}, using the hardware interface from \autoref{sec:hardwareinterface}. 
Slotted ALOHA~\cite{Roberts:1975:APS:1024916.1024920} is a \gls{tdma} protocol, that enable communication between nodes in a wireless network, by randomly selecting a time slot to transmit, while listening in every other time slot. Pseudocode describing the Slotted ALOHA protocol can be seen in \autoref{algo:slottedaloha}. This method of selecting time slots for transmitting will have a high probability of collisions happening, where multiple nodes transmit at the same time.\medbreak

%An important part of any \gls{tdma} protocol is that only a single node should transmit at a given timeslot, otherwise collisions will occur. Thus, the Slotted ALOHA protocol is a good choice for 
% TODO: Add number of arguments as parameter
\begin{algorithm}[ht]
    \DontPrintSemicolon
    \SetKwFunction{FSlottedALOHA}{Slotted ALOHA}
    \SetKwProg{Fn}{procedure}{}{}

    \Fn{\FSlottedALOHA{}}{
        \Repeat{\textit{protocol terminates}}{
            selected $\leftarrow$ randomly select a slot $\in$ {1, 2, \dots, S}\;

            \ForEach{current $\in$ {1, 2, \dots, S}}{
                \If{select = current}{
                    \KwTransmit\;
                }
                \Else{
                    \KwListen\;
                }
            }
        }
    }

    \caption{The Slotted ALOHA protocol~\cite{Roberts:1975:APS:1024916.1024920}.}
    \label{algo:slottedaloha}
\end{algorithm}

\todo[inline]{At some point, we should be able to create a graph to measure received packets and collisions for the Slotted ALOHA protocol.}

\subsection{Hardware Functions}\label{sec:hwfuncspseudo}
The three essential functions for hardware emulation is: broadcasting (\autoref{algo:hwfuncstransmit}), listening (\autoref{algo:hwfuncslisten}), and sleeping (\autoref{algo:hwfuncssleep}).
  
\begin{algorithm}[ht]
    \DontPrintSemicolon
    \SetKwFunction{FBroadcast}{Broadcast}
    \SetKwProg{Fn}{Function}{}{}
    
    \Fn{\FBroadcast{packet}}{
        \KwSend packet \KwTo controller\;
        \KwAwait ack \KwFrom controller\;
    }
    \caption{The Broadcast Function.}
    \label{algo:hwfuncstransmit}
\end{algorithm}

The \texttt{Broadcast} function takes any arbitrary data packet, sends this to the controller using the \gls{mpi}, and waits for an acknowledgement from the controller. The controller takes care of distributing the packet to neighbouring nodes, including computing the probability of the neighbouring node receiving the packet.

\begin{algorithm}[ht]
    \DontPrintSemicolon
    \KwResult{list of packets}
    \SetKwFunction{FListen}{Listen}
    \SetKwProg{Fn}{Function}{}{}
    
    \Fn{\FListen{time}}{
        packets $\leftarrow$ empty list\;
    
        \KwSend time \KwTo controller\;
        c $\leftarrow$ \KwAwait count \KwFrom controller\;
        \For{i $\leftarrow$ 0 \KwTo c}{
            p $\leftarrow$ \KwAwait packet \KwFrom controller\;
            \KwAppend p \KwTo packets\;
        }
        
        \KwRet packets\;
    }
    
    \caption{The Listen Function.}
    \label{algo:hwfuncslisten}
\end{algorithm}

The \texttt{Listen} function takes an amount of time units, and sends this to the controller. The controller will return any packets the listening node have received within the time interval.

\begin{algorithm}[ht]
    \DontPrintSemicolon
    \SetKwFunction{FSleep}{Sleep}
    \SetKwProg{Fn}{Function}{}{}
    
    \Fn{\FSleep{time}}{
        \KwSend time \KwTo controller\;
        \KwAwait wakeup \KwFrom controller\;
    }
    
    \caption{The Sleep Function.}
    \label{algo:hwfuncssleep}
\end{algorithm}

The \texttt{Sleep} function also takes an amount of time units, and sends this to the controller. The controller will in turn send a wakeup message to the node after the time has passed.


\begin{algorithm}[ht]
    \DontPrintSemicolon
    
    \textbf{Structure} Message \{ action, source, data, time \}\;
    \textbf{Structure} Node \{ id, state, location, time, packets \}\;
    \;
    
    nodes $\leftarrow$ map containing all nodes with id as key\;
    packets $\leftarrow$ empty map\;
    currenttime $\leftarrow$ 0\;
 
    \caption{The Shared State}
    \label{algo:mpisharedstate}
\end{algorithm}


\begin{algorithm}[ht]
    \DontPrintSemicolon
    \SetKwFunction{FMessageHandler}{MessageHandler}
    \SetKwProg{Fn}{procedure}{}{}
    

    \Fn{\FMessageHandler{}}{
        \Repeat{\textit{protocol terminates}}{
            m $\leftarrow$ \KwAwait message \KwFrom any node\;
        
            \If{m.action = transmit}{
                nodes(m.source).time $\leftarrow$ nodes(m.source).time + 1\;
                nodes(m.source).state $\leftarrow$ transmitting\;
                packets(m.source) $\leftarrow$ m.data\;
            }
            \ElseIf{m.action = listen}{
                nodes(m.source).time $\leftarrow$ nodes(m.source).time + m.time\;
                nodes(m.source).state $\leftarrow$ listening\;
            }
            \ElseIf{m.action = sleep}{
                nodes(m.source).time $\leftarrow$ nodes(m.source).time + m.time\;
                nodes(m.source).state $\leftarrow$ sleeping\;
            }
            \ElseIf{m.action = location}{
                nodes(m.source).location $\leftarrow$ m.data as location\;
            }
        }
    }
    
    \caption{The Message Handler.}
    \label{algo:mpimessagehandler}
\end{algorithm}


\begin{algorithm}[ht]
    \DontPrintSemicolon
    \SetKwFunction{FProcessor}{Controller}
    \SetKwProg{Fn}{procedure}{}{}
    
    \Fn{\FProcessor{}}{
        \Repeat{\textit{protocol terminates}}{
            \KwAwait every node $\in$ nodes, where node.time $>$ currenttime\;
            
            currenttime $\leftarrow$ currenttime + 1\;
        
            \ForEach{node $\in$ nodes}{
                \If{node.state = transmitting}{
                    %m $\leftarrow$ Message\;
                    %m.action $\leftarrow$ transmit\;
                    data $\leftarrow$ packets(node.id)\;
                    
                    \ForEach{neighbour $\in$ neighbours(node)}{
                        \If{neighbour.state = listening \textbf{and} isreachable(neighbour)}{
                            \KwAppend data \KwTo neighbour.packets\;
                        }
                    }
                    
                    \KwRemove packets(node.id)\;
                    \KwSend ack \KwTo node.id\;
                }
                \ElseIf{node.state = sleeping}{
                    \If{node.time = currenttime}{
                        \KwSend wakeup \KwTo node.id\;
                    }
                }
            }
            
            \ForEach{node $\in$ nodes}{
                \If{node.state = listening \KwAnd node.time = currenttime}{
                    \ForEach{packet $\in$ node.packets}{
                        \KwSend packet \KwTo node.id\;
                    }
                    
                    \KwClear node.packets\;
                }
            }
        
        }
    }
    
    \caption{The Controller Part.}
    \label{algo:mpicontrollerpart}
\end{algorithm}
\section{Link Model Computer}\label{sec:mpi:lmc}
% intro
To continuously compute the link model, we introduce the \gls{lmc}. The \gls{lmc} will be responsible for computing the link model and making it readily available for the centralised controller. We introduce a separate compute node, whose only task is to compute the link model, in order to offload work from the controller.
\todo[inline]{WIP, finish}
%In \autoref{sec:mpinodelocation} we mentioned that the location of all nodes should be known by the controller, 

%The reason for introducing a separate compute node, dedicated to the link model, is to offload the controller. Both handling packets transmission, synchronising time and computing the link model is too much responsibility for a single node. Offloading also allows for simplifying the logic in the controller.

%When the network is initialized, the controller will transmit the current image of the network to the \gls{lmc}. The \gls{lmc} will then compute the link model based on the received information. During the simulation, the controller can update the stored network image. This allows for removing and adding nodes to the network, while also allowing the controller to update a nodes location data, thereby keeping an updated image of the network. The \gls{lmc} will first re-compute the link model, when an update has happened. When the simulation is to be stopped, the controller will notify the \gls{lmc}, and the compute node will terminate.
\section{Hardware Interface (hardware.h)}\label{sec:hardwareinterface}
In this section we describe the design of the hardware (\gls{lowerlayer}) interface, used by an \gls{upperlayer} protocol to simulate radio communication between nodes in a \gls{manet}. The section will serve as the documentation, as well as a programmers guide, for the \mintinline{cpp}{hardware.h} interface.

%The hardware interface is implemented in modern C++, using templates, which will allow a protocol implementation to transmit instances of arbitrary structures or classes between nodes, provided that the structure or class is a trivially copyable type~\cite{website:cpptriviallycopyable}.

\begin{description}[style=nextline,leftmargin=0cm]
    \item[\mintinline{cpp}{void hardware::init(const Location &loc)}]
        Initialises the hardware functionality by initialising the \gls{mpi} functionality, as well as registering the node with the \gls{mpi} controller. The location is stored on the controller, and can later be updated by using the \mintinline{cpp}{set_location()} function. The location of a node is used to compute neighbourhood information, as well as the \gls{pathloss} experienced when transmitting data between nodes. This function has to be called exactly once, before calling any other hardware functions.
    
    \item[\mintinline{cpp}{void hardware::deinit()}]
        De-initialises the hardware functionality by un-registering the node from the \gls{mpi} controller, as well as de-initialising the \gls{mpi} functionality. This function has to be called exactly once, before terminating the protocol.
        
    \item[\mintinline{cpp}{std::chrono::microseconds}\\\mintinline{cpp}{hardware::broadcast(std::vector<unsigned char> &packet)}]
        Transmit a vector of bytes. \autoref{algo:hwfuncstransmit} contains a pseudo code description of this function. Returns the duration of the transmission, in microseconds.

    \item[\mintinline{cpp}{std::vector<unsigned char>}\\\mintinline{cpp}{hardware::listen(std::chrono::microseconds duration)}]
        Listen for data packets for a given duration of microseconds. \autoref{algo:hwfuncslisten} contains a pseudo code description of this function.
    
    \item[\mintinline{cpp}{void hardware::sleep(std::chrono::microseconds duration)}]
        Sleep for a given duration of microseconds. \autoref{algo:hwfuncssleep} contains a pseudo code description of this function.

    \item[\mintinline{cpp}{void report_localtime()}]
        Report the current local time to the controller.    
    
    \item[\mintinline{cpp}{unsigned long hardware::get_id()}]
        Gets the unique identifier of the node, assigned by the \gls{mpi} library. This function will return 0, if the \mintinline{cpp}{init_hardware()} function has not yet been called.
    
    \item[\mintinline{cpp}{unsigned long hardware::get_world_size()}]
        Gets the total amount of nodes registered to the \gls{mpi} controller. This function will return 0, if the \mintinline{cpp}{init_hardware()} function has not yet been called.
    
    \item[\mintinline{cpp}{bool hardware::set_location(const Location &loc)}]
        Updates the location registered on the \gls{mpi} controller. Returns \mintinline{cpp}{true} if the location was successfully updated, and \mintinline{cpp}{false} if the location update failed on the controller, or if the \mintinline{cpp}{init_hardware()} function has not yet been called.

\end{description}

A sample implementation of the Slotted ALOHA protocol using the hardware interface described above, can be found at \autoref{minted:cpp:slottedaloha} in \autoref{app:slottedaloha}.
