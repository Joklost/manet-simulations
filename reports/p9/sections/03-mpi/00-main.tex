\chapter{Message Passing Interface}
%In this chapter, we will introduce our \gls{mpi} library.

To be able to simulate radio communication using \gls{mpi}, it is necessary to expose a series of functions, such that \gls{manet} nodes will be able to communicate with each other. This will be done in three parts. The first part is a set of hardware emulating functions to enable transmission and reception of data packets between nodes. This part is introduced in \autoref{sec:mpiprotocol}. The second part is a centralised controller, used to control the networking between the nodes. The controller is presented in \autoref{sec:mpicontroller}. The third part is the \acrfull{lmc}, used to continuously compute the link model. The \acrlong{lmc} is presented in \autoref{sec:mpi:lmc}. Finally, \autoref{sec:saloha} presents the Slotted ALOHA protocol, as a sample \gls{tdma} protocol that we can simulate using our \gls{mpi}. The overall architecture of the system can be seen on \autoref{figure:mpi_architecture}.

\begin{figure}[ht]
    \centering
    \includegraphics[width=\textwidth]{figures/mpi_architecture.png}
    \caption{The architecture of the \gls{mpi} framework.}
    \label{figure:mpi_architecture}
\end{figure}


\section{Time}\label{sec:mpi:time}
\todo[inline]{write me}

To ensure that all nodes in the network are synchronised, tracking the time of every node is necessary. The controller described in \autoref{sec:mpicontroller} handles the synchronisation of time. 

Time is real in the framework, because the protocols that will be simulated may rely on real time. This poses a problem. Some networks sleep most the time to save energy. This creates time spans with unused resources in a simulation. To avoid this, the framework introduces virtual time. When the controller finds a time span where no node in the network is transmitting or listening, it means all nodes are either sleeping or doing some computations locally. This means that time can be forwarded to the next point in time where something will happen. This both ensures that resources will not go unused and has the potential to drastically speed up the simulations.
\input{sections/03-mpi/02-protocols.tex}
\input{sections/03-mpi/03-controller.tex}
\section{Link Model Computer (LMC)}\label{sec:mpi:lmc}
% intro
To continuously compute the link model, we introduce a \gls{lmc}. The \gls{lmc} will be responsible for computing the link model, such that the centralised controller can fetch the result.

The reason for introducing a separate compute node, dedicated to the link model, is to offload the controller. Both handling packets transmission, synchronising time and computing the link model is too much responsibility for a single node. Offloading also allows for simplifying the logic in the controller.

When the network is initialized, the controller will transmit the current image of the network to the \gls{lmc}. The \gls{lmc} will then compute the link model based on the received information. During the simulation, the controller can update the stored network image. This allows for removing and adding nodes to the network, while also allowing the controller to update a nodes location data, thereby keeping an updated image of the network. The \gls{lmc} will first re-compute the link model, when an update has happened. When the simulation is to be stopped, the controller will notify the \gls{lmc}, and the compute node will terminate.
\input{sections/03-mpi/05-hardwareinterface.tex}
