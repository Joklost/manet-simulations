\section{Hardware Emulation}\label{sec:mpiprotocol}
In this section, we introduce the hardware functions required to emulate the radio hardware on a wireless sensor node. The three essential functions for hardware emulation is: broadcasting (\autoref{algo:hwfuncstransmit}), listening (\autoref{algo:hwfuncslisten}), and sleeping (\autoref{algo:hwfuncssleep}). Additionally, a function for reporting the current local time (\autoref{algo:hwfuncsupdatelocaltime}) of the node to the controller is added. This function should be used whenever none of the above functions are applicable, and will allow the controller to continue processing requests from other nodes.

\subsection{Time}\label{sec:mpi:time}
To ensure that all nodes in the network are synchronised, we monitor the local time of every node. The controller described in \autoref{sec:mpicontroller} uses the local time of nodes to track when it is able to process whenever nodes listen for transmissions. To do this, the controller requires regular time updates. These time updates are included in the \acrshort{mpi} message sent to the controller from a node, whenever the node broadcasts, listens or sleeps. The local time of nodes are tracked in real-time, as the network protocols we simulate may rely on this. \smallbreak
As some protocols are designed to conserve power in a mobile device, nodes will spend a large amount of time sleeping, which can create time periods where nothing actually happens, during a simulation. To counteract this, we introduce virtual time. Virtual time means that the controller will be able to skip periods of inactivity, which ensures that resources will not go unused, and we are able to speed up the time required to run the simulations.

\subsection{Location}\label{sec:mpinodelocation} % gps?
Recall that in \autoref{sec:linkmodel}, we use the location of the nodes to compute the link model for the wireless network topology. As we emulate the physical hardware of a mobile device, we also need to emulate the location of the device. The current location of each node should be known by the centralised controller, and whenever the location of a node is updated, a location update should be sent to the controller.

%Time is real in the framework, because the protocols that will be simulated may rely on real time. This poses a problem. Some networks sleep most the time to save energy. 

%This creates time spans with unused resources in a simulation. To avoid this, the framework introduces virtual time. When the controller finds a time span where no node in the network is transmitting or listening, it means all nodes are either sleeping or doing some computations locally. This means that time can be forwarded to the next point in time where something will happen. This both ensures that resources will not go unused and has the potential to drastically speed up the simulations.

\subsection{Hardware Functions}\label{sec:hwfuncspseudo}
The hardware functions listed in this section depend on the following local state. The local state is unique for each node.\smallbreak

clock $\leftarrow$ \Now

localtime $\leftarrow$ 0

id $\leftarrow$ unique identifier 
\smallbreak

%as well as a sample protocol that we can implement, using these functions. 

%\autoref{sec:saloha} will present the Slotted ALOHA protocol, and \autoref{sec:hwfuncspseudo} will present the three functions, along with pseudocode descriptions of these.



%\begin{algorithm}[ht]
%    \DontPrintSemicolon
%    \SetAlgorithmName{Data Structure}{Data Structure}
%
%    \textbf{Structure} Message \{ action, source, localtime, duration, data \}\;
%    \;
%    clock $\leftarrow$ \Now\;
%    localtime $\leftarrow$ 0\;
%    id $\leftarrow$ unique identifier\;
% 
%    \caption{The shared variables and structures used by the hardware functions.}
%    \label{algo:protocolsharedstate}
%\end{algorithm}

The local state for the hardware functions consist of three variables: clock, localtime, and id. The clock variable is used to measure the real-time spent by the node between calls to the hardware function. The localtime variable is used to track the local time of the node, as well as to enable virtual time (introduced in \autoref{sec:mpi:time}). Finally, the id variable is a unique identifier assigned to each particular node. A common element for all of the hardware functions is that we initially update the localtime variable with the elapsed time since the clock variable was reset last, before transmitting the current local time, as well as a duration, to the controller. The clock variable is reset at the end of all of the functions.\medbreak

\begin{algorithm}[ht]
    \DontPrintSemicolon
    \KwResult{The time it took to broadcast the packet}
    \SetKwFunction{FBroadcast}{Broadcast}
    \SetKwProg{Fn}{Function}{}{}
    
    \Fn{\FBroadcast{packet}}{
        localtime $\leftarrow$ (\Now $-$ clock) $+$ localtime\;
        duration $\leftarrow$ transmission\_time(|packet|)\;
        m $\leftarrow$ \{ transmit, id, localtime, duration, packet \}\;
        \Send m \KwTo controller\;
        localtime $\leftarrow$ localtime + duration\;
        clock $\leftarrow$ \Now\;
        \KwRet duration\;
    }

    \caption{The Broadcast Function.}
    \label{algo:hwfuncstransmit}
\end{algorithm}

The \texttt{Broadcast} (\autoref{algo:hwfuncstransmit}) function takes any arbitrary data packet and sends this to the controller using the \gls{mpi}. The controller takes care of distributing the packet to neighbouring nodes, including computing the probability of the neighbouring node receiving the packet. The duration required for transmitting a packet is computed using the baudrate, as specified by the hardware parameters described in \autoref{sec:radio-communication}. This is sent to the controller, along with the packet, and finally we update the localtime variable with the duration, reset the clock variable and return the duration of the transmission.\medbreak

\begin{algorithm}[ht]
    \DontPrintSemicolon
    \KwResult{list of packets}
    \SetKwFunction{FListen}{Listen}
    \SetKwProg{Fn}{Function}{}{}
    
    \Fn{\FListen{duration}}{
        localtime $\leftarrow$ (\Now $-$ clock) $+$ localtime\;
        m $\leftarrow$ \{ listen, id, localtime, duration \}\;
        \Send m \KwTo controller\;

        packets $\leftarrow$ empty list\;
        c $\leftarrow$ \Await count \From controller\;
        \For{i $\leftarrow$ 0 \KwTo c}{
            p $\leftarrow$ \Await packet \From controller\;
            \Append p \KwTo packets\;
        }

        localtime $\leftarrow$ localtime + duration\;
        clock $\leftarrow$ \Now\;
        \KwRet packets\;
    }
    
    \caption{The Listen Function.}
    \label{algo:hwfuncslisten}
\end{algorithm}

The \texttt{Listen} (\autoref{algo:hwfuncslisten}) function takes a duration, and sends this, along with the updated local time, to the controller. The controller will return any packets the listening node have received within the duration.\medbreak

\begin{algorithm}[ht]
    \DontPrintSemicolon
    \SetKwFunction{FSleep}{Sleep}
    \SetKwProg{Fn}{Function}{}{}
    
    \Fn{\FSleep{duration}}{
        localtime $\leftarrow$ (\Now $-$ clock) $+$ localtime\;
        m $\leftarrow$ \{ sleep, id, localtime, duration \}\;
        \Send m \KwTo controller\;

        localtime $\leftarrow$ localtime + duration\;
        clock $\leftarrow$ \Now\;
    }
    
    \caption{The Sleep Function.}
    \label{algo:hwfuncssleep}
\end{algorithm}

The \texttt{Sleep} (\autoref{algo:hwfuncssleep}) function also takes an duration, and sends this, along with the updated local time to the controller.\medbreak

\begin{algorithm}[ht]
    \DontPrintSemicolon
    \SetKwFunction{FReportLocalTime}{ReportLocalTime}
    \SetKwProg{Fn}{Function}{}{}
    
    \Fn{\FReportLocalTime{}}{
        localtime $\leftarrow$ (\Now $-$ clock) $+$ localtime\;
        m $\leftarrow$ \{ report-localtime, id, localtime, 0 \}\;
        \Send m \KwTo controller\;
        clock $\leftarrow$ \Now\;
    }
    
    \caption{The ReportLocaltime Function.}
    \label{algo:hwfuncsupdatelocaltime}
\end{algorithm}

Finally, the \texttt{ReportLocalTime} (\autoref{algo:hwfuncsupdatelocaltime}) function is included in the case where none of the above hardware functions are applicable. The function will send a message to the controller with the updated local time, and reset the \texttt{clock} variable.