\section{Hardware Interface (hardware.h)}\label{sec:hardwareinterface}
In this section we describe the design of the hardware (\gls{lowerlayer}) interface, used by an \gls{upperlayer} protocol to simulate radio communication between nodes in a \gls{manet}. The section will serve as the documentation, as well as a programmers guide, for the \mintinline{cpp}{hardware.h} interface.

%The hardware interface is implemented in modern C++, using templates, which will allow a protocol implementation to transmit instances of arbitrary structures or classes between nodes, provided that the structure or class is a trivially copyable type~\cite{website:cpptriviallycopyable}.

\begin{description}[style=nextline,leftmargin=0cm]
    \item[\mintinline{cpp}{void hardware::init(const Location &loc)}]
        Initialises the hardware functionality by initialising the \gls{mpi} functionality, as well as registering the node with the \gls{mpi} controller. The location is stored on the controller, and can later be updated by using the \mintinline{cpp}{set_location()} function. The location of a node is used to compute neighbourhood information, as well as the \gls{pathloss} experienced when transmitting data between nodes. This function has to be called exactly once, before calling any other hardware functions.
    
    \item[\mintinline{cpp}{void hardware::deinit()}]
        De-initialises the hardware functionality by un-registering the node from the \gls{mpi} controller, as well as de-initialising the \gls{mpi} functionality. This function has to be called exactly once, before terminating the protocol.
        
    \item[\mintinline{cpp}{std::chrono::microseconds}\\\mintinline{cpp}{hardware::broadcast(std::vector<unsigned char> &packet)}]
        Transmit a vector of bytes. \autoref{algo:hwfuncstransmit} contains a pseudo code description of this function. Returns the duration of the transmission, in microseconds.

    \item[\mintinline{cpp}{std::vector<unsigned char>}\\\mintinline{cpp}{hardware::listen(std::chrono::microseconds duration)}]
        Listen for data packets for a given duration of microseconds. \autoref{algo:hwfuncslisten} contains a pseudo code description of this function.
    
    \item[\mintinline{cpp}{void hardware::sleep(std::chrono::microseconds duration)}]
        Sleep for a given duration of microseconds. \autoref{algo:hwfuncssleep} contains a pseudo code description of this function.

    \item[\mintinline{cpp}{void report_localtime()}]
        Report the current local time to the controller.    
    
    \item[\mintinline{cpp}{unsigned long hardware::get_id()}]
        Gets the unique identifier of the node, assigned by the \gls{mpi} library. This function will return 0, if the \mintinline{cpp}{init_hardware()} function has not yet been called.
    
    \item[\mintinline{cpp}{unsigned long hardware::get_world_size()}]
        Gets the total amount of nodes registered to the \gls{mpi} controller. This function will return 0, if the \mintinline{cpp}{init_hardware()} function has not yet been called.
    
    \item[\mintinline{cpp}{bool hardware::set_location(const Location &loc)}]
        Updates the location registered on the \gls{mpi} controller. Returns \mintinline{cpp}{true} if the location was successfully updated, and \mintinline{cpp}{false} if the location update failed on the controller, or if the \mintinline{cpp}{init_hardware()} function has not yet been called.

\end{description}

A sample implementation of the Slotted ALOHA protocol using the hardware interface described above, can be found at \autoref{minted:cpp:slottedaloha} in \autoref{app:slottedaloha}.