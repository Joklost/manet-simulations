\chapter{Introduction}
% - motivation / aim
% - Why do the project?
Verifying that \gls{manet} protocols work as intended, is essential if the protocol is to be used in a real world setting, especially in critically important settings. Testing a \gls{manet} protocol with many nodes and emulating the setting the protocol will be used in can be problematic, as the setting might be hard to recreate in a way that makes the collection of data possible or creating a large enough network.\medbreak
% tickle some more
% - Run the actual software, in a realistic setting


This gives reason to develop a framework, capable of simulating arbitrary \gls{manet} protocols. Designing the framework to run on a cluster will allow for the usage of \gls{mpi}, thus increasing the possible scale and speed of the simulation.
Since the protocols are of the wireless nature, we need a way to simulate the probability of loosing a packet or the packet containing errors. In other words calculating the \gls{pep}.

At the time of writing, there exists several well-established \gls{mpi} implementations for several languages. Our framework will utilise one of those well-established implementations and add a layer on top of that, where the \gls{pep} calculations will happen.

Lastly we will test the framework on an \gls{manet} protocol, on the cluster available at Aalborg University, to verify that the framework works on a real world protocol.