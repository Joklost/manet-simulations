\section{Link Path Loss}\label{sec:linkmodel}

In this section, we will present a method for simulating \gls{manet} link \gls{pathloss} from \cite{paper:linkmodel}. For our simulations, we would like to be able to simulate the performance of a \gls{manet}. The performance is, however, heavily dependent on network conditions and the capabilities of the technology~\cite[p.~10]{paper:linkmodel}. The author of \cite{paper:linkmodel} presents methods for evaluating the performance of a wireless networks, and proceeds to introduce methods for simulating \gls{pathloss} on a multi-link model, based on a real-world performance measurement. \autoref{sec:pathloss} will summarise the method for simulating \gls{pathloss} on a \gls{manet}, and \autoref{sec:simulatingvalues} present the method, as well as present an example of simulating the \gls{pathloss} on a network. Note that the actual values used in this Section, is specific to the Reachidevices, and are based on on-site measurements.
\todo[inline]{Introduce Reachi earlier in the report.} 

\subsection{Units}
In the following sections, we will be using different units of measurements. For distances, all measurements (e.g., the distance between two nodes) will be in meters. We describe the strength of a radio signal using \acrfull{db}~\cite{website:isadbdbm}, and the transmission power, as well as the \gls{rssi}, as \acrfull{dbm}~\cite{website:isadbdbm}.

\subsection{Path Loss}\label{sec:pathloss}
The \gls{pathloss} of a link is dependent on the distance between transmitter and receiver, as well as a stochastic shadow fading term. The \gls{pathloss} vector \vect{l_{pl}} is the sum of two parts:

%\vect{l_{pl}} = \vect{l_d} + \vect{l_{fading}}

\begin{description}[style=nextline]
    \item[$\vect{l_d}$] A deterministic distance dependent part, which describes the mean signal attenuation at any given link distance (distance between transmitter and receiver), in \gls{db}. It is a vector of size $n$, where $n$ is the number of links in the network.
    \item[$\vect{l_{fading}}$] A stochastic slow/shadow fading part, which specifies the local mean of the signal, in \gls{db}. The slow fading variable is, more or less, constant in the same local area, as it is caused by terrain, buildings, vegetation, and cars. It is a vector of size $n$, where $n$ is the number of links in the network.
\end{description}
%\begin{eq}\label{eq:pathlossdb}
%    \vect{PL_{dB}} = \vect{PL_{determ,dB}(d)} + \vect{PL_{fading,dB}}
%\end{eq}

\begin{eq}\label{eq:pathlossdb}
    \vect{l_{pl}} = \vect{l_d} + \vect{l_{fading}}
\end{eq}

The deterministic distance dependent \gls{pathloss} is an independent value, while the stochastic slow/shadow fading value depends on physical objects and terrain. This means that links existing in the same physical environment should have a similar slow fading. The similarity is modelled by introducing a correlation between the slow fading on different links:

\begin{description}
    \item[Cross correlation (interlink)] describes the correlation between two or more links in the same spatial (physical) environment (\textbf{spatial correlation}).
    \item[Auto correlation (intralink)] describes the correlation in the development of slow fading for a single link over time (\textbf{temporal correlation}).
\end{description}

The \gls{pathloss} can be used, along with the transmission power in \acrshort{dbm} to simulate the \gls{rssi} on a link.
%\begin{eq}\label{eq:computerssi}
%    RSSI_{dBm} = P_{tx,dBm} - PL_{dB}
%\end{eq}

\subsection{Simulating Link Path Loss}\label{sec:simulatingvalues}
In this Section, we will summarise how \gls{pathloss} and \gls{rssi} values are simulated for a \gls{manet} according to \cite{paper:linkmodel}, and we will present how we expect to use this in our own simulations. Again, note that the values used throughout this section are based on on-site measurements made with the Reachi devices in the Philippines and are heavily dependent on the environment, as described in \cite{paper:linkmodel}. \medbreak

%With a \gls{manet} consisting of $N$ nodes, we have the quadratic link matrix, $L$.
%\begin{eq}
%    l'_{i,j} = l'_{j,i} | i,j = \{0, 1, \ldots , N \}
%\end{eq}

With a \gls{manet} consisting of $N$ nodes, we have the link matrix $\textbf{L}$.

\begin{eq}
    \textbf{L} = 
    \begin{bmatrix}
        0 & l'_{1,2} & l'_{1,3} & \dots & l'_{1,N} \\
        l'_{2,1} & 0 & l'_{2,3} & \dots & l'_{2,N} \\
        l'_{3,1} & l'_{3,2} & 0 & \dots & l'_{3,N} \\
        \vdots & \vdots & \vdots & \vdots & \vdots \\
        l'_{N,1} & l'_{N,2} & l'_{N,3} & \dots & 0 \\
\end{bmatrix}
\end{eq}

Links are undirected and show equal loss in both directions. The diagonal $l'_{i, i}$ is zero, as there is no link from a node to itself. From the link matrix $\textbf{L}$, we can identify the unique links in the link matrix as the vector \vect{l}, that contains all unique links in the link matrix; the elements of the upper triangle of the link matrix, excluding the diagonal.

\begin{eq}\label{eq:uniquelinkvec}
    \vect{l} =
    \begin{bmatrix}
        l'_{1,2} & l'_{1,3} & \dots & l'_{1,N} & l'_{2,3} & l'_{2,4} & \dots & l'_{2,N} & \dots l'_{N-1,N}
    \end{bmatrix}^T
\end{eq}

The length of the unique link vector is denoted by the function $len(\vect{l})$.

\begin{eq}\label{eq:lengthoflinks}
    len(\vect{l}) = \sum\limits_{i=1}^{k=N-1} i = \frac{N(N+1)}{2} - N
\end{eq}
%This link matrix describe the path loss in \gls{db} on the link between node $i$ and $j$. Links are undirected, and show equal loss in both directions. The diagonal $l'_{i,i}$ is zero, as there are no link from a node to itself. To describe the \gls{pathloss} on each link and the correlation between links, we identify $l$ as the vector of unique links in the link matrix. 

\begin{figure}[H]
    \centering
    \begin{tikzpicture}
        %%
        \begin{scope}[xshift=4cm]
        \node[main node] (1) {$1$};
        \node[main node] (2) [left = 2cm  of 1]  {$2$};
        \node[main node] (3) [below = 2cm  of 2] {$3$};
        \node[main node] (4) [right = 2cm  of 3] {$4$};

        \path[draw,thick]
        (1) edge node[above] {$l_1$} (2)
        (1) edge node[above=.8cm, right] {$l_2$} (3)
        (1) edge node[right] {$l_3$} (4)
        (2) edge node[left] {$l_4$} (3)
        (2) edge node[below=.8cm, right] {$l_5$} (4)
        (3) edge node[below] {$l_6$} (4)
        ;
        \end{scope}
    \end{tikzpicture}
    \caption{Sample graph \textbf{G} with 4 nodes and 6 links.}
    \label{figure:lm-sample}
\end{figure}

\autoref{figure:lm-sample} is a sample network graph \textbf{G} consisting of 4 nodes, with 6 unique links. In the sample, the links $l_1$ and $l_4$ are both 100 meters long, and the angle between them is 90\textdegree. The length of a $l$ link will be denoted by the function $d(l)$, and the angle between two unique links, $k$ and $l$, will be denoted by the function $\theta(k,l)$, where $k$ and $l$ are unique links. The graph \textbf{G} would be equivalent to the following link vector:

\begin{eq}\label{eq:uniquelinkvec}
    \vect{l_\textbf{G}} =
    \begin{bmatrix}
        l_1 & l_2 & l_3 & l_4 & l_5 & l_6
    \end{bmatrix}^T
\end{eq}

%With this, we can compute the \gls{pathloss} of each unique link as the sum of the distance dependent part and the slow shadow fading part.

%\begin{eq}\label{eq:pathlosslink}
%    \vect{l_{pl}} = \vect{l_d} + \vect{l_{fading}}
%\end{eq}

First, we compute the distance dependent part of the path loss. The deterministic distance dependent part \vect{l_d} is obtained for each unique link in the network by:
\begin{eq}\label{eq:pathlossdeterm}
    \vect{l_d} = 
        \begin{bmatrix}
            10 \gamma \log_{10} \left( d(l_1) \right) - c\\
            10 \gamma \log_{10} \left( d(l_2) \right) - c \\
            \vdots \\
            10 \gamma \log_{10} \left( d(l_n) \right) - c\\
        \end{bmatrix}
\end{eq}
where the \gls{pathloss} exponent $\gamma = 5.5$, the constant offset $c = -18.8$, $d(l_i)$ is the distance between the two nodes of the link, in meters, and $n = len(\vect{l})$. The \gls{pathloss} exponent is obtained by \gls{tls} regression in \cite{paper:linkmodel}, and the constant offset is the signal strength of a link with $d(l) = 1$, in \gls{db}. \medbreak

For our sample network \textbf{G}, we can compute the distance dependent part as follows:
\begin{eq}\label{eq:pathlossdetermG}
    \vect{l_{d, \textbf{G}}} = 
        \begin{bmatrix}
            55 \log_{10} \left( d(l_1) \right) - 18\\
            55 \log_{10} \left( d(l_2) \right) - 18\\
            \vdots \\
            55 \log_{10} \left( d(l_6) \right) - 18\\
        \end{bmatrix}
        =
        \begin{bmatrix}
            92\\
            100.2\\
            \vdots \\
            92\\
        \end{bmatrix}
\end{eq} \medbreak

With the distance dependent part computed, we can move on to the stochastic slow fading part. This part is a bit more complicated, as it is not an independent value like the distance dependent part. Instead, the values depend on the correlation between links, and as such, we need the correlations presented in \autoref{sec:pathloss}. First, we introduce the spatial correlation. $\textbf{C}$ is the correlation matrix determining the correlation coefficient between links. $\textbf{C}$ is a quadratic matrix of size $M \times M$, where $M = len(\vect{l})$.

%\begin{eq}\label{eq:nodesfunc}
%    nodes(l) = \left\{n_i, n_j\right\}
%\end{eq}

\begin{eq}\label{eq:correlationmatrix}
    \textbf{C} = 
    \begin{cases} 
        r_{k,l} \left( \theta(k,l) \right) & \text{if} \  k \neq l \ \text{and} \  nodes(k) \cap nodes(l) \neq \emptyset \\
        1 & \text{if} \ k = l \\
        0 & \text{otherwise}
    \end{cases} 
\end{eq}

$\theta(k,l)$ is the angle between links $k$ and $l$, $nodes(l)$ is a function that returns the set of nodes of a link $l$, and $r_{k,l} \left( \theta(k,l) \right)$ is an auto-correlation function, and it is parameterised based on the Reachi measurements in \cite{paper:linkmodel}.

\begin{eq}\label{eq:pathlossautocorrelation}
    r_{k,l}\left( \theta(k,l) \right) = 0.595e^{-0.064 * \theta(k,l)} + 0.092
\end{eq}

With this, we can generate the correlation matrix for our sample graph \textbf{G} as:

\begin{eq}
    \textbf{C}_{\textbf{G}} = 
    \begin{bmatrix}
        1     & 0.125 & 0.094 & 0.094 & 0.125 & 0     \\
        0.125 & 1     & 0.125 & 0.125 & 0     & 0.125 \\
        0.094 & 0.125 & 1     & 0     & 0.125 & 0.094 \\
        0.094 & 0.125 & 0     & 1     & 0.125 & 0.094 \\
        0.125 & 0     & 0.125 & 0.125 & 1     & 0.125 \\
        0     & 0.125 & 0.094 & 0.094 & 0.125 & 1     \\
\end{bmatrix}
\end{eq}

The correlation matrix $\textbf{C}$ is used to create the covariance matrix $\boldsymbol{\Sigma} = \sigma^2\textbf{C}$ by multiplying the correlation matrix with a standard deviation of $\sigma = 11.4$ \gls{db} squared. Next, we draw an \gls{iid} multivariate Gaussian vector, with the standard deviation $I = 1$.

\begin{eq}\label{eq:pathlossnormaldist}
    \vect{x} =  \big[ x_1 \  x_2 \  \ldots \  x_{len(\vect{l})} \big]^T \sim N(0, I) 
\end{eq}

The independent variables in the vector are made dependent by multiplication with the lower triangular Cholesky decomposition~\cite[p. 143]{Golub:1996:MC:248979} (\autoref{algo:cholesky}) of the covariance matrix $\textbf{Q} = cholesky\left(\boldsymbol{\Sigma}\right)$.

\begin{eq}\label{eq:pathlossstoch}
    \vect{l_{fading}} = \textbf{Q}\vect{x}
\end{eq}

With this, it is possible for us to compute a single realisation of the \gls{pathloss} of a link matrix, accounting for the distance dependent \gls{pathloss} and the spatial correlation. For our sample graph \textbf{G}, this could be:

\begin{eq}\label{eq:pathlossfadingG}
    \vect{l_{fading, \textbf{G}}} = 
        \textbf{Q}_{\textbf{G}} \cdot \vect{x}
        =
        \begin{bmatrix}
            -5.79\\
            -4.34\\
            \vdots \\
            10.04\\
        \end{bmatrix}
\end{eq} \medbreak

Note that as $\vect{x}$ is stochastic, the values in \autoref{eq:pathlossfadingG} are a random realisation of the stochastic shadow fading part. \medbreak

Next, the slow shadow fading part is expanded to include the temporal correlation. $l_{fading}\left(t\right)$ is the vector describing the shadow fading \gls{pathloss} at time $t$. As the distance dependent part is time invariant, it needs no further modifications.

\begin{eq}\label{eq:pathlosstemporal}
    \vect{l_{fading}}(t + \Delta t) = \overbrace{\textbf{Q}(t + \Delta t)\vect{x}}^{spatial correlation} \overbrace{\sqrt{1 - \rho_{\Delta t}} + \vect{l_{fading}}(t)\rho_{\Delta t}}^{temporal correlation}
\end{eq}

The temporal correlation is computed based on the temporal correlation coefficient, $\rho_{\Delta t}$, describing the correlation after both transmitter and receiver have moved $|d_t|$ and $|d_r|$ meters, respectively, and with a decorrelation distance of 20 meters.

\begin{eq}
    \rho_{\Delta t} = e^{-\frac{|d_t|+|d_r|}{20}\ln (2)}
\end{eq}

Again, coming back to our sample graph \textbf{G}, and assuming that all devices would have moved 10 meters in the same direction (preserving angles and distances between links), the stochastic shadow fading part at $t = 1$ would determined by:

\begin{eq}\label{eq:pathlossfadingGtemporal}
    \vect{l_{fading, \textbf{G}}}(t=1) = 
        \textbf{Q}_{\textbf{G}} \cdot \vect{x} \cdot \sqrt{1 - \rho_{\Delta t}} + \vect{l_{fading,\textbf{G}}}(t=0) \cdot \rho_{\Delta t}
\end{eq} \medbreak

where $\rho_{\Delta t} = e^{-\frac{10+10}{20}\ln (2)} = 0.5$, $\vect{x}$ would be a new realisation of the Gaussian vector, and $\textbf{Q}_{\textbf{G}}$ would be recomputed based on the new locations of the nodes, should the angles between the nodes have changed. \medbreak

With this, the vectors \vect{l_d} and \vect{l_{fading}} can be combined as in \autoref{eq:pathlossdb}, to generate the \gls{pathloss} vector $\vect{l_{pl}}$ for all unique links in the network.

\begin{eq}\label{eq:pathlosslink}
    \vect{l_{pl,\textbf{G}}}(t) = \vect{l_{d,\textbf{G}}} + \vect{l_{fading,\textbf{G}}}(t)
\end{eq}

All that remains is to subtract the \gls{pathloss} for a particular link $l$, from the transmission power of the transmitting node, to compute the \gls{rssi} in \acrshort{dbm}. For example, if a node in our sample graph \textbf{G} transmits with a transmission power of $26$ \acrshort{dbm}, the \gls{rssi} on the receiving node would be:


%$P_{tx}(l)$ is a function returning the transmission power of the transmitting node in a given link.

\begin{eq}\label{eq:rssidbm}
    RSSI_{dBm}(l_1, t = 0) = 26 \ \text{dBm} - l_{pl,\textbf{G},l_1}(t) = {-60.21} \ \text{dBm}
\end{eq}

%\begin{eq}\label{eq:pathlosslink}
%    \vect{l_{pl}}(t) = \vect{l_d} + \vect{l_{fading}}(t)
%\end{eq}

%\begin{eq}\label{eq:pathlossfadingG}
%    \vect{l_{fading, \textbf{G}}} = 
%        \textbf{Q}_{\textbf{G}} \cdot \vect{x}
%        =
%        \begin{bmatrix}
%            -5.79\\
%            -4.34\\
%            \vdots \\
%            10.04\\
%        \end{bmatrix}
%\end{eq} \medbreak


\subsection{Packet Error Probability}
Now that we are able to simulate the \gls{rssi} when transmitting on from one node to another, we need to be able to simulate whether the packet should arrive on the receiving node. With radio communication, there is a chance that a receiving node may not receive the entire packet correctly, which is why we need some way of computing the probability of this happening. For this, we introduce \acrfull{pep}. The computations in this section are derived from \cite{massoud2007digital}, as well as personal communication with the author of \cite{paper:linkmodel}. \medbreak

To begin we calculate the noise power $P_{N,db}$ in \acrshort{db}. The noise power is calculated with the thermal noise and noise figure. For the Reachi devices we assume that the $thermal\_noise = -119.66$ and the $noise\_figure = 4.2$.

\begin{eq}\label{eq:noisepower}
    P_{N,dB} = thermal\_noise + noise\_figure
\end{eq}

Next we calculate the \gls{snr} $\gamma_{dB}$ in \acrshort{db}. The \gls{snr} is the ratio between signal and noise, that compares the level of the signal to the level of the background noise, and is computed by subtracting the noise power computed in \autoref{eq:noisepower}. We expand upon this further to include the interference of other transmissions happening at the same time. This is done by computing the \gls{rssi} from interfering transmitters, and subtracting this from the \gls{snr}, which gives us the \acrlong{snir}:

\todo[inline]{add interference to equation}
\begin{eq}
    \gamma_{dB} = RSSI_{dBm} - P_{N,dB} - \text{interference}
\end{eq}

We use this to compute the bit error probability:

%\begin{eq}
%    \gamma = 10^{\frac{\gamma_{dB}}{10}}
%\end{eq}

\begin{eq}
    P_b = \frac{1}{2}erfc \left( \sqrt{ \left( \frac{10^{\frac{\gamma_{dB}}{10}}}{2} \right)} \right)
\end{eq}

which we finally can use to compute the \acrlong{pep}:

%\begin{eq}
%    P_b = \frac{1}{2}erfc \left( \sqrt{ \left( \frac{\gamma}{2} \right)} \right)
%\end{eq}

\begin{eq}
    P_p = 1 - \left( 1 - P_b \right) ^{packet\_size}
\end{eq}



If we assume that the \gls{rssi} between a transmitter and a receiver is $-105.3$ \acrshort{dbm}, with no outside interference, the probability that a transmission of 20 bytes would be unsuccessful is:

\begin{eq}
    \gamma_{dB} = -105.3 - (-119.66 + 4.2) = 10.16
\end{eq}

\begin{eq}
    P_b = \frac{1}{2}erfc \left( \sqrt{ \left( \frac{10^{\frac{\gamma_{dB}}{10}}}{2} \right)} \right) = 6.386 \cdot 10^{-4}
\end{eq}

\begin{eq}
    P_p = 1 - \left( 1 - P_b \right) ^{20 \cdot 8} = 0.097
\end{eq}

Which equals to a probability of approximately 10 \% packet loss, with no interference, and an \gls{rssi} of $-105.3$ \acrshort{dbm}.
