\chapter{Link Modelling}\label{ch:linkmodel}
The goal of this chapter is to be able to estimate the probability, based on the topology of a wireless network, of how likely it is that packet loss will happen during a transmission between two nodes. \autoref{sec:linkmodel} presents the method for simulating the link model in a \gls{manet}, \autoref{sec:pep} presents the method for computing the packet loss probability during a transmission, and \autoref{sec:optimization} proposes two ways for optimising the computational time required to compute the link model.


\begin{figure}[ht]
    \centering
    \includegraphics[width=.7\textwidth]{figures/manet_with_terrain.png}
    \caption{A sample wireless network topology.}
    \label{figure:manetwithter}
\end{figure}

\autoref{figure:manetwithter} shows a sample wireless network topology for a \acrfull{manet}. The network consists of mobile devices (nodes), and the communication between these (links). Nodes are \doublequote{linked} with other nodes when they are able to communicate wirelessly. Wireless communication relies on the transmission and reception of electromagnetic waves~\cite[p.~10]{paper:linkmodel} (radio signals), and the strength, or quality, of a wireless link is described by the signal loss occurring when propagating the signal from transmitter to receiver, and is measured by the \gls{rssi} (a negative value, where a higher value is better). In \cite{paper:linkmodel}, the term \gls{pathloss} is used to describe this signal loss, and is determined in part by the physical distance between transmitter and receiver, but also by physical objects and terrain, like buildings or forests. \autoref{sec:linkmodel} will elaborate further on the \gls{pathloss} of a link, as well as present a method for simulating the \gls{pathloss} for a mobile network topology. Another major consideration for mobile networks is that the topology is dynamic. Nodes move around, causing existing links to disappear, or new links to appear, thus changing the topology of the network.

%Whether nodes are able to communicate wirelessly with each other depends on the \gls{pathloss} 

%In \autoref{figure:manetwithter} we see a topology of six nodes, with seven communication links between them.
%When working with wireless mobile networks, it is important keep in mind that the topology these networks are dynamic. Nodes move, causing the topology to change. When nodes move in or out of reach of other nodes, links will appear or disappear dynamically.

%to keep two things in mind: First, the topology a mobile network is, per definition, dynamic.

\section{Link Path Loss}\label{sec:linkmodel}
In this section, we present the method for simulating link \gls{pathloss} (the link model) from \cite{paper:linkmodel}. We want to simulate the performance of nodes in a \gls{manet}. The performance is, however, heavily dependent on network conditions and the capabilities of the technology~\cite[p.~10]{paper:linkmodel}. The author of \cite{paper:linkmodel} presents methods for evaluating the performance of a wireless networks, and proceeds to introduce methods for simulating \gls{pathloss} on a multi-link model, based on a real-world performance measurement. \autoref{sec:pathloss} will summarise the method for simulating \gls{pathloss} on a \gls{manet}, and \autoref{sec:simulatingvalues} present the method, as well as an example of simulating the \gls{pathloss} on a network topology.% Note that the actual values used in this Section, is specific to the Reachi devices, and are based on on-site measurements. 

\subsection{Units}
In the following sections, we use different units of measurements. For distances, all measurements (e.g., the distance between two nodes) will be in meters. We describe the strength of a radio signal using \acrshort{db}~\cite{website:isadbdbm}, and the transmission power, as well as the \gls{rssi} as \acrshort{dbm}~\cite{website:isadbdbm}. The \gls{rssi} is on a logarithmic scale, and the closer the value is to 0 the better.

\subsection{Path Loss}\label{sec:pathloss}
The \gls{pathloss} of a link is dependent on the distance between transmitter and receiver, as well as a stochastic shadow fading term. The \gls{pathloss} vector \vect{l_{pl}} is the sum of two parts:

\begin{description}[style=nextline]
    \item[$\vect{l_d}$] A deterministic distance dependent part, which describes the mean signal attenuation at any given link distance (distance between transmitter and receiver), in \gls{db}. It is a vector of size $n$, where $n$ is the number of links in the network.
    \item[$\vect{l_{fading}}$] A stochastic slow/shadow fading part, which specifies the local mean of the signal, in \gls{db}. The slow fading variable is, more or less, constant in the same local area, as it is caused by terrain, buildings, vegetation, and cars. It is a vector of size $n$, where $n$ is the number of links in the network.
\end{description}

\begin{eq}\label{eq:pathlossdb}
    \vect{l_{pl}} = \vect{l_d} + \vect{l_{fading}}
\end{eq}

The deterministic distance dependent \gls{pathloss} is an independent value, while the stochastic slow/shadow fading value depends on physical objects and terrain. This means that links existing in the same physical environment should have a similar slow fading. The similarity is modelled by introducing a correlation between the slow fading on different links:

\begin{description}
    \item[Cross correlation (interlink)] describes the correlation between two or more links in the same spatial (physical) environment (\textbf{spatial correlation}) \cite[p.~16]{paper:linkmodel}.
    \item[Auto correlation (intralink)] describes the correlation in the development of slow fading for a single link over time (\textbf{temporal correlation}) \cite[p.~15]{paper:linkmodel}.
\end{description}

The \gls{pathloss} can be used, along with the transmission power in \acrshort{dbm} to simulate the \gls{rssi} on a link.

\subsection{Simulating Link Path Loss}\label{sec:simulatingvalues}
In this section, we summarise how \gls{pathloss} and \gls{rssi} values are simulated for a \gls{manet} according to \cite{paper:linkmodel}, and we present how we expect to use this in our own simulations. Again, note that the values used throughout this section are based on on-site measurements made with the Reachi devices in the Philippines and are heavily dependent on the environment, as described in \cite{paper:linkmodel}. \medbreak

%With a \gls{manet} consisting of $N$ nodes, we have the link matrix $\textbf{L}$.

%\begin{eq}
%    \textbf{L} = 
%    \begin{bmatrix}
%        0 & l'_{1,2} & l'_{1,3} & \dots & l'_{1,N} \\
%        l'_{2,1} & 0 & l'_{2,3} & \dots & l'_{2,N} \\
%        l'_{3,1} & l'_{3,2} & 0 & \dots & l'_{3,N} \\
%        \vdots & \vdots & \vdots & \vdots & \vdots \\
%        l'_{N,1} & l'_{N,2} & l'_{N,3} & \dots & 0 \\
%\end{bmatrix}
%\end{eq}
%
%Links are undirected and show equal loss in both directions. The diagonal $l'_{i, i}$ is zero, as there is no link from a node to itself. From the link matrix $\textbf{L}$, we can identify the unique links in the link matrix as the vector \vect{l}, that contains all unique links in the link matrix; the elements of the upper triangle of the link matrix, excluding the diagonal.
%
%\begin{eq}\label{eq:uniquelinkvec}
%    \vect{l} =
%    \begin{bmatrix}
%        l'_{1,2} & l'_{1,3} & \dots & l'_{1,N} & l'_{2,3} & l'_{2,4} & \dots & l'_{2,N} & \dots l'_{N-1,N}
%    \end{bmatrix}^T
%\end{eq}
%

\begin{figure}[H]
    \centering
    \begin{tikzpicture}
        \begin{scope}[xshift=4cm]
        \node[main node] (1) {$1$};
        \node[main node] (2) [left = 2cm  of 1]  {$2$};
        \node[main node] (3) [below = 2cm  of 2] {$3$};
        \node[main node] (4) [right = 2cm  of 3] {$4$};

        \path[draw,thick]
        (1) edge node[above] {$l_1$} (2)
        (1) edge node[above=.8cm, right] {$l_2$} (3)
        (1) edge node[right] {$l_3$} (4)
        (2) edge node[left] {$l_4$} (3)
        (2) edge node[below=.8cm, right] {$l_5$} (4)
        (3) edge node[below] {$l_6$} (4)
        ;
        \end{scope}
    \end{tikzpicture}
    \caption{Sample graph \textbf{G} with 4 nodes and 6 links.}
    \label{figure:lm-sample}
\end{figure}

\autoref{figure:lm-sample} is a sample network topology \textbf{G} consisting of 4 nodes, with 6 unique links. In the sample, the links $l_1$ and $l_4$ are both 100 meters long, and the angle between them is 90\textdegree. Links are undirected and show equal loss in both directions. The length of a $l$ link will be denoted by the function $d(l)$, $nodes(l)$ is a function that return the set of nodes of a link $l$, $\textbf{N}$ is the set of nodes in the network topology, and finally, the angle between two unique links, $k$ and $l$, will be denoted by the function $\theta(k,l)$, where $k$ and $l$ are unique links that share a common node ($ k \neq l \ \text{and} \  nodes(k) \cap nodes(l) \neq \emptyset $). \smallbreak
The graph \textbf{G} would be equivalent to the following link vector:

\begin{eq}\label{eq:uniquelinkvecG}
    \vect{l_\textbf{G}} =
    \begin{bmatrix}
        l_1 & l_2 & l_3 & l_4 & l_5 & l_6
    \end{bmatrix}^T
\end{eq}

The length of the unique link vector of a fully connected network is denoted by the function $len(\vect{l})$.

\begin{eq}\label{eq:lengthoflinks}
    len(\vect{l}) = \sum\limits_{i=1}^{|\textbf{N}|-1} i = \frac{|\textbf{N}|(|\textbf{N}|+1)}{2} - |\textbf{N}|
\end{eq}

First, we compute the distance dependent part of the path loss. The deterministic distance dependent part \vect{l_d} is obtained for each unique link in the network by:
\begin{eq}\label{eq:pathlossdeterm}
    \vect{l_d} = 
        \begin{bmatrix}
            10 \gamma \log_{10} \left( d(l_1) \right) - c\\
            10 \gamma \log_{10} \left( d(l_2) \right) - c \\
            \vdots \\
            10 \gamma \log_{10} \left( d(l_n) \right) - c\\
        \end{bmatrix}
\end{eq}

where the \gls{pathloss} exponent $\gamma = 5.5$, the constant offset $c = -18.8$, $d(l_i)$ is the distance between the two nodes of the link, in meters. The \gls{pathloss} exponent is obtained by \gls{tls} regression in \cite{paper:linkmodel}, and the constant offset is the signal strength of a link with $d(l) = 1$, in \gls{db}. \medbreak

For our sample network \textbf{G}, we can compute the distance dependent part as follows:
\begin{eq}\label{eq:pathlossdetermG}
    \vect{l_{d, \textbf{G}}} = 
        \begin{bmatrix}
            55 \log_{10} \left( d(l_1) \right) - 18\\
            55 \log_{10} \left( d(l_2) \right) - 18\\
            \vdots \\
            55 \log_{10} \left( d(l_6) \right) - 18\\
        \end{bmatrix}
        =
        \begin{bmatrix}
            92\\
            100.2\\
            \vdots \\
            92\\
        \end{bmatrix}
\end{eq} \medbreak

With the distance dependent part computed, we can move on to the stochastic slow fading part. This part is a bit more complicated, as it is not an independent value like the distance dependent part. Instead, the values depend on the correlation between links, and as such, we need the correlations presented in \autoref{sec:pathloss}. First, we introduce the spatial correlation. $\textbf{C}$ is the correlation matrix determining the correlation coefficient between links. $\textbf{C}$ is a quadratic symmetric matrix of size $M \times M$, where $M = len(\vect{l})$.

\begin{eq}\label{eq:correlationmatrix}
    \textbf{C} = 
    \begin{cases} 
        r \left( k, l \right) & \text{if} \  k \neq l \ \text{and} \  nodes(k) \cap nodes(l) \neq \emptyset \\
        1 & \text{if} \ k = l \\
        0 & \text{otherwise}
    \end{cases} 
\end{eq}

%$\theta(k,l)$ is the angle between links $k$ and $l$, $nodes(l)$ is a function that returns the set of nodes of a link $l$, and 
The function $r \left( k, l \right)$ is an auto-correlation function, and it is parameterised based on the Reachi measurements in \cite{paper:linkmodel}.

\begin{eq}\label{eq:pathlossautocorrelation}
    r\left( k, l \right) = 0.595e^{-0.064 * \theta(k,l)} + 0.092
\end{eq}

With this, we can generate the correlation matrix for our sample graph \textbf{G} as:

\begin{eq}
    \textbf{C}_{\textbf{G}} = 
    \begin{bmatrix}
        1     & 0.125 & 0.094 & 0.094 & 0.125 & 0     \\
        0.125 & 1     & 0.125 & 0.125 & 0     & 0.125 \\
        0.094 & 0.125 & 1     & 0     & 0.125 & 0.094 \\
        0.094 & 0.125 & 0     & 1     & 0.125 & 0.094 \\
        0.125 & 0     & 0.125 & 0.125 & 1     & 0.125 \\
        0     & 0.125 & 0.094 & 0.094 & 0.125 & 1     \\
    \end{bmatrix}
\end{eq}

The correlation matrix $\textbf{C}$ is used to create the covariance matrix $\boldsymbol{\Sigma} = \sigma^2\textbf{C}$ by multiplying the correlation matrix with a standard deviation of $\sigma = 11.4$ \gls{db} squared. Next, we draw an \gls{iid} multivariate Gaussian vector, with the standard deviation $I = 1$.

\begin{eq}\label{eq:pathlossnormaldist}
    \vect{x} =  \big[ x_1 \  x_2 \  \ldots \  x_{len(\vect{l})} \big]^T \sim N(0, I) 
\end{eq}

The independent variables in the vector are made dependent by multiplication with the lower triangular Cholesky decomposition~\cite[p. 143]{Golub:1996:MC:248979}\cite[p. 100]{Press:2007:NRE:1403886} (\autoref{algo:cholesky}) of the covariance matrix $\textbf{Q} = cholesky\left(\boldsymbol{\Sigma}\right)$.

%{11.4, 0, 0, 0, 0, 0}
%{1.42955, 11.31, 0, 0, 0, 0}
%{1.07017, 1.30566, 11.2743, 0, 0, 0}
%{1.07017, 1.3057, -0.252793, 11.2715, 0, 0}
%{1.42955, -0.180692, 1.33076, 1.3609, 11.1472, 0}
%{0, 1.44093, 0.915234, 0.935987, 1.2618, 11.1614}

%{-0.121966, -1.08682, 0.68429, -1.07519, 0.0332695, 0.744836};


%{-1.39041, -12.4664, 6.17022, -13.8368, -0.158379, 6.41379};
\begin{eq}\label{eq:pathlossstoch}
    \vect{l_{fading}} = \textbf{Q}\vect{x}
\end{eq}

%With this, it is possible for us to compute a single realisation of the \gls{pathloss} of a link matrix, accounting for the distance dependent \gls{pathloss} and the spatial correlation. 
For our sample graph \textbf{G}, this could be:

\begin{eq}\label{eq:pathlossfadingG}
    \vect{l_{fading, \textbf{G}}} = 
        \textbf{Q}_{\textbf{G}} 
    \cdot
        \begin{bmatrix}
            0.122\\
            -1.087\\
            0.684\\
            -1.075\\
            0.033\\
            0.744
        \end{bmatrix}
    =
        \begin{bmatrix}
            1.390\\
            -12.466\\
            6.170\\
            -13.837\\
            -0.158\\
            6.413
        \end{bmatrix}
\end{eq}

Note that as $\vect{x}$ is stochastic, the values in \autoref{eq:pathlossfadingG} are a random realisation of the stochastic shadow fading part. \medbreak

Next, the shadow fading part is expanded to include the temporal correlation. The vector $l_{fading}\left(t\right)$ describes the shadow fading \gls{pathloss} at time $t$. As the distance dependent part is not time dependent, it needs no further modifications.

\begin{eq}\label{eq:pathlosstemporal}
    \vect{l_{fading}}(t + \Delta t) = \overbrace{\textbf{Q}(t + \Delta t)\vect{x}}^{spatial correlation} \overbrace{\sqrt{1 - \rho_{\Delta t}} + \vect{l_{fading}}(t)\rho_{\Delta t}}^{temporal correlation}
\end{eq}

The temporal correlation is computed based on the temporal correlation coefficient, $\rho_{\Delta t}$, describing the correlation after both transmitter and receiver have moved $|d_t|$ and $|d_r|$ meters, respectively, and with a decorrelation distance of 20 meters.

\begin{eq}
    \rho_{\Delta t} = e^{-\frac{|d_t|+|d_r|}{20}\ln (2)}
\end{eq}

Again, coming back to our sample graph \textbf{G}, and assuming that all devices would have moved 10 meters in the same direction (preserving angles and distances between links), the stochastic shadow fading part at $t = 1$ would determined by:

\begin{eq}\label{eq:pathlossfadingGtemporal}
    \vect{l_{fading, \textbf{G}}}(t=1) = 
        \textbf{Q}_{\textbf{G}} \cdot \vect{x} \cdot \sqrt{1 - \rho_{\Delta t}} + \vect{l_{fading,\textbf{G}}}(t=0) \cdot \rho_{\Delta t}
\end{eq} 

where $\rho_{\Delta t} = e^{-\frac{10+10}{20}\ln (2)} = 0.5$, $\vect{x}$ would be a new realisation of the Gaussian vector, and $\textbf{Q}_{\textbf{G}}$ would be recomputed based on the new locations of the nodes, should the angles between the nodes have changed. \medbreak

With this, the vectors \vect{l_d} and \vect{l_{fading}} can be combined as in \autoref{eq:pathlossdb}, to generate the \gls{pathloss} vector $\vect{l_{pl}}$ for all unique links in the network.

\begin{eq}\label{eq:pathlosslink}
    \vect{l_{pl,\textbf{G}}}(t = 0) = \vect{l_{d,\textbf{G}}} + \vect{l_{fading,\textbf{G}}}(t) =
    \begin{bmatrix}
        93.39\\
        87.734\\
        98.17\\
        78.163\\
        100.042\\
        98.413
    \end{bmatrix}
        %-64.610\\
        %-61.734\\
        %-72.170\\
        %-52.163\\
        %-74.042\\
        %-72.413
\end{eq}

All that remains is to subtract the \gls{pathloss} for a particular link $l$, from the transmission power of the transmitting node, to compute the \gls{rssi} in \acrshort{dbm}. $RSSI_{dBm}(n, m, t = 0)$ denotes the \gls{rssi} on the link $l_{n,m}$ between nodes $n$ and $m$ at time $t$, in \acrshort{dbm}.

\begin{eq}\label{eq:rssidbm}
%    RSSI(n, m, t = 0) = 26 \ \text{dBm} - l_{pl,l_{n,m}}(t)
    RSSI_{dBm}(n, m, t = 0) = tx_{power} - l_{pl,l_{n,m}}(t)
%    RSSI_{dBm}(l_1, t = 0) = 26 \ \text{dBm} - l_{pl,\textbf{G},l_1}(t) = {-60.21} \ \text{dBm}
\end{eq}

For example, if node $1$ in our sample graph \textbf{G} transmits with a transmission power of $26$ \acrshort{dbm} to node $2$, the \gls{rssi} on the receiving node would be $RSSI_{dBm}(n_1, n_2) = 26 \ \text{dBm} - l_{pl,\textbf{G},l_{n_1,n_2}}(t) = {-64.610}$ \acrshort{dbm}. In the following section, we show that a \acrshort{dbm} of $-64.610$ on a link will have a probability for packet loss of approximately $0.001\ \%$, given that no other nodes transfer at the same time.


%$P_{tx}(l)$ is a function returning the transmission power of the transmitting node in a given link.

%\begin{eq}\label{eq:rssidbmsample}
%    RSSI(1, 2) = 26 \ \text{dBm} - l_{pl,\textbf{G},l_{1,2}}(t) = {-64.610} \ \text{dBm}
%\end{eq}

%\begin{eq}\label{eq:pathlosslink}
%    \vect{l_{pl}}(t) = \vect{l_d} + \vect{l_{fading}}(t)
%\end{eq}

%\begin{eq}\label{eq:pathlossfadingG}
%    \vect{l_{fading, \textbf{G}}} = 
%        \textbf{Q}_{\textbf{G}} \cdot \vect{x}
%        =
%        \begin{bmatrix}
%            -5.79\\
%            -4.34\\
%            \vdots \\
%            10.04\\
%        \end{bmatrix}
%\end{eq} \medbreak


\section{Packet Error Probability}\label{sec:pep}
Now that we are able to simulate the \gls{rssi} when transmitting from one node to another, we need to be able to simulate whether the packet should arrive on the receiving node. With wireless radio communication, there is a chance that a receiving node may not receive the entire packet correctly, which is why we need some way of computing the probability of this happening. For this, we introduce the packet error probability. The computations in this section are derived from \cite{massoud2007digital}, as well as personal communication with the author of \cite{paper:linkmodel}. \medbreak

\subsection{Radio Model}

First we calculate the noise power $P_{N,db} = thermal\_noise + noise\_figure$ in \acrshort{db}. The noise power is calculated with the thermal noise and noise figure, and is the level of background noise affecting the wireless radio communication. For the Reachi devices we assume the $thermal\_noise = -119.66$ \acrshort{db} and the $noise\_figure = 4.2$ \acrshort{db}. \medbreak

Next we need to add the noise from interfering transmissions happening at the same time. We do this by adding the sum of the \gls{rssi} from interfering transmitters to the noise power $P_{N,dB}$, giving us $P_{NI,dB}$ on the link between the receiving node $n_r$ and the transmitting node $m_t$: 
\begin{eq}\label{eq:noisepower}
    P_{NI,db}(n_r, m_t) = 10 \log_{10}\left( 10^{\frac{P_{N,dB}}{10}} + \mathlarger{\sum}\limits_{m \in nodes_t}  10^{\frac{RSSI_{dBm}(n_{r}, m)}{10}} [m \neq m_{t}] \right) 
\end{eq}

The set of currently transmitting nodes are denoted by $nodes_t$ and the function $RSSI(n, m)$ denotes the RSSI on the link between nodes $n$ and $m$.\smallbreak

Next we calculate the signal to noise (and interference) ratio $\gamma_{dB}$ in \acrshort{db}. The ratio $\gamma_{dB}$ is the ratio between signal and noise, that compares the level of the signal to the level of the background noise (including the interference from other transmissions), and is computed by subtracting the noise power $P_{NI,db}$:

%We expand upon this further to include the interference of other transmissions happening at the same time. This is done by computing the \gls{rssi} from interfering transmitters, and subtracting this from the \gls{sinr}, which gives us the \acrlong{snir}:

% n receiver, m transmitter

%where we assume that the $thermal\_noise = -119.66$ \acrshort{db} and the $noise\_figure = 4.2$ \acrshort{db}.

%\begin{eq}
%    P_{N,db} = thermal\_noise + noise\_figure
%\end{eq}

%Signal to noise ratio with interference, on a link from $n_r$ (receiver) to $n_t$ (transmitter). The set of currently transmitting nodes are denoted by $nodes_t$. The function $RSSI(n, m)$ denotes the RSSI on the link between nodes $n$ and $m$. We subtract the sum of the RSSI between $n_r$ and any currently transmitting nodes, excluding $m_t$.

\begin{eq}
    \gamma_{dB}(n_r, m_t) = RSSI(n_r, m_t) - P_{NI,dB}(n_r, m_t)
\end{eq}

We use this to compute the bit error probability $P_b$:

%\begin{eq}
%    \gamma = 10^{\frac{\gamma_{dB}}{10}}
%\end{eq}

\begin{eq}
    P_b(n_r, m_t) = \frac{1}{2}erfc \left( \sqrt{ \left( \frac{10^{\frac{\gamma_{dB}(n_r, m_t)}{10}}}{2} \right)} \right)
\end{eq}

which we finally can use to compute the packet error probability $P_p$. The packet error probability is the probability that we experience a bit error for any of the bits in our packet, during transmission.

%\begin{eq}
%    P_b = \frac{1}{2}erfc \left( \sqrt{ \left( \frac{\gamma}{2} \right)} \right)
%\end{eq}

\begin{eq}
    P_p(n_r, m_t) = 1 - \left( 1 - P_b(n_r, m_t) \right) ^{packetsize}
\end{eq}

%\begin{figure}[ht]
%    \centering
%    \includegraphics[width=.7\textwidth]{figures/pep/pep.png}
%    \caption{Limits.}
%    \label{figure:pepegraph}
%\end{figure}

\subsection{Example}
In \autoref{sec:linkmodel}, we computed the link model for a sample network topology \textbf{G}. \autoref{eq:rssivector} shows a vector containing the \gls{rssi}, in \acrshort{dbm}, for each of the six links in the network, assuming a transmission power of 26 \acrshort{dbm}. For the following, it is assumed that $thermal\_noise = -119.66$ \acrshort{db}, the $noise\_figure = 4.2$ \acrshort{db}, and $packetsize = 160$.

\begin{eq}\label{eq:rssivector}
    \vect{RSSI_{dBm, \textbf{G}}} = 
        \begin{bmatrix}
            -64.610\\
            -61.734\\
            -72.170\\
            -52.163\\
            -74.042\\
            -72.413
        \end{bmatrix}
\end{eq}

If we assume that $n_2$ is currently listening, and the currently transmitting nodes $nodes_t = {n_1, n_3, n_4}$. What is the probability for packet error on the link between nodes $n_2$ and $n_4$ with interference from nodes $n_1$ and $n_3$? First, we compute the noise power $P_{NI,db}$, according to \autoref{eq:noisepower}:

%\begin{eq}
%    P_{NI,db}(n_2, n_4) = 10 \log_{10}\left( 10^{\frac{(-119.66 + 4.2)}{10}} + \mathlarger{\sum}\limits_{m \in {n_1, n_3, n_4}}  10^{\frac{RSSI_{dBm}(n_{2}, m)}{10}} [m \neq n_{4}] \right) 
%\end{eq}

\begin{eq}
    P_{NI,db}(n_2, n_4) = 10 \log_{10}\left( 10^{\frac{(-119.66 + 4.2)}{10}} + 10^{\frac{-64.610}{10}} + 10^{\frac{-52.163}{10}}  \right) = -51.923
    %\mathlarger{\sum}\limits_{m \in {n_1, n_3}}  10^{\frac{RSSI_{dBm}(n_{2}, m)}{10}} [m \neq n_{4}] \right) 
\end{eq}

We subtract the noise power $P_{NI,db}$ from the \gls{rssi} to get the signal to noise (and interference) ratio $\gamma_{dB}$:

\begin{eq}
    \gamma_{dB}(n_2, n_4) = -74.042 - (-51.923) = -22.119
\end{eq}

With which we can compute the bit error probability:

\begin{eq}
    P_b(n_2, n_4) = \frac{1}{2}erfc \left( \sqrt{ \left( \frac{10^{\frac{-22.119}{10}}}{2} \right)} \right) = 0.469
\end{eq}

Finally we can compute the packet error probability using the bit error probability:

\begin{eq}
    P_p(n_2, n_4) = 1 - \left( 1 - 0.469 \right) ^{160} = 1.0
\end{eq}

With a probability for bit errors of approximately 47 \%, we can say with 100 \% probability that we will experience packet loss on the link from $n_2$ to $n_4$, with two interfering transmitters.



%If we assume that $n_2$ is currently listening, and the currently transmitting nodes are $nodes_t = {n_1, n_3, n_4}$. Now, we want to compute the probability for packet error on the link between nodes $n_2$ and $n_4$.

%\subsection*{Example}
%
%\begin{figure}[H]
%    \centering
%    \begin{tikzpicture}
%        \begin{scope}[xshift=4cm]
%        \node[main node] (1) {$1$};
%        \node[main node] (2) [left = 2cm  of 1]  {$2$};
%        \node[main node] (3) [below = 2cm  of 2] {$3$};
%        \node[main node] (4) [right = 2cm  of 3] {$4$};
%
%        \path[draw,thick]
%        (1) edge node[above] {$l_1$} (2)
%        (1) edge node[above=.8cm, right] {$l_2$} (3)
%        (1) edge node[right] {$l_3$} (4)
%        (2) edge node[left] {$l_4$} (3)
%        (2) edge node[below=.8cm, right] {$l_5$} (4)
%        (3) edge node[below] {$l_6$} (4)
%        ;
%        \end{scope}
%    \end{tikzpicture}
%    \caption{Sample graph \textbf{G} with 4 nodes and 6 links.}
%    \label{figure:lm-sample}
%\end{figure}
%

%The vector \vect{RSSI_{\textbf{G}}} contains the RSSI for all links in \textbf{G}. It is assumed that $thermal\_noise = -119.66$ \acrshort{db}, the $noise\_figure = 4.2$ \acrshort{db}, and $packetsize = 160$. \medbreak

%If we assume that $n_2$ is currently listening, and the currently transmitting nodes are $nodes_t = {n_1, n_3, n_4}$. Now, we want to compute the probability for packet error on the link between nodes $n_2$ and $n_4$.
%(-119.66 + 4.2)
%\begin{eq}
%    \gamma_{dB}(n_2, n_4) = RSSI(n_2, n_4) - P_{N,dB} - \mathlarger{\sum}\limits_{m \in \{n_1, n_3, n_4\}} RSSI(n_2, m)[m \neq n_4]
%\end{eq}
%
%\begin{eq}
%    \gamma_{dB}(n_2, n_4) = -74.042 - (-119.66 + 4.2) - (-64.610 + (-52.163)) = 158.191
%\end{eq}
%
%\begin{eq}
%    P_b(n_2, n_4) = \frac{1}{2}erfc \left( \sqrt{ \left( \frac{10^{\frac{158.191}{10}}}{2} \right)} \right) = 1.406 %\cdot 10^{-36}
%\end{eq}
%
%\begin{eq}
%    P_p(n_2, n_4) = 1 - \left( 1 - P_b(n_2, n_4) \right) ^{160} = 0.0
%\end{eq}

%Which equals to a probability of approximately 10 \% packet loss, with no interference, and an \gls{rssi} of $-105.3$ \acrshort{dbm}.


\section{Optimising the Link Model}\label{sec:optimization}
In this section we explore two methods for optimising the computational time required for computing the link model introduced in \autoref{sec:linkmodel}. \autoref{sec:cholesky} presents the Cholesky decomposition, along with the issues we have experienced in using the decomposition. In \autoref{sec:clustering}, the clustering addition to the link model is presented, along with a formalised problem statement and discussion of the choice of algorithm, and its implementation.

\subsection{The Cholesky Decomposition}\label{sec:cholesky}
In \autoref{sec:simulatingvalues}, we utilised the Cholesky decomposition on the covariance matrix in the stochastic shadow fading part. The Cholesky decomposition is a decomposition algorithm for \gls{symmetric}, \gls{pd-matrix} into the product of a \gls{lt-matrix} and its \gls{conjugate-transpose}, and is primarily used for solving systems of linear equations~\cite{Press:2007:NRE:1403886}. In this Section, we present and describe the Cholesky decomposition, as well as the problems the decomposition creates for our computation time of the stochastic shadow fading part of the link model, as well as possible ways for us to optimise our usage of the Cholesky decomposition. \autoref{algo:cholesky} contains a pseudo code description of the Cholesky decomposition. \medbreak

\begin{algorithm}[H]
    \DontPrintSemicolon
    \KwResult{The Cholesky decomposition of the input matrix}
    \SetKwFunction{Cholesky}{Cholesky}
    \SetKwProg{Fn}{Function}{:}{}
    \Fn{\Cholesky{matrix, N}}{
        result $\leftarrow$ empty matrix of size N $\times$ N \;
        \For{n $\leftarrow$ 0, n < N}{
            \For{m $\leftarrow$ 0, m < n + 1}{
                sum $\leftarrow$ 0\;
                \For{i $\leftarrow$ 0, i < m}{
                    sum $\leftarrow$ sum + result$_{n,i} \cdot$ result$_{m,i}$\;
                }
                \If{n = m}{
                    \If{$\text{matrix}_{n,n} - \text{sum} \leq 0$}{
                        throw error; matrix is not positive-definite
                    }

                    results$_{n,m}$ $\leftarrow$ $\sqrt{\text{matrix}_{n,n} - \text{sum}}$\;
                }
                \Else{
                    results$_{n,m} \leftarrow \frac{1}{\text{result}_{m,m}} \cdot (\text{matrix}_{n,m} - sum)$\;
                }
            }
        }
        \KwRet result
    }
    \caption{Cholesky decomposition}
    \label{algo:cholesky}
\end{algorithm}
\medbreak
The first issue we have found with the Cholesky decomposition, or more specifically with the covariance matrix, is that the covariance matrix is not guaranteed to be a \gls{pd-matrix}. The covariance matrix is based on the relation between links in the network, which means that whether the matrix is positive-definite or not is entirely based on the network. To work around this, we have employed a tool called NearPD~\cite{website:nearPD} to transform our covariance matrix into a new matrix that has the positive-definite property, while minimising the Frobenius norm~\cite{website:frobieniusnorm} of the difference between the original and the new matrix. This significantly increases the time required to compute the link model, however, which leads us to our next major issue: The computational time required by the Cholesky decomposition itself. The computational time required for the NearPD tool and the Cholesky decomposition can be seen in \autoref{table:cholesky:spdtime}.\smallbreak

The Cholesky decomposition has a computational complexity of $O(n^3)$~\cite{Press:2007:NRE:1403886}. With a fully connected network of 1000 nodes, we would have a $\frac{1000(1000+1)}{2} - 1000 = 499500$ (\autoref{eq:lengthoflinks}) unique links, which means that our correlation (and covariance) matrix would be of size $499500 \times 499500$. This is a major issue, as we would like to be able to compute the link model in a relatively short amount of time. To combat this issue, we have two possible solutions: removing links with a distance over a certain threshold, and clustering nodes that are very close to each-other. We present the first solution in this section, and clustering in \autoref{sec:clustering}.

\begin{table}[H]
    \centering
    \begin{tabular}{|c|c|c|c|}
        \hline
        Nodes & Links & NearPD          & Cholesky                 \\\hline
        10    & 45    & 3 milliseconds  & \textless{}1 millisecond \\\hline
        20    & 190   & 88 milliseconds & 3 milliseconds           \\\hline
        30    & 435   & 1 seconds       & 34 milliseconds          \\\hline
        40    & 780   & 6 seconds       & 200 milliseconds         \\\hline
        50    & 1225  & 32 seconds      & 720 milliseconds         \\\hline
        60    & 1770  & 113 seconds     & 3 seconds                \\\hline
        70    & 2415  & 285 seconds     & 6 seconds                \\\hline
        80    & 3160  & 584 seconds     & 12 seconds               \\\hline
        90    & 4005  & 22 minutes      & 26 seconds               \\\hline
        100   & 4950  & 35 minutes      & 45 seconds               \\\hline
        110   & 5995  & 75 minutes      & 93 seconds               \\\hline
        120   & 7140  & 127 minutes     & 155 seconds              \\\hline
        130   & 8385  & 235 minutes     & 250 seconds              \\\hline
        140   & 9730  & Timeout         & \dots                    \\\hline%582 seconds              \\\hline
        %150   & 11175 & \dots           & 14 minutes               \\\hline
        %160   & 12720 & \dots           & 17 minutes               \\\hline
        %170   & 14365 & \dots           & 20 minutes               \\\hline
        %180   & 16110 & \dots           & 29 minutes               \\\hline
        %190   & 17955 & \dots           & 39 minutes               \\\hline
        %200   & 19900 & \dots           & 52 minutes               \\\hline
    \end{tabular}
    \caption{Computation time measurement for NearPD and Cholesky decomposition.}
    \label{table:cholesky:spdtime}
\end{table}

\autoref{table:cholesky:spdtime} shows time measurements from running the cholesky decomposition computation with different network topology sizes with the NearPD tool. The measurements shows that the NearPD tool is not fast enough, as at only 130 nodes with 8385 links the tool takes 235 minutes to compute the \gls{pd-matrix}. This is not fast enough and a better solution will have to be found. The cholesky decomposition computation speed is also problematic, as the goal is 1000 nodes and already at 120 nodes the decomposition takes more than 2 minutes.

\subsubsection{Distance Threshold}\label{sec:distancethreshold}
In \autoref{sec:linkmodel}, we saw that the distance dependent \gls{pathloss} has significantly more importance in the total \gls{pathloss} than the stochastic shadow fading part. In \autoref{eq:pathlossdetermG}, we see that the distance \gls{pathloss} is 92 \acrshort{dbm} for links of 100 meters, and 100.2 \acrshort{dbm} for the diagonal links of $141.42$ meters, and in \autoref{eq:pathlossfadingG}, we see (stochastic) \gls{pathloss} values between $-13.837$ and $6.413$. This means that, entirely based on the distance part of the link model, according to the formula for computing packet error probability in \autoref{sec:pep}, with a distance of 1000 meters, and a transmission power of 26 \acrshort{dbm}, we would have a link \gls{pathloss} of $147$ \acrshort{dbm} (disregarding the stochastic shadow fading part of the \gls{pathloss}), which in turn would mean that the \gls{rssi} on the receiving end of the link would be $26 - 147 = -121$, or equivalent to a packet loss probability of 99.999~\% (assuming the noise figure and thermal noise is the same as in \autoref{sec:pep}, and the packet size is 160 bits). Hence removing links that are far away can potentially reduce the size of the correlation matrix significantly. To show this, we ran an experiment, where we generated 1000 nodes with random locations in a $25 \times 25$ kilometre area, created links between these nodes based on their location, and iteratively scaled the maximum distance allowed between nodes (the distance threshold) and calculated the link path loss, based on the distance threshold. The results of this experiment can be seen in \autoref{table:cholesky:distance-threshold}. Recall that a fully connected network topology of 1000 nodes would have a total of 499500 links. With a distance threshold of 1000 meters, we get a total of 2346 links in our $25 \times 25$ kilometre area, which is a significant improvement as we would be able to compute the Cholesky decomposition in approximately six seconds (disregarding the positive-definiteness of the correlation matrix) according to \autoref{table:cholesky:spdtime}.

%The experiment consisted of generating 1000 random nodes in a 25 kilometre square area, create links from those nodes based on their location, iteratively scale the maximum allowed distance of the links (distance threshold) and calculate the link path loss. The results can be read in \autoref{table:cholesky:distance-threshold}.

\begin{table}[H]
    \centering
    \begin{tabular}{|c|c|c|}
        \hline
        Distance threshold & Links & Probability \\\hline % & Cholesky         \\\hline
        % 100 meters         & 182   & \\\hline % & 3 milliseconds   \\\hline
        % 125 meters         & 256   & \\\hline % & 7 milliseconds   \\\hline
        % 150 meters         & 360   & \\\hline % & 20 milliseconds  \\\hline
        % 175 meters         & 483   & \\\hline % & 46 milliseconds  \\\hline
        200 meters         & 92    & 0.001~\%    \\\hline % & 97 milliseconds  \\\hline
        % 225 meters         & 773   & \\\hline % & 185 milliseconds \\\hline
        250 meters         & 139   & 0.001~\%    \\\hline % & 333 milliseconds \\\hline
        % 275 meters         & 1157  & \\\hline % & 646 milliseconds \\\hline
        300 meters         & 208   & 0.001~\%    \\\hline % & 1 second         \\\hline
        % 325 meters         & 1589  & \\\hline % & 1 second         \\\hline
        350 meters         & 296   & 0.001~\%    \\\hline % & 2 seconds        \\\hline
        % 375 meters         & 2076  & \\\hline % & 4 seconds        \\\hline
        400 meters         & 394   & 0.001~\%    \\\hline % & 6 seconds        \\\hline
        % 425 meters         & 2642  & \\\hline % & 8 seconds        \\\hline
        450 meters         & 490   & 0.001~\%    \\\hline % & 11 seconds       \\\hline
        % 475 meters         & 3290  & \\\hline % & 15 seconds       \\\hline
        500 meters         & 592   & 6.000~\%        \\\hline % & 21 seconds       \\\hline
        % 525 meters         & 4035  & \\\hline % & 28 seconds       \\\hline
        550 meters         & 723   & 82.333~\%    \\\hline % & 37 seconds       \\\hline
        % 575 meters         & 4830  & \\\hline % & 49 seconds       \\\hline
        600 meters         & 846   & 99.999~\%   \\\hline % & 62 seconds       \\\hline
        % 625 meters         & 5723  & \\\hline % & 82 seconds       \\\hline
        650 meters         & 994   & 99.999~\%   \\\hline % & 142 seconds      \\\hline
        %675 meters         & 6656  & \\\hline % & 184 seconds      \\\hline
        700 meters         & 1164  & 99.999~\%   \\\hline % & 273 seconds      \\\hline
        % 725 meters         & 7608  & \\\hline % & 375 seconds      \\\hline
        750 meters         & 1349  & 99.999~\%   \\\hline % & 454 seconds      \\\hline
        % 775 meters               & 8696 \\\hline %  & 454 seconds      \\\hline
        800 meters         & 1507  & 99.999~\%   \\\hline %  & 580 seconds      \\\hline
        % 825 meters               & 9837 & \\\hline %  & 708 seconds      \\\hline
        850 meters         & 1714  & 99.999~\%   \\\hline % & 544 seconds      \\\hline
        % 875 meters               & 1101 & \\\hline %5 & 571 seconds      \\\hline
        900 meters         & 1910  & 99.999~\%   \\\hline % & 665 seconds      \\\hline
        % 925 meters               & 1225 & \\\hline %1 & 783 seconds      \\\hline
        950 meters         & 2102  & 99.999~\%   \\\hline % & 916 seconds      \\\hline
        % 975 meters               & 1356 & \\\hline %8 & 1058 seconds     \\\hline
        1000 meters        & 2346  & 99.999~\%   \\\hline % & 1214 seconds     \\\hline
    \end{tabular}
    \caption{Results from the distance threshold experiments.}
    \label{table:cholesky:distance-threshold}
\end{table}


% as the last row in \autoref{table:cholesky:distance-threshold} of 750m gives good results. More aggressive thresholds drastically reduces the total amount of links more, but has the potential of removing links that should have been there.

% 55 \log_{10} \left( d(l_1) \right) - 18


% \subsection{Cholesky decomposition}\label{sec:cholesky}
%In this section we present and describe the Cholesky decomposition, and the problem it creates for our computation time, and how we propose to optimise the decomposition algorithm for our particular needs.
%sec:simulatingvalues
% The cholesky decomposition or cholesky factorization is a matrix decomposition, of a positive-definite matrix, resulting in a lower triangular matrix and its conjugate transpose.

%In \autoref{sec:linkmodel} we utilise the Cholesky decomposition in \autoref{eq:pathlossstoch}. The Cholesky decomposition is a matrix decomposition, on a \gls{pd-matrix}. The decomposition results in a \gls{lt-matrix} and its \gls{conjugate-transpose}. The Cholesky decomposition is an expensive computation of cubic time complexity, as such we intend to speed up the algorithm. Furthermore since the decomposition requires an \gls{pd-matrix} to work, we choose to verify our auto-correlation matrix before decomposing it, to ensure that the decomposition will run correctly.

% is a decomposition of a Hermitian, positive-definite matrix into the product of a lower triangular matrix and its conjugate transpose,


% In \autoref{sec:linkmodel} we utilise the Cholesky decomposition in \autoref{eq:pathlossstoch}. 
% In \autoref{sec:linkmodel} we utilise the Cholesky decomposition. The Cholesky decomposition is an expensive computation of cubic time complexity and, as such, it needs to be more efficient for our use case. 

%Initially we propose to optimise the algorithm by changing the data structure from a matrix to an ordered map of key-value pairs. The keys will a tuple of links and the value will be the result of the auto-correlation function from \autoref{eq:pathlossautocorrelation}.\medbreak



%, where the pair will be sorted after the link with the largest id, will be the first element in the pair, eg. $l_1.id = 1$ and $l_2.id = 2$ then $key = (l_2, l_1)$. The map must be ordered since the cholesky decomposition uses previous calculated values, to calculate the next.

% shortly introduce cholesky

% our intended improvements



\subsection{Clustering}\label{sec:clustering}
%Another way of optimising the computation of the link model is to introduce clustering of nodes, used to reduce the amount of links required for the link model computation. 
The second option for optimising the computation of the link model, is to introduce clustering of nodes. As mentioned in \autoref{sec:pathloss}, links existing in the same physical environment should experience similar shadow fading \gls{pathloss}. This means that we are be able to cluster nodes with a minimum loss of precision, provided that we minimise difference of the correlation (the angle) between links, before and after clustering.

%minimise the difference of the angles between links before and after clustering (correlation)

%This means that we should be able to compute clusters of nodes, and use the clusters to faster compute the link model of multiple nodes at the same time, as we would, depending on the number of clusters, have a far smaller link matrix. \medbreak

%For our purposes, we have chosen to use a density-based clustering algorithm, more specifically the \gls{optics} algorithm (\autoref{algo:optics}). The idea behind density-based clustering is that for every node in a cluster, the neighbourhood, within a given radius, has to contain at least a minimum number of other nodes~\cite[p.~50]{Ankerst:1999:OOP:304182.304187}. A set of nodes in a network, $\textbf{N} \subseteq \mathcal{X}$, along with a distance parameter $\varepsilon$, denoting the maximum radius of a nodes neighbourhood, and $MinNodes \geq 2$ denoting the minimum number of nodes required to form a cluster, such that a single node will not be considered a cluster. The goal is to find a set of clusters, $C$, such that we are able to form a new network with the centroids of our set of clusters, as well any outlying node not contained in a cluster, $\text{NOISE} \subseteq \textbf{N}$, while minimising the radius of the clusters.

%\subsubsection{Metric Space}
% Minimise the diameter using parameters
% Maximum diameter as input, give me the smallest number of k-nodes, such that every other node is within the diameter of at least one node.
% Output: Some number of nodes, which is the smallest possible diameter

%metric space
\subsubsection{Metric Space}
A metric space is a pair $( \mathcal{X}, d )$ where $ \mathcal{X} $ is a set and $\textbf{d}:\mathcal{X} \times \mathcal{X} \rightarrow [0, \infty )$ is a metric, satisfying the following axioms:

\begin{itemize}
    \item Positivity: $d(x, y) \geq 0$
    \item Reflexivity: $d(x, y) = 0 \Longleftrightarrow x = y$
    \item Symmetry: $d(x, y) = d(y, x)$
    \item Triangle inequality: $d(x, z) \leq d(x, y) + d(y, z)$
\end{itemize}

Our chosen metric space is $\mathbb{R}^2$ using Euclidean distance.

%$\mathbb{R}^2$

\subsubsection{Problem Statement}

%A set $\textbf{N} \subseteq \mathcal{X}$, is provided together with parameters $\varepsilon$. The goal is to find a subset of clusters, such that the maximum angle between any node to nodes in the cluster is minimised. 
A set $\textbf{N} \subseteq \mathcal{X}$, is provided together with a parameter $k$, $k \leq |\textbf{N}|$. Initially, the set $\textbf{N}$ is a set of clusters, each containing a single node. The goal is to find a set of clusters, such that the computation time required for computing the link model can be reduced, while minimising the loss of precision in the stochastic shadow fading part. We minimise the loss of precision by minimising the difference of the correlation ($\Delta\theta$) between links to any node in a cluster, to the centroid of the cluster. The centroid of a cluster $x$ is defined as $cent(x) = \sum\limits_{n \in x} n \div |x|$. \smallbreak
%
The problem is defined as follows:\bigbreak

%This correlation difference is computed using the autocorrelation function from \autoref{eq:pathlossautocorrelation}.


% \autoref{figure:clusteringgoal} contains an example of this
%For example, we want to minimise the difference $\Delta\theta$ between the links $l_{n,u}$, $l_{n, c}$, in relation to the link $l_{n,m}$ in \autoref{figure:clusteringgoal}. The difference is computed using the autocorrelation function from \autoref{eq:pathlossautocorrelation}: $\Delta\theta = r(l_{n,m}, l_{n,u}) - r(l_{n,m},l_{n,c})$, where $u \in c$, $n \not\in c$, $m \not\in c$.

%The goal is to find a 
%subset of clusters, such that the difference of the correlation between ($\Delta\theta$) links to any node in a cluster, to the centroid of the cluster, is minimised, while reducing the number of clusters, thus reducing the computational time required to compute the link model, while minimising the loss of precision.

%s an example, in \autoref{figure:clusteringgoal}, we want to minimise the difference $\Delta\theta$, between the links $l_{n,u}$, $l_{n, C}$, in relation to the link $l_{n,m}$, or $\Delta\theta = r(l_{n,m}, l_{n,u}) - r(l_{n,m},l_{n,C})$, where $u \in C$, $n \not\in C$, $m \not\in C$, and $r$ is the autocorrelation function from \autoref{eq:pathlossautocorrelation}. 

%such that the difference between an angle any node from outside the cluster to the centroid of the cluster, and the original node in the cluster is minimised.
%\autoref{figure:clusteringgoal} demonstrate an example of this, where we see three clusters of nodes.
%$\delta\theta$

%such that the difference in correlation between links is minimised, while maximising the number of clusters. 

%Minimise the difference between the original correlation matrix, and the new correlation matrix created using the centroid of the clusters.

For a metric space $(\mathcal{X}, d)$,
\begin{itemize}
    \item Input: A set $\textbf{N} \subseteq \mathcal{X}$ and a parameter $k$, $k \leq |\textbf{N}|$.
    \item Output: A set of clusters $C \subseteq 2^{\textbf{N}}$, such that
          \begin{enumerate}
              \item $\bigcup\limits_{x \in C}x = \textbf{N}$,
              \item $\forall x, y \in C, \ \text{if} \ x \neq y \ \text{then} \ x \cap y = \emptyset$,
              \item $|C| \leq k$, and
              \item $C(n)$ is the cluster in $C$ that contains $n$, where $n \in C(n)$
          \end{enumerate}
    \item Goal: Minimise the $\Delta\theta(C)$ function:\smallbreak $\Delta\theta(C) = \mathlarger{\sum}\limits_{n_1, n_2, n_3 \ \in \ \textbf{N}} \left( r(l_{n_1,n_2}, l_{n_1,n_3}) - r(l_{cent(C(n_1),cent(C(n_2)}, l_{cent(C(n_1),cent(C(n_3)}) \right)^2$

          % \item Input: a set $\textbf{N} \subseteq \mathcal{X}$, and a parameter $\varepsilon$.
          %       %    \item Output: a set of clusters $C$ where, for each nodes in the cluster, the neighbourhood of a given radius $\varepsilon$ of the nodes has to contain at least a minimum number of nodes $MinNodes$.
          % \item Output: a set of clusters $C$. % færre elementer end det set vi startede med
          % \item Goal: minimise the difference in correlation, $\Delta\theta$, between links $l_{n,c}$, $l_{n,u}$, with relation to $l_{n, m}$, for any $c \in C$, $u \in c$, $n \not\in c$, and $m \not\in c$, where $n \neq m$.

          %$\forall c \in C, \forall u \in c, \forall m \not\in c, \forall n \not\in c, $

          %$\Delta\theta$, $\forall c \in C,  $  $\Delta\theta = r(l_{n,m}, l_{n,u}) - r(l_{n,m},l_{n,C})$
          %    \item Goal: minimise the cost $r^C_\infty(\textbf{N}) = \max\limits_{n \in \textbf{N}} d(n, C)$.
          %\item Goal: minimise the size of the network by creating a new network consisting of the set of nodes $centroids(C) \cup \text{NOISE}$
          %\item Goal: Minimise the cost $r^C_\infty(\textbf{P}) = \max{p \in \textbf{P}} d(p, C)$
\end{itemize}

%\todo[inline]{How to describe the goal? Is it the goal of the particular clustering algorithm, or our goal for using the clustering algorithm? In the second case, the goal is to reduce the number of overall nodes in the network, by creating clusters of nodes very close to each other, and using the centroid of the clusters for the link model.}

\autoref{figure:clusteringgoal} illustrates the goal, where we want to minimise the correlation difference $\Delta\theta$ between the nodes $n$, $m$, and $u$ (as well of the rest of the nodes), to the centroids of their respective clusters $C(n)$, $C(m)$, and $C(u)$.

\begin{figure}[ht]
    \centering
    \includegraphics[width=.5\textwidth]{figures/clustering/clustering.png}
    \caption{Minimising the difference of the correlation ($\Delta\theta$) between links to nodes in a cluster to the centroid of the cluster.}
    \label{figure:clusteringgoal}
\end{figure}


\subsubsection{Clustering algorithm choice}\label{sec:clustering:choice}
In the paper~\cite{proceeding:clustering-Survey} the authors conduct a survey on density based clustering algorithms, giving an overview on existing algorithms. As mentioned above, we want the clustering to minimise $\Delta\theta$ between nodes. This is done be clustering nodes that are relative close to each other, to ensure that the clusters radius will not become to large. The algorithm needs to handle variable shaped clusters, variable densities and handle noise (outliers) which is nodes that was not added to a cluster.

The \gls{dbscan} is a widely used algorithm, but does not give results with variable densities. The \gls{vdbscan} is a variation of the \gls{dbscan} that fixes the problem with variable densities, and even automatically finds the parameters normally needed when running the algorithm. This however does not make it possible, to more aggressively specify the density threshold. The \gls{vdbscan}'s found optimal parameters, might not create the desired small clusters that is needed according to our clustering problem statement.\smallbreak

% The paper mainly presents \gls{dbscan} and specialised variations, as it is a widely used algorithm for clustering based on density. \gls{dbscan} has the flaw that it does not handle variable dense densities, which is a requirement for our case. Most of the specialised variations has that same flaw, as speed is often the main focus of the specialisation.

\gls{optics} however can handle variable densities although slower than \gls{dbscan}. The complexity of \gls{optics} is O$(n^2)$, but if implemented with a spatial index has the complexity of O$(n\log n)$. As such we choose to use the \gls{optics} algorithm.



\subsubsection{Approximation using OPTICS}
We have chosen a greedy approach using the \gls{optics} algorithm~\cite{Ankerst:1999:OOP:304182.304187} to approximate solutions for our clustering problem. \gls{optics} is an algorithm for finding density-based clusters, and the run-time of the algorithm is O($n \ \cdot $ run-time of a neighbourhood query)~\cite[p.~53]{Ankerst:1999:OOP:304182.304187}. \smallbreak

The algorithm requires three parameters: our input set $\textbf{N}$, $\varepsilon$, describing the maximum distance between two nodes to consider for clustering, and $MinPts$, describing the minimum number of nodes required to form a new cluster. For our case, the parameter $MinPts = 2$. With this, the \gls{optics} algorithm will return the set of clusters $C \subseteq 2^{\textbf{N}}$, where any nodes not satisfying the $MinPts$ parameter will be a singleton cluster. \autoref{algo:clustering} presents the pseudo code description of our greedy approach, that repeatedly computes the a set of clusters $C$, until $|C| \leq k$, each time incrementing the $\varepsilon$ variable by 10 meters.

\begin{algorithm}[H]
    \DontPrintSemicolon
    \KwResult{A set of clusters $C \subseteq 2^{\textbf{N}}$, where $|C| \leq k$}
    \SetKwFunction{FCluster}{Cluster}
    \SetKwProg{Fn}{Function}{:}{}
    \Fn{\FCluster{\textbf{N}, $k$}}{
        $\varepsilon \leftarrow 0$ m\;
        $C \leftarrow \text{OPTICS}(\textbf{N}, \varepsilon, 2)$\;

        \Repeat{$|C| \leq k$}{
            $\varepsilon \leftarrow \varepsilon + 10$ m\;
            $C \leftarrow \text{OPTICS}(\textbf{N}, \varepsilon, 2)$\;
        }

        \KwRet $C$\;
    }

    \caption{Greedy approach using the \gls{optics} algorithm.}
    \label{algo:clustering}
\end{algorithm}

% The way we want to use the \gls{optics} algorithm is to create many clusters with a higher density (a lower value for $\varepsilon$), as bigger, lower density clusters, would not make sense for the purposes of the link model, given the loss of precision this would incur. The \gls{optics} algorithm in itself does not assign cluster memberships to nodes in the network~\cite[p.~52]{Ankerst:1999:OOP:304182.304187}. Instead, the algorithm produces a particular ordering of the nodes, from which we can extract actual cluster memberships from. This ordering is produced by computing two values for every node in the network: the \textit{core-distance} (\autoref{eq:coredist}) and the \textit{reachability-distance} (\autoref{eq:reachdist}) Additionally, a node is a \textit{core node} if at least $MinNodes$ are found within its $\varepsilon$-neighbourhood~\cite[p.~52]{Ankerst:1999:OOP:304182.304187}.
% \begin{equation}\label{eq:coredist}
%     coredist_{\varepsilon,MinNodes}(n) =
%     \begin{cases}
%         \text{UNDEFINED}                                          & \text{if} |N_\varepsilon(n)| < MinNodes \\
%         MinNodes\text{-th smallest distance to}\ N_\varepsilon(n) & \text{otherwise}
%     \end{cases}
% \end{equation}
% \begin{equation}\label{eq:reachdist}
%     reachdist_{\varepsilon,MinNodes}(o, n) =
%     \begin{cases}
%         \text{UNDEFINED}                                    & \text{if} |N_\varepsilon(n)| < MinNodes \\
%         max(coredist_{\varepsilon,MinNodes}(n), dist(o, n)) & \text{otherwise}
%     \end{cases}
% \end{equation}


% The \gls{optics} (\autoref{algo:optics}) algorithm processes each node only once, and performs one $\varepsilon$-neighbourhood query during the processing of a node, at \autoref{algo:optics:getneighbours1} and \autoref{algo:optics:getneighbours2}. This means that the run-time of the algorithm is heavily dependent on the $\varepsilon$-neighbourhood query, i.e., the run-time for the \gls{optics} algorithm is O($n \ \cdot $ run-time of an $\varepsilon$-neighbourhood query)~\cite[p.~53]{Ankerst:1999:OOP:304182.304187}. The algorithm starts by setting the reachability-distance of each node to \texttt{UNDEFINED} on \autoref{algo:optics:setreachdistundefined}. Note that the very first node processed will be added to the ordered list, at \autoref{algo:optics:addfirstnode}, with a reachability distance of \texttt{UNDEFINED}, and that we assume \texttt{UNDEFINED} to be greater than any defined distance~\cite[p.~54]{Ankerst:1999:OOP:304182.304187}. Next, we check if the node $n$ is a core node on \autoref{algo:optics:checkifcorepoint1}, and if yes, we initialise an empty priority queue \texttt{Seeds} on \autoref{algo:optics:initseeds}. With this, we call the \texttt{Update} function on \autoref{algo:optics:updateseeds}, which updates the priority queue with the reachability-distance of any unprocessed node in the $\varepsilon$-neighbourhood of $n$. Finally, we repeat the process for any node $m$ in the priority queue, and move on to the next unprocessed node in $N$.\medbreak

% \begin{algorithm}[ht]
%     \DontPrintSemicolon
%     \KwResult{Ordered list of nodes N$'$}
%     \SetKwFunction{FOptics}{Optics}
%     \SetKwFunction{FUpdate}{Update}
%     \SetKwProg{Fn}{Function}{:}{}
%     \Fn{\FOptics{N, $\varepsilon$, MinNodes}}{
%         N$'$ = ordered list\;

%         \ForEach{$n \in$ N}{
%             n.reachdist = UNDEFINED\;\label{algo:optics:setreachdistundefined}
%             n.processed = false\;
%         }

%         \ForEach{unprocessed $n \in$ N}{
%             N$_\varepsilon$ = getNeighbours(n, $\varepsilon$)\;\label{algo:optics:getneighbours1}
%             n.processed = true\;
%             N$'$.insert(n)\;\label{algo:optics:addfirstnode}

%             \If{coredist$_{\varepsilon,MinNodes}$(n) $\neq$ UNDEFINED}{\label{algo:optics:checkifcorepoint1}
%                 Seeds = empty priority queue\;\label{algo:optics:initseeds}
%                 Update(N$_\varepsilon$, n, Seeds, $\varepsilon$, MinNodes)\;\label{algo:optics:updateseeds}

%                 \ForEach{next $m \in \text{Seeds}$}{
%                     M$_\varepsilon$ = getNeighbours(m, $\varepsilon$)\;\label{algo:optics:getneighbours2}
%                     m.processed = true\;
%                     N$'$.insert(m)\;

%                     \If{coredist$_{\varepsilon,MinNodes}$(m) $\neq$ UNDEFINED}{\label{algo:optics:checkifcorepoint2}
%                         Update(M$_\varepsilon$, m, Seeds, $\varepsilon$, MinNodes)\;
%                     }
%                 }
%             }
%         }

%         \KwRet N$'$\;
%     }\;

%     \Fn{\FUpdate{N, n, Seeds, $\varepsilon$, MinNodes}}{
%         coredist = coredist$_{\varepsilon,MinNodes}$(n)\;

%         \ForEach{$o \in$ N}{
%             \If{o.processed = false}{
%                 reachdist = reachdist$_{\varepsilon,MinNodes}$(o, n)\;

%                 \If{o.reachdist = UNDEFINED}{
%                     \tcp{o is not in Seeds}
%                     o.reachdist = reachdist\;
%                     Seeds.insert(o, reachdist)\;
%                 }
%                 \Else{
%                     \tcp{o is in Seeds, check for improvement}
%                     \If{reachdist $<$ o.reachdist}{
%                         o.reachdist = reachdist\;
%                         Seeds.move-up(o, reachdist)\;
%                     }
%                 }


%             }
%         }
%     }

%     \caption{The OPTICS algorithm~\cite{Ankerst:1999:OOP:304182.304187}.}
%     \label{algo:optics}
% \end{algorithm}

% \todo[inline]{Write pseudocode for the cluster extraction algorithm.}
% %\todo[inline]{Add reachability plot for a sample run}
% \todo[inline]{Add table demonstrating parameterised runs, amount of clusters vs resulting nodes in the network, as well as the radius of the clusters}
% \todo[inline]{Maybe use a smaller area and with fewer/smaller actual clusters for the demonstration.  }

% \begin{table}[H]
%     \centering
%     \begin{tabular}{|l|l|l|l|}
%         \hline
%         $\varepsilon$ & Clusters & Correlation difference & Duration \\\hline
%         0.05 km       & 99       & 2.330                  & 2 ms     \\\hline
%         0.06 km       & 99       & 2.330                  & 2 ms     \\\hline
%         0.07 km       & 99       & 2.330                  & 2 ms     \\\hline
%         0.08 km       & 99       & 2.330                  & 2 ms     \\\hline
%         0.09 km       & 97       & 4.909                  & 2 ms     \\\hline
%     \end{tabular}
%     \caption{Computation time measurement for NearPD and Cholesky decomposition.}
%     \label{table:clusteringtime}
% \end{table}



\subsubsection{Experiments}\label{sec:clustering:experiments}
To measure the performance of our clustering approach, we ran an experiment. The experiment consisted of running the clustering algorithm on 1000 randomly generated nodes, create a fully connected network of the clusters and measure the computational time used on clustering. \autoref{table:clustering-benchmark} contain the results of this experiment.

\begin{table}[H]
    \begin{tabular}{|c|c|c|c|}\hline
        Min clusters ($k$) & Resulting clusters & Links  & Clustering       \\\hline%& Cholesky \\\hline
        1000               & 1000               & 499500 & 244 milliseconds \\\hline%& \dots    \\\hline
        900                & 884                & 390286 & 1 second         \\\hline%& \dots    \\\hline
        800                & 771                & 296835 & 2 seconds        \\\hline%& \dots    \\\hline
        700                & 692                & 239086 & 2 seconds        \\\hline%& \dots    \\\hline
        600                & 592                & 174936 & 3 seconds        \\\hline%& \dots    \\\hline
        500                & 463                & 106953 & 4 seconds        \\\hline%& \dots    \\\hline
        450                & 424                & 89676  & 4 seconds        \\\hline%& \dots    \\\hline
        400                & 389                & 75466  & 4 seconds        \\\hline%& \dots    \\\hline
        350                & 322                & 51681  & 5 seconds        \\\hline%& \dots    \\\hline
        300                & 299                & 44551  & 5 seconds        \\\hline%& \dots    \\\hline
        %  250          & 229                & 26106  & 249 milliseconds & \dots           \\\hline
        %  200          & 183                & 16653  & 250 milliseconds & 32 minutes      \\\hline
        %  150          & 136                & 9180   & 250 milliseconds & 325 seconds     \\\hline
        %  100          & 98                 & 4753   & 250 milliseconds & 45 seconds      \\\hline
        %  50           & 37                 & 666    & 253 milliseconds & 67 milliseconds \\\hline
    \end{tabular}
    \caption{Experiment results from benchmarking clustering time.}
    \label{table:clustering-benchmark}
\end{table}

\todo[inline]{add table showing the results of the $\Delta\theta(C)$ function (the correlation difference)}

%To verify that the clustering actually improves the cholesky computations, experiments have been run. The results can be soon in \autoref{table:clustering-benchmark}.\medbreak

%The experiment consisted of iteratively running our clustering on 1000 randomly generated nodes until the amount of clusters desired is achieved, create a fully connected network of the clusters and measure the used on clustering and computing the path loss with the cholesky. The columns in \autoref{table:clustering-benchmark} are as followed:

%\begin{description}
%    \item[Min clusters:] The minimum allowed amount of clusters. In the \gls{optics} algorithm, one does not specify an amount of desired clusters, hence a minimum needs to be specified to stop clustering once this minimum has been reached.
%    \item[Resulting clusters:] The resulting amount of clusters.
%    \item[Links:] The total amount of links.
%    \item[Clustering time:] How long the clustering took.
%    \item[Cholesky:] How long the cholesky computation took. Some fields in the cholesky column are marked as \dots. This is because the computation took too long to let it finish as the correlation matrix simply became too big.
%\end{description}


% write about the results 

%\autoref{figure:clustering:nodes} demonstrates a sample run of the algorithm on a network of 2000 nodes, with 1000 nodes uniformly distributed across the entire 15.5 km by 7.8 km area, two clusters containing 200 nodes each, and two clusters containing 300 nodes each. The low $\varepsilon = 0.08$ km value means that only very densely packed nodes will be added to the clusters. Ideally, a lot of the outliers in the figure should be able to be clustered with $MinNodes=2$, but increasing the $\varepsilon$ could also mean that the radius of the clusters could become too large.

%based on features called \textit{core-distance} and\textit{reachability-distance}~\cite{Ankerst:1999:OOP:304182.304187}, 
%1179
%\begin{figure}[ht]
%    \centering
%    \begin{subfigure}[t]{.31\textwidth}
%        \includegraphics[width=\linewidth]{figures/clustering/2k.png}
%        %\caption{2000 nodes.}
%        \label{figure:clustering:2knodes}
%    \end{subfigure}
%    \begin{subfigure}[t]{.3223\textwidth}
%        \includegraphics[width=\linewidth]{figures/clustering/2k80m2pts.png}
%        %\caption{2000 nodes clustered.}
%        \label{figure:clustering:2k80m2ptsnodes}
%    \end{subfigure}
%    \caption{Result of the OPTICS algorithm with parameters $\varepsilon = 0.08$ km, and %$MinNodes = 2$.}
%    \label{figure:clustering:nodes}
%\end{figure}

%7.81 km * 15.57 km

%\begin{figure}[ht]
%    \centering
%    \includegraphics[width=.7\textwidth]{figures/clustering/2k-80m-2pts.png}
%    \caption{2000 nodes}
%    \label{figure:clustering:2k}
%\end{figure}
%
\section{Optimization}\label{sec:optimization}
% \subsection{Cholesky decomposition}\label{sec:cholesky}
%In this section we present and describe the Cholesky decomposition, and the problem it creates for our computation time, and how we propose to optimise the decomposition algorithm for our particular needs.

% The cholesky decomposition or cholesky factorization is a matrix decomposition, of a positive-definite matrix, resulting in a lower triangular matrix and its conjugate transpose.

In \autoref{sec:linkmodel} we utilise the Cholesky decomposition in \autoref{eq:pathlossstoch}. The Cholesky decomposition or Cholesky factorization is a matrix decomposition, on a \gls{pd-matrix}. The decomposition results in a \gls{lt-matrix} and its \gls{conjugate-transpose}. The Cholesky decomposition is an expensive computation of cubic time complexity, as such we intend to speed up the algorithm. Furthermore since the decomposition requires an \gls{pd-matrix} to work, we choose to verify our auto-correlation matrix before decomposing it, to ensure that the decomposition will run correctly.

In the following sections we discuss our optimizations.

% is a decomposition of a Hermitian, positive-definite matrix into the product of a lower triangular matrix and its conjugate transpose,


% In \autoref{sec:linkmodel} we utilise the Cholesky decomposition in \autoref{eq:pathlossstoch}. 
% In \autoref{sec:linkmodel} we utilise the Cholesky decomposition. The Cholesky decomposition is an expensive computation of cubic time complexity and, as such, it needs to be more efficient for our use case. 

%Initially we propose to optimise the algorithm by changing the data structure from a matrix to an ordered map of key-value pairs. The keys will a tuple of links and the value will be the result of the auto-correlation function from \autoref{eq:pathlossautocorrelation}.\medbreak



%, where the pair will be sorted after the link with the largest id, will be the first element in the pair, eg. $l_1.id = 1$ and $l_2.id = 2$ then $key = (l_2, l_1)$. The map must be ordered since the cholesky decomposition uses previous calculated values, to calculate the next.

% shortly introduce cholesky

% our intended improvements


\begin{algorithm}[ht]
    \DontPrintSemicolon
    \KwResult{}
    \SetKwFunction{Cholesky}{Cholesky}
    \SetKwProg{Fn}{Function}{:}{}
    \Fn{\Cholesky{matrix}}{
        result$_{n,m}$\;
       \For{n = 0, n < matrix.size}{
            \For{m = 0, m < n + 1}{
                sum = 0\;
                \For{i = 0, i < m}{
                    sum = sum + result$_{n,i} \cdot$ result$_{m,i}$\;
                }
                \If{n == m}{
                    results$_{n,m}$ = $\sqrt{\text{matrix}_{n,n} - \text{sum}}$\;
                }
                \Else{
                    results$_{n,m} = \frac{1}{\text{result}_{m,m}} \cdot (\text{matrix}_{n,m} - sum)$\;
                }
            }
        }
        \KwRet result
    }
    \caption{Cholesky decomposition}
    \label{algo:cholesky}
\end{algorithm}
















% øøøøøh tænker at alt nedefra kan nærmest dumbes

%In \autoref{eq:correlationmatrix} we create the correlation matrix. The correlation matrix will be used later in the computations, and such optimising the matrix is %relevant. We will do the following steps to optimise:
%
%\begin{enumerate}
%    \item Remove duplicates
%    \item Remove elements of value 0 or value below a given threshold
%\end{enumerate}
%
%
%
%
%First we remove duplicates. The correlation matrix is a symmetric matrix, which means that $m_{i,j} = m_{j,i}$. As such we can safely remove either $m_{i,j}$ or %$m_{j,i}$ from our data structure, as it would be redundant to keep duplicate values representing the same link pair. Secondly we remove link pairs where their %calculated correlation is below a threshold. When elements has the value 0, it means that the link pair has no correlation, and can therefore be omitted. Link pairs %where their calculated correlation is below a defined threshold, will also be omitted. This is done because link pairs with a very low correlation will have a very high %packet error probability, where the probability for successfully receiving a packet will be too low.
%
%\subsubsection{Benchmarks}
%
%\begin{table}[ht]
%\begin{tabular}{|l|l|l|l|l|l|l|}
%\hline
% & 50 & 100 & 200 & 500 & 1000  \\ \hline
%Cholesky &  &  &  &  &  \\ \hline
%Cholesky\_improved &  &  &  &  &  \\ \hline
%\end{tabular}
%\end{table}