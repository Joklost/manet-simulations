\section{Radio Communication}\label{sec:radio-communication}
In this section, several necessary terms needed to fully understand the main content is presented and explained.

\begin{description}[style=nextline]
    \item[Transmission power] The transmission power, is the power level that the transmitter is configured to.
    \item[\gls{rssi}] \gls{rssi} is an indication of the power level of a received signal, in the receiver radio. The measurement is measured after the possible loss uncured by the radio and/or cable. The \gls{rssi} minimum and maximum values are not standardised. The manufactures of the specific chips decided that, eg. Cisco systems \gls{rssi} range from 0-100, while Atheros range from 0-60 \cite{website:rssi-metageek}.
    \item[Baud rate] The baud rate is a measure of symbols per second. In the citation, they describe it in the context of serial ports, but in this case the symbols are bits \cite{website:baudrate-mathworks}.
\end{description}