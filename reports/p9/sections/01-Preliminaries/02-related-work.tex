\section{Related Work}\label{sec:relatedworks}

% Work that propose a different method to solve the same problem
% Work that uses the same proposed method to solve a different problem
% A method that is similar to your method that solves a relatively similar problem
% A discussion of a set of related problems that covers your problem domain.

%\todo[inline]{See comments}

%\href{Modeling and Evaluation of Wireless Sensor Network Protocols by Stochastic Timed Automata}{https://www.sciencedirect.com/science/article/pii/S1571066113000479},
%\href{Modeling and Efficient Verification of Broadcasting Actors}{https://link.springer.com/chapter/10.1007/978-3-319-24644-4_5}

In \doublequote{Modeling and Efficient Verification of Broadcasting Actors}~\cite{DBLP:conf/fsen/YousefiGK15} the authors present an extension to the actor-based modelling language Rebeca~\cite{Sirjani2004ModelingAV}, that enable broadcast communication between actors (nodes), in order to allow modelling of \gls{manet}s. The authors provide a framework to model \gls{manet}s for a static topology, with no support for mobility. The same authors further extend Rebeca to add key features of wireless ad hoc networking, such as mobility (dynamic topologies), local broadcasting within a transmission range, and energy consumption in \doublequote{Modeling and efficient verification of wireless ad hoc networks}~\cite{DBLP:journals/fac/YousefiGK17}. The modelling language lacks features such as lossy transmissions (packet loss) and non-deterministic behaviour, and generally abstracts away from wireless communication, in order to focus on modelling and verification of \gls{manet} protocols.\smallbreak

The paper \doublequote{Modeling and Evaluation of Wireless Sensor Network Protocols by Stochastic Timed Automata}~\cite{article:maeofwsnpbsta} propose a method to analyse and evaluate \gls{wsn} protocols using Stochastic Timed Automata, along with the non-deterministic behaviour of \gls{wsn}s, such as lossy transmission and dynamic topologies. The authors utilise statistical model checking to evaluate the performance of \gls{wsn} protocols, as well as checking the correctness of the protocols.\smallbreak