% Set the scene

A \gls{manet} is a decentralised wireless network that requires no pre-existing infrastructure, such as routers or access points~\cite{inproceedings:routingsurvery}. Instead, each node in the ad-hoc network are battery powered, and communicates directly with each-other using a radio. Due to this, \gls{manet}s rely on wireless networking protocols to provide energy efficient communication in the network. Because of the ad-hoc nature of \gls{manet}s, common applications consist of enabling communication in emergency situations such as natural disasters, or for military conflicts.\smallbreak

% Outline the goal
The goal of this project is to simulate wireless network protocols for communication in a mobile setting. Ideally the capabilities of a wireless network protocol are tested in a real-life scenario, using physical devices for radio communication. This poses interesting challenges such as scalability, and repeatability. Scaling a real-life test requires a significant amount of effort and investment of both money and time, and repeating the same test over and over becomes near impossible, the larger the scale. Our goal is to be able to perform repeatable experiments in a control topology environment with up to 1000 devices, faster than a real-life scenario.\smallbreak

Our project proposes a \gls{mpi} C++ library for writing, and running, simulations of the network protocol behind the mesh communication in a \gls{manet}, using state-of-the-art modelling of link \gls{pathloss}~\cite{paper:linkmodel} behind the wireless communication, to simulate packet loss and collisions caused by interfering transmitters, where the physical devices, and the communication between these, are emulated entirely using software. With our library, it is be possible to write a C++ implementation of communication protocols, such as \gls{lmac}~\cite{paper:lmac_protocol} or Slotted ALOHA~\cite{Roberts:1975:APS:1024916.1024920}, using a simple interface header file resembling a traditional hardware interface, and perform simulations with these, where each physical device is emulated in real-time by different CPUs on the MCC compute cluster at AAU~\cite{website:mccaau}. \smallbreak

% Outline contributions
The primary contribution of this project is the centralised controller that facilitate wireless communication between the emulated physical devices. The centralised controller allow us to simulate wireless communication in virtual time, where the controller is able to skip periods of inactivity, reducing the time required to run real-time simulations. In addition to the centralised controller, we suggest two methods for optimising the computational time required for computing the link \gls{pathloss}, as this poses a significant scalability challenge.

\newpage