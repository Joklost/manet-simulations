\documentclass[11pt,twoside,a4paper,openright]{report}
%%%%%%%%%%%%%%%%%%%%%%%%%%%%%%%%%%%%%%%%%%%%%%%%
% Language, Encoding and Fonts
% http://en.wikibooks.org/wiki/LaTeX/Internationalization
%%%%%%%%%%%%%%%%%%%%%%%%%%%%%%%%%%%%%%%%%%%%%%%%
% Select encoding of your inputs. Depends on
% your operating system and its default input
% encoding. Typically, you should use
%   Linux  : utf8 (most modern Linux distributions)
%            latin1
%   Windows: ansinew
%            latin1 (works in most cases)
%   Mac    : applemac
% Notice that you can manually change the input
% encoding of your files by selecting "save as"
% an select the desired input encoding.
\usepackage{scrextend}
\usepackage[utf8]{inputenc}
%\usepackage{float}
\usepackage{floatrow}
\usepackage{booktabs}

\usepackage[table,xcdraw]{xcolor}
% Make latex understand and use the typographic
% rules of the language used in the document.
\usepackage[english]{babel}
% Use the vector font Latin Modern which is going
% to be the default font in latex in the future.
\usepackage{lmodern}
% Choose the font encoding
\usepackage[T1]{fontenc}
\usepackage{lastpage} % Tilføjet af Jesper, får sidetal til at virke.
%%%%%%%%%%%%%%%%%%%%%%%%%%%%%%%%%%%%%%%%%%%%%%%%
% Graphics and Tables
% http://en.wikibooks.org/wiki/LaTeX/Importing_Graphics
% http://en.wikibooks.org/wiki/LaTeX/Tables
% http://en.wikibooks.org/wiki/LaTeX/Colors
%%%%%%%%%%%%%%%%%%%%%%%%%%%%%%%%%%%%%%%%%%%%%%%%
% load a colour package
\usepackage[table]{xcolor}

\definecolor{aaublue}{RGB}{33,26,82}% dark blue
\definecolor{editgreen}{HTML}{008000}% dark green
% The standard graphics inclusion package
\usepackage{graphicx}
\usepackage{environ}
% Set up how figure and table captions are displayed
\usepackage[within=none]{caption}
\captionsetup{%
  font=footnotesize,% set font size to footnotesize
  labelfont=bf % bold label (e.g., Figure 3.2) font
}
% Make the standard latex tables look so much better
\usepackage{array,booktabs}
% Enable the use of frames around, e.g., theorems
% The framed package is used in the example environment
\usepackage{framed}

\usepackage{enumitem}
%%%%%%%%%%%%%%%%%%%%%%%%%%%%%%%%%%%%%%%%%%%%%%%%
% Mathematics
% http://en.wikibooks.org/wiki/LaTeX/Mathematics
%%%%%%%%%%%%%%%%%%%%%%%%%%%%%%%%%%%%%%%%%%%%%%%%
% Defines new environments such as equation,
% align and split
\usepackage{amsmath}
% Adds new math symbols
\usepackage{amssymb}
% Use theorems in your document
% The ntheorem package is also used for the example environment
% When using thmmarks, amsmath must be an option as well. Otherwise \eqref doesn't work anymore.
\usepackage[framed,amsmath,thmmarks]{ntheorem}

%%%%%%%%%%%%%%%%%%%%%%%%%%%%%%%%%%%%%%%%%%%%%%%%
% Page Layout
% http://en.wikibooks.org/wiki/LaTeX/Page_Layout
%%%%%%%%%%%%%%%%%%%%%%%%%%%%%%%%%%%%%%%%%%%%%%%%
% Change margins, papersize, etc of the document
\usepackage[top=1in, bottom=1.5in, left=1in, right=1in]{geometry}
\setlength\parindent{0pt}
%\usepackage[
%  inner=28mm,% left margin on an odd page
%  outer=41mm,% right margin on an odd page
%  ]{geometry}
% Modify how \chapter, \section, etc. look
% The titlesec package is very configureable
\usepackage{titlesec}
\titleformat*{\section}{\normalfont\Large\bfseries\color{aaublue}}
\titleformat*{\subsection}{\normalfont\large\bfseries\color{aaublue}}
\titleformat*{\subsubsection}{\normalfont\normalsize\bfseries\color{aaublue}}
%\titleformat*{\paragraph}{\normalfont\normalsize\bfseries\color{aaublue}}
%\titleformat*{\subparagraph}{\normalfont\normalsize\bfseries\color{aaublue}}

% Clear empty pages between chapters
\let\origdoublepage\cleardoublepage
\newcommand{\clearemptydoublepage}{%
  \clearpage
  {\pagestyle{empty}\origdoublepage}%
}
\let\cleardoublepage\clearemptydoublepage

% Change the headers and footers
\usepackage{fancyhdr}
\pagestyle{fancy}
\fancyhf{} %delete everything
\renewcommand{\headrulewidth}{0pt} %remove the horizontal line in the header
\fancyhead[RE]{\color{aaublue}\small\nouppercase\leftmark} %even page - chapter title
\fancyhead[LO]{\color{aaublue}\small\nouppercase\rightmark} %uneven page - section title
\fancyhead[LE,RO]{\thepage} %page number on all pages
\setlength{\headheight}{14pt}
% Do not stretch the content of a page. Instead,
% insert white space at the bottom of the page
\raggedbottom
% Enable arithmetics with length. Useful when
% typesetting the layout.
\usepackage{calc}

%%%%%%%%%%%%%%%%%%%%%%%%%%%%%%%%%%%%%%%%%%%%%%%%
% Bibliography
% http://en.wikibooks.org/wiki/LaTeX/Bibliography_Management
%%%%%%%%%%%%%%%%%%%%%%%%%%%%%%%%%%%%%%%%%%%%%%%%
% Add the \citep{key} command which display a
% reference as [author, year]
\usepackage[numbers]{natbib}
% Appearance of the bibliography
\bibliographystyle{plain} %plain, unsrtnat

%%%%%%%%%%%%%%%%%%%%%%%%%%%%%%%%%%%%%%%%%%%%%%%%
% Misc
%%%%%%%%%%%%%%%%%%%%%%%%%%%%%%%%%%%%%%%%%%%%%%%%
% Add bibliography and index to the table of
% contents
\usepackage[nottoc]{tocbibind}
% Add the command \pageref{LastPage} which refers to the
% page number of the last page
\usepackage[
%  disable, %turn off todonotes
  colorinlistoftodos, %enable a coloured square in the list of todos
  textwidth=\marginparwidth, %set the width of the todonotes
  textsize=scriptsize, %size of the text in the todonotes
  ]{todonotes}

\usepackage[T1]{fontenc}

\usepackage{color}

\usepackage[cache=false]{minted}
\renewcommand{\listingscaption}{Code}

\usepackage[bitheight=6ex]{bytefield}

%%%%%%%%%%%%%%%%%%%%%%%%%%%%%%%%%%%%%%%%%%%%%%%%
% Hyperlinks
% http://en.wikibooks.org/wiki/LaTeX/Hyperlinks
%%%%%%%%%%%%%%%%%%%%%%%%%%%%%%%%%%%%%%%%%%%%%%%%
% Enable hyperlinks and insert info into the pdf
% file. Hypperref should be loaded as one of the
% last packages
\usepackage{hyperref}
\hypersetup{%
	plainpages=false,%
	pdfauthor={Charlie Dittfeld Byrdam \& Jonas Kloster Jacobsen},%
	pdftitle={MassiveSimulations},%
	pdfsubject={Project Report},%
	bookmarksnumbered=true,%
	colorlinks,%
	citecolor=aaublue,%
	filecolor=aaublue,%
	linkcolor=aaublue,% you should probably change this to black before printing
	urlcolor=aaublue,%
	pdfstartview=FitH%
}

\usepackage{courier}
\usepackage[normalem]{ulem}
% Quotations package
\usepackage[autostyle]{csquotes}
% PDF package
\usepackage[final]{pdfpages}
\usepackage[acronym,toc]{glossaries}
\usepackage{longtable}
\usepackage{subcaption}
\addtokomafont{labelinglabel}{\sffamily\bfseries}
\usepackage{dirtytalk}

\usepackage{silence}
\WarningFilter{glossaries}{Deprecated command}% Removes warning starting with "Deprecated command"
\WarningFilter{glossaries}{Empty glossary}

\usepackage[ruled, vlined, commentsnumbered, linesnumbered]{algorithm2e}
\SetArgSty{textnormal}
\SetFuncArgSty{textnormal}
