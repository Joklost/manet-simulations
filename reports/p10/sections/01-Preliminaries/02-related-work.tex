\section{Related Work}\label{sec:relatedworks}
%\bibtodo{no changes yet}
% Work that propose a different method to solve the same problem
% Work that uses the same proposed method to solve a different problem
% A method that is similar to your method that solves a relatively similar problem
% A discussion of a set of related problems that covers your problem domain.

%\todo[inline]{See comments}

%\href{Modeling and Evaluation of Wireless Sensor Network Protocols by Stochastic Timed Automata}{https://www.sciencedirect.com/science/article/pii/S1571066113000479},
%\href{Modeling and Efficient Verification of Broadcasting Actors}{https://link.springer.com/chapter/10.1007/978-3-319-24644-4_5}

In \doublequote{Modeling and Efficient Verification of Broadcasting Actors}~\cite{DBLP:conf/fsen/YousefiGK15}
the authors present an extension to the actor-based modelling language Rebeca~\cite{Sirjani2004ModelingAV},
that enable broadcast communication between actors (nodes), to allow modelling of \gls{manet}s. The
authors provide a framework to model \gls{manet}s for a static topology, with no support for mobility. The
same authors further extend Rebeca to add key features of wireless ad hoc networking, such as mobility
(dynamic topologies), local broadcasting within a transmission range, and energy consumption in
\doublequote{Modeling and efficient verification of wireless ad hoc
networks}~\cite{DBLP:journals/fac/YousefiGK17}. The modelling language lacks features such as lossy
transmissions (packet loss) and non-deterministic behaviour, and generally abstracts away from wireless
communication, to focus on modelling and verification of \gls{manet} protocols. Our approach is
generally more un-restrictive, as we allow protocol implementations in C++, whereas the paper is restricted to
a fixed formalism. One of the major advantages of using Rebeca is that the modelling language makes it
possible model check and verify \gls{manet} protocol implementations, and explore the full state space for the
model of a protocol.\medbreak

The paper \doublequote{Modeling and Evaluation of Wireless Sensor Network Protocols by Stochastic Timed
Automata}~\cite{article:maeofwsnpbsta} proposes a method to analyse and evaluate \gls{wsn} protocols using
Stochastic Timed Automata, along with the non-deterministic behaviour of \gls{wsn}s, such as lossy
transmission and dynamic topologies. The authors utilise statistical model checking to evaluate the
performance of \gls{wsn} protocols, as well as checking the correctness of the protocols. This approach is
somewhat similar to the approach in \cite{DBLP:conf/fsen/YousefiGK15} and
\cite{DBLP:journals/fac/YousefiGK17}, but the protocols evaluated are instead modelled using the UPPAAL model
checker and does include non-deterministic behaviour. They show in the paper that their method can
model and evaluate network topologies of up to 100 nodes by using statistical model checking in UPPAAL.
\medbreak

In the paper \doublequote{Simulating MANETS: A Study using Satellites with AODV and AntHocNet}~\cite{7813192},
the authors present a network simulator for satellite networks called SatSim. The authors argue that a
satellite constellation can be thought of as an extreme example of a \gls{manet}. SatSim includes features
such as a bit error rate, determined for each packet using a representative link budget, where packets are
randomly dropped if the bit error rate exceeds a specific threshold. This approach is similar to ours in that
we also use a link budget (the link \gls{pathloss} model) to compute the probability for packet errors. Our
approach differs, in that we expand upon this by also simulating collisions, caused by interfering
transmitters.

% Compare our approach with others; mention both strengths and weaknesses

% Should answer the following questions:
    % Where do the ideas come from?
    % Have similar ideas been published/proposed earlier?
    % What is really new in the paper?