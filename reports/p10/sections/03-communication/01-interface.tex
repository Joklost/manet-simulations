\section{Abstract Hardware}\label{sec:interface}
To simulate the physical radio hardware in our simulations, we define a hardware interface consisting of two
basic hardware functions, Transmit, and Listen. The two hardware function to emulate the functionality of
radio hardware. The Transmit function takes a packet of arbitrary data as input, and the node transmits the
packet to nearby listening nodes. The duration of the transmission is computed based on the size of the
packet, using \autoref{eq:transmission-time} from \autoref{sec:hardwarephysics}. The Listen function takes a
duration as input, and the node listens for a packet, for the duration. If a packet is received, the node
stops listening, and returns the packet immediately. If nothing is received while listening, the function will
return \KwNull. 

%\todo[inline]{Say what is the hardware the we are now implementing. Say what the concrete values for the hardware is.}
%

%Additionally, the hardware interface includes a fourth function, Inform. This function is not 

%. The hardware
%interface consists of four functions: Transmit, Listen, 
%Pseudo code descriptions of the four functions can be found in \autoref{sec:hwfuncspseudo}.

%To emulate physical radio hardware, we first need to define an
% explain the physical reachi hardware? NeoCortec NC1000 modules, Texas Instruments CC1110 radio chipset
%\begin{description}[style=nextline,font=\normalfont]
%    \item[Transmit(\textit{packet})] The Transmit function takes a packet of arbitrary data as input, and the
%          node transmits the packet to nearby listening nodes. The duration of the transmission is computed
%          based on the size of the packet, using \autoref{eq:transmission-time} from
%          \autoref{sec:hardwarephysics}.
%    \item[Listen(\textit{duration})] The Listen function takes a duration as input, and the node listens for a
%          packet. If a packet is received, the node stops listening, and returns the packet immediately. If
%          nothing is received while listening, the function will return \KwNull. 
%    %\item[Sleep(\textit{duration})] The Sleep function takes a duration, and puts the node to sleep for the
%    %      duration. 
%    %\item[Inform()] The Inform function
%\end{description}


\section{2-Phase Broadcast and Receive}\label{sec:2pcomm}
As mentioned previously in \autoref{sec:interface}, the time required for transmitting a packet is dependent
on the size of the packet. To aid with the timing of transmitting and listening, we introduce a two-phase
system. The two-phase system consists of two functions, Broadcast (\autoref{algo:2phase-broadcast}) and
Receive (\autoref{algo:2phase-receive}).

\begin{algorithm}[ht]
    \DontPrintSemicolon
    \SetKwFunction{FBroadcast}{Broadcast}
    \SetKwProg{Fn}{Function}{}{}
    \textit{delay} $\leftarrow$ short delay between first and second transmission\;
    \;
    \Fn{\FBroadcast{\textit{packet}}}{
        \textit{header} $\leftarrow$ construct header packet with $|\mathit{packet}|$\;\label{algo:2phase-broadcast:constructheader}
        Transmit(\textit{header})\;\label{algo:2phase-broadcast:transmitheader}
        Sleep(\textit{delay})\;
        Transmit(\textit{packet})\;
    }

    \caption{The Broadcast function.}
    \label{algo:2phase-broadcast}
\end{algorithm}

The Broadcast function takes a \textit{packet} of arbitrary data as input. With the packet of data, the
function constructs a header packet that includes the size of the input \textit{packet}
(\autoref{algo:2phase-broadcast:constructheader}). The size of the header packet is dependent on the design on
the radio hardware. The header packet is transmitted via a call to the Transmit function, after which the
function sleeps for a pre-defined \textit{delay}. This delay allows for any listening nodes to start listening
for the actual packet, after receiving the header packet, before the actual packet is being transmitted. The
pre-defined delay is a global value, known by all nodes.

\begin{algorithm}[ht]
    \DontPrintSemicolon
    \KwResult{A packet or \KwNull}
    \SetKwFunction{FReceive}{Receive}
    \SetKwProg{Fn}{Function}{}{}
    \textit{delay} $\leftarrow$ short delay between first and second transmission\;
    \;
    \Fn{\FReceive{\textit{duration}}}{
        \textit{data} $\leftarrow$ Listen(\textit{duration})\;\label{algo:2phase-receive:listen1}
        \If{\textit{data} $\neq$ header packet}{\label{algo:2phase-receive:isheader}
            \KwRet \KwNull\;
        }
        $\mathit{packetsize} \leftarrow$ extract size of packet from header \textit{data}\;
        $d^\prime \leftarrow \mathit{transmissiontime}(\mathit{packetsize})$\;
        $\mathit{packet} \leftarrow \text{Listen}(d^\prime + \mathit{delay} \cdot 2)$\;
        \tcp{The \textit{packet} returned from Listen may be \KwNull}
        \KwRet \textit{packet}\;
    }

    \caption{The Receive function.}
    \label{algo:2phase-receive}
\end{algorithm}

The Receive function takes a \textit{duration} as input, and begins listening for the duration
(\autoref{algo:2phase-receive:listen1}). The \textit{duration} input is how long a node wants to listen for a
packet. A node stops listening the moment a packet has been received, or after the duration if nothing was
received. The Listen function either returns a packet of data, if anything was received, or \KwNull. The
Receive function checks if the received data is a header packet (\autoref{algo:2phase-receive:isheader}), and
returns \KwNull if the received packet was not a header packet. Otherwise, the size of the incoming packet is
extracted from the header packet, and the expected transmission time is calculated using
\autoref{eq:transmission-time} from \autoref{sec:hardwarephysics}, after which the functions starts listening,
but adds two times the delay to the listening duration, to ensure that the entire packet can be received in
the listening window. Finally, the function returns either the received packet, if any, or \KwNull.