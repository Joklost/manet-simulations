\section{Abstract Hardware Simulation}\label{sec:coordinator}
%\todo[inline]{bibliographical remark: a lot of changes from Controller}
To coordinate the communication between nodes using the hardware interface, we introduce the Coordinator. The
Coordinator works by gathering actions (\textit{transmit}, \textit{listen}, \textit{sleep}, and
\textit{inform}) sent from nodes using the hardware functions from \autoref{sec:hwfuncspseudo}. The pseudo
code description of the Coordinator can be seen in \autoref{algo:mpicoordinator}.

\subsection{Virtual Time}\label{sec:virtual-time}
%\bibtodo{some of 3.1.1 from old report has been included here}
When simulating radio communication it is essential that we factor in time. Time is a large part of a
\gls{mac} protocol, as is the case with the Slotted ALOHA~\cite{Roberts:1975:APS:1024916.1024920} and
LMAC~\cite{paper:lmac_protocol} protocols that rely heavily on time slots to be accurate to a point where
packets are not dropped because a node starts listening for a synchronisation packet too early or too late.
Additionally, as we use an \acrshort{mpi} for the emulation of the radio hardware, we need to account for the
asynchronous nature of the \acrshort{mpi}. For example, simulating with 50 nodes on a machine with 8
cores would start 51 individual processes, and the operating system would have to schedule accordingly,
meaning that not every process (node) would be able to work at all times. Furthermore, as some protocols are
designed to conserve power, as the radio hardware might be battery powered, the nodes will spend a large
amount of time sleeping, which can create time periods where nothing happens during a simulation. To aid with
this, we introduce the notion of virtual time. Virtual time helps us with coordinating the communication
between the nodes, as rather than communicating directly with each other using the \gls{mpi}, nodes
communicate through the Coordinator. \smallbreak

For our simulations, virtual time is, essentially, the real-time a node spends during the execution of a
\gls{mac} protocol, plus the time spent on transmitting, listening or sleeping. As we emulate the radio,
transmitting or receiving packets is a matter of sending messages with the \gls{mpi}, and we do not need to
power down the hardware to sleep. To track the virtual time, each node keeps a local real-time clock,
measuring the real-time spent during the simulation, and a local time variable. Every time a node either
transmits, listens, or sleeps for a duration, the time difference measured by the real-time clock since the
node last performed an action, as well as the duration of the transmission, listen, or sleep is added to the
local time. This way, any action the node does can be completed almost instantly, and the node may continue
executing the simulation, rather than, for example, have to sleep for 10 seconds in real-time, where the
process does nothing.

\subsection{Hardware Functions}\label{sec:hwfuncspseudo}
%\bibtodo{changed according to new functionality}
In this section, we present the four hardware functions nodes use to communicate with the Coordinator. Common
for all four hardware functions is that they all construct an Action object
(\autoref{algo:mpicoordinator:actiontuple} in \autoref{algo:mpicoordinator}) and send it to the Coordinator
using the \gls{mpi}. Note that for any action $a$ sent by any node, $a.\mathit{start} \leq a.\mathit{end}$.
The hardware functions depend on the following local state. The local state is unique for each node.
\smallbreak

\textit{clock} $\leftarrow$ \KwNow

\textit{localtime} $\leftarrow$ 0

\textit{id} $\leftarrow$ unique identifier \smallbreak

In the hardware functions, we utilise a special keyword \KwNow, which represents the real-time hardware clocks
of a node. It is assumed that all clocks run at the same speed. The \textit{clock} variable is used to measure
the real-time spent by the node between calls to hardware functions. Initially, we store the current time in
the \textit{clock} variable, and use the \textit{clock} to compute the real-time difference between the
calling of hardware functions and add the difference to the \textit{localtime} variable (e.g.,
\textit{localtime} $\leftarrow$ (\KwNow $-$ \textit{clock}) $+$ \textit{localtime}).

The unique identifier (\textit{id}) of a node is meant to function as the address of a node for passing
messages between nodes and the Coordinator. The identifier of the Coordinator = $0$, and the identifier of the
nodes are in the range $\{ 1, 2, 3, \ldots, N \}$ for $N$ nodes.

\begin{algorithm}[ht]
    \DontPrintSemicolon
    \SetKwFunction{FTransmit}{Transmit}
    \SetKwProg{Fn}{Function}{}{}
    
    \Fn{\FTransmit{\textit{packet}}}{
        $\mathit{localtime} \leftarrow (\KwNow - \mathit{clock}) + \mathit{localtime}$\;
        $duration \leftarrow \mathit{transmissiontime}(|\mathit{packet}|)$\;
        $\mathit{end} \leftarrow \mathit{localtime} + \mathit{duration}$\;
        $a \leftarrow (\mathit{transmit}, \mathit{id}, \mathit{localtime}, \mathit{end}, \mathit{packet})$\;
        \KwSend $a$ \KwTo Coordinator\;
        $\mathit{localtime} \leftarrow \mathit{end}$\;
        $\mathit{clock} \leftarrow$ \KwNow\;
    }

    \caption{The Transmit function.}
    \label{algo:hwfuncstransmit}
\end{algorithm}

The Transmit (\autoref{algo:hwfuncstransmit}) function transmits a data packet. The packet is sent
to the Coordinator using the \gls{mpi}, and the Coordinator takes care of distributing the packet to
neighbouring nodes listening for packets. The duration of a transmission is computed based on the
\gls{baudrate} (the number of bits the hardware can transmit per second~\cite{website:baudrate-mathworks}) as
well as the size of the packet, using \autoref{eq:transmission-time} found in \autoref{sec:hardwarephysics}.
% or our hardware, we assume a \gls{baudrate} $f_s = 34800$ Hz, 

After computing the duration, the transmit action is sent to the Coordinator, the \textit{localtime} variable
is set to the end time and the clock is set to \KwNow before the function returns. \medbreak

\begin{algorithm}[ht]
    \DontPrintSemicolon
    \KwResult{A \textit{packet} or \KwNull}
    \SetKwFunction{FListen}{Listen}
    \SetKwProg{Fn}{Function}{}{}
    
    \Fn{\FListen{\textit{duration}}}{
        $\mathit{localtime} \leftarrow (\KwNow - \mathit{clock}) + \mathit{localtime}$\;
        $a \leftarrow (\mathit{listen}, \mathit{id}, \mathit{localtime}, \mathit{localtime} + \mathit{duration}, \KwNull)$\;
        \KwSend $a$ \KwTo Coordinator\;
        $\mathit{localtime} \leftarrow$ \KwAwait $\mathit{end}$ \KwFrom Coordinator\; \label{algo:hwfuncslisten:awaitend}
        $\mathit{packet} \leftarrow$ \KwAwait $\mathit{packet}$ \KwFrom Coordinator\;
        \tcp{The \textit{packet} returned from Coordinator may be \KwNull}
        \textit{clock} $\leftarrow$ \KwNow\;
        \KwRet \textit{packet}\;
    }
    
    \caption{The Listen function.}
    \label{algo:hwfuncslisten}
\end{algorithm}

The Listen (\autoref{algo:hwfuncslisten}) function takes a duration as input and sends a listen action to the
Coordinator. After sending its action, the function waits for a response from the Coordinator at
\autoref{algo:hwfuncslisten:awaitend}. The \KwAwait keyword is blocking, which means that no other functions
can be called from a node while the node is listening for a packet. When the Coordinator processes a listen
action, two messages will be sent to the node. The first is the end time, which is assigned to the
\textit{localtime} variable, and the second is the packet received (if any). If no packet has been received,
the end time received from the Coordinator will be the same as the end time in the action sent to the
Coordinator (\textit{localtime} $+$ \textit{duration}), and the packet received will be \KwNull. If a packet
has been received, the Coordinator will send the time when the packet was received, with the packet following
right after.

At most a single packet may be received on a call to the Listen function, but depending on the number
of transmissions in the same time interval, no packet could be received, as multiple transmissions either will
provide interference for each other, creating collisions, or no transmissions may have happened in the time
interval.

After receiving a response from the Coordinator, the function will set the clock variable to \KwNow, and
return either the packet or \KwNull. \medbreak

\begin{algorithm}[ht]
    \DontPrintSemicolon
    \SetKwFunction{FSleep}{Sleep}
    \SetKwProg{Fn}{Function}{}{}
    
    \Fn{\FSleep{\textit{duration}}}{
        \textit{localtime} $\leftarrow$ (\KwNow $-$ \textit{clock}) $+$ \textit{localtime}\;
        \textit{end} $\leftarrow$ \textit{localtime} $+$ \textit{duration}\;
        $a$ $\leftarrow$ (\textit{sleep}, \textit{id}, \textit{localtime}, \textit{end}, \KwNull)\;
        \KwSend $a$ \KwTo Coordinator\;
        \textit{localtime} $\leftarrow$ \textit{end}\;
        \textit{clock} $\leftarrow$ \KwNow\;
    }
    
    \caption{The Sleep function.}
    \label{algo:hwfuncssleep}
\end{algorithm}

The Sleep (\autoref{algo:hwfuncssleep}) function takes a duration as input and sends a \textit{sleep} action to the
Coordinator. As no response is expected of the Coordinator, the function sets the \textit{localtime} and
\textit{clock} variables immediately after sending the action. \medbreak

\begin{algorithm}[ht]
    \DontPrintSemicolon
    \SetKwFunction{FInformLocalTime}{Inform}
    \SetKwProg{Fn}{Function}{}{}
    
    \Fn{\FInformLocalTime{}}{
        \textit{localtime} $\leftarrow$ (\KwNow $-$ \textit{clock}) $+$ \textit{localtime}\;
        $a$ $\leftarrow$ (\textit{inform}, \textit{id}, \textit{localtime}, \textit{localtime}, \KwNull)\;
        \KwSend $a$ \KwTo Coordinator\;
        \textit{clock} $\leftarrow$ \KwNow\;
    }
    
    \caption{The Inform function.}
    \label{algo:hwfuncsupdatelocaltime}
\end{algorithm}

The Inform function is equivalent to the Sleep function in the sense that it behaves like a call to the Sleep
function with the duration set to 0. The function is included in the case none of the other hardware
functions is applicable, e.g., in the case where the node is performing longer computations. Regularly
informing the Coordinator of a nodes \textit{localtime} will allow the Coordinator to continually process actions from
other nodes.

\subsection{Coordinator}\label{sec:coordinatorpseudo}
The pseudo code of the Coordinator is separated into three parts, where each of the parts is repeatedly
executed in sequence until the protocol terminates. The first part takes care of receiving actions from
nodes, the second part maintains and cleans the $\mathit{transmits}$ set of \textit{transmit} actions, and the
third part processes, and removes, actions from the $\mathit{waiting}$ queue. \medbreak

The Coordinator works by continuously awaiting actions from any node
(\autoref{algo:mpicoordinator:action-await}). The \KwAwait keyword is blocking, and blocks until any node
sends an action to the Coordinator. An action is the 5-tuple Action = $(\mathit{type},\ \mathit{source},\
\mathit{start},\ \mathit{end},\ \mathit{packet})$. Accessing an element of an action is done with the dot
($.$) operator, using named access. For example, if node $1$ starts listening at time $2$ and ends listening
at time $5$, we have the \textit{listen} action $a = (\mathit{listen}, 1, 2, 5, \KwNull)$. We can access the
source of the action using the dot operator as such: $a.source = 1$. The $\mathit{type}$ element denotes the
type of the action, and is one of either \textit{transmit}, \textit{listen}, \textit{sleep}, or
\textit{inform}. The $\mathit{source}$ element is the unique identifier of the source node that submitted the
action to the Coordinator. The $\mathit{start}$ and $\mathit{end}$ elements are timestamps for points of time
in the execution, where $\mathit{start} \leq \mathit{end}$. Finally, the $\mathit{packet}$ element is the data
packet sent during transmission or \KwNull for any action where $\mathit{type} \neq \mathit{transmit}$.
\medbreak

Whenever an action is submitted to the Coordinator (\autoref{algo:mpicoordinator:action-await}), the action is
enqueued in the $\mathit{waiting}$ queue and added to the \textit{discovered} set. The $\mathit{waiting}$
queue is a priority queue of Action objects, where the actions are ordered by their end time, with
\textit{transmit} actions before \textit{listen} actions, in case of a tie. The \textit{discovered} set is
a set used to track all actions submitted to the Coordinator. The ordering of \textit{sleep} and
\textit{inform} in relation to \textit{transmit} or \textit{listen} actions in the \textit{waiting} queue are
irrelevant. If the received action is a \textit{transmit} action, the action is also added to the
$\mathit{transmits}$ set (\autoref{algo:mpicoordinator:message-transmit}). Note that any action we receive
from a particular node during this part will always have a start time greater than or equal to the end time of
the last action received from that node. After receiving an action, the Coordinator continues to the second
part. \medbreak

The $\mathit{transmits}$ set is used to gather any \textit{transmit} actions that may cause interference when
processing \textit{transmit} actions in the third part of the Coordinator. To make sure that the size
of the $\mathit{transmits}$ set does not grow indefinitely, we remove \textit{transmit} actions that will not
interfere with future \textit{transmit} actions. To do this we check that the $\mathit{waiting}$ queue
contains at least one action from each node (\autoref{algo:mpicoordinator:ifactions}). Next, we find the
earliest start time of all actions in the $\mathit{waiting}$ queue. With this, we can remove any
\textit{transmit} actions where the end time is strictly less than the earliest start time found in the
$\mathit{waiting}$ queue (\autoref{algo:mpicoordinator:cleantransmits}). \medbreak

Finally, in the third part of the Coordinator, we process actions from the \textit{waiting} queue. The
Coordinator only processes actions when the \textit{waiting} queue fulfils the same condition as in the second
part, where the \textit{waiting} queue has to contain at least one action from each node. While this is the
case, the Coordinator dequeues a single action (\autoref{algo:mpicoordinator:dequeue}) from the
\textit{waiting} queue. Any \textit{sleep} or \textit{inform} actions are processed implicitly, as they only
have to be present in the \textit{waiting} queue, to satisfy the condition of the loop. Only the
\textit{transmit} and \textit{listen} actions need processing by the Coordinator. \medbreak 

When processing a \textit{transmit} action, the Coordinator first gathers all \textit{transmit} actions from
the \textit{transmits} set, that interfere with the \textit{transmit} action being processed
(\autoref{algo:mpicoordinator:foreachinterfere}). A \textit{transmit} actions causes interference with
another if at any point in time their time intervals intersect
(\autoref{algo:mpicoordinator:transmitintersects}). All applicable \textit{transmit} actions are stored in the
\textit{interferers} set, to be used when computing the probability for packet error. Next, the Coordinator
iterates through all \textit{listen} actions in the \textit{waiting} queue, and if the time interval of the
\textit{transmit} action is fully within the time interval of the \textit{listen} action, the \textit{listen}
action is considered relevant for the \textit{transmit} action, and we compute the probability for the packet
being received by the listening node (\autoref{mpi:coordinator:pep}). The probability of the packet being
received is computed using the probability for packet error function $P_p$ (\autoref{eq:pep} in
\autoref{sec:radiomodel}). The probability for packet error function is called using the source of the
\textit{listen} action $l$, the source of the \textit{transmit} action $a$, the \textit{interferers} set, the
size of the packet, as well as the end time of the \textit{transmit} action $a$. Finally, we randomly chose
whether the packet should be received by using the computed probability $p$
(\autoref{algo:mpicoordinator:shouldreceive}). If the listening node has been chosen to receive the packet,
the \textit{listen} action $l$ is removed from the \textit{waiting} queue
(\autoref{algo:mpicoordinator:removelisten}), and the end time of the \textit{transmit} action $a$, as well as
the packet is sent to the source of the \textit{listen} node. \medbreak

Processing a \textit{listen} action (\autoref{algo:mpicoordinator:unblock2}) is trivial, as all the
Coordinator does is send the end time of the \textit{listen} action and \KwNull to the source of the
\textit{listen} action. We do this as dequeuing a \textit{listen} action means that no transmission has been
received by the listening node. When the Coordinator responds to the source of the \textit{listen} action,
the node is unblocked and may continue executing its protocol. \medbreak

The complete source code for the C++ implementation can be found on GitHub:

{\small \url{https://github.com/Joklost/manet-simulations/tree/master/src/coordinator}} \medbreak

\begin{algorithm}[H]
    \DontPrintSemicolon
    \SetKwFunction{FCoordinator}{Coordinator}
    \SetKwProg{Fn}{procedure}{}{}

    Action = $(\mathit{type},\ \mathit{source},\ \mathit{start},\ \mathit{end},\ \mathit{packet})$\; \label{algo:mpicoordinator:actiontuple}
    \;

    \Fn{\FCoordinator{}}{
        $\mathit{waiting}$ $\leftarrow$ priority queue of Action objects, ordered by $\mathit{end}$ time, \;
        \hspace{18.5mm}\textit{sleep} and \textit{inform} before \textit{transmit}, \textit{transmit} before \textit{listen}\;
        $\mathit{discovered} \leftarrow$ empty set of Action objects\; \label{algo:mpicoordinator:discovered}
        $\mathit{transmits}$ $\leftarrow$ empty set of Action objects\;
        %\;

        \Repeat{\textit{protocol terminates}}{
            \tcp{Part 1. Handle incoming message.}
            $a$ $\leftarrow$ \KwAwait Action \KwFrom any node\; \label{algo:mpicoordinator:action-await}
            \KwAppend $a$ \KwTo $\mathit{discovered}$\;
            \If{$a$.$\mathit{type}$ = \textit{transmit}}{ \label{algo:mpicoordinator:message-transmit}
                \KwAppend $a$ \KwTo $\mathit{transmits}$\;
            }
            \KwEnqueue $a$ \KwTo $\mathit{waiting}$\;
            \;
            \tcp{Part 2. Clean $\mathit{transmits}$ set.}
            \If{each node has at least one Action \KwInn $\mathit{waiting}$}{ \label{algo:mpicoordinator:ifactions}
                $\mathit{starttime}$ $\leftarrow$ earliest $\mathit{start}$ time of all actions \KwInn $\mathit{waiting}$\;
                \ForEach{$t$ $\in$ $\mathit{transmits}$ \KwWhere $t$.$\mathit{end}$ < $\mathit{starttime}$}{ \label{algo:mpicoordinator:cleantransmits}
                    \KwRemove $t$ \KwFrom $\mathit{transmits}$\;
                }
            }
            \;
            \tcp{Part 3. Process $\mathit{waiting}$ actions.}
            \While{each node has at least one Action \KwInn $\mathit{waiting}$}{ \label{algo:mpicoordinator:whileactions}
                $a$ $\leftarrow$ \KwDequeue Action \KwFrom $\mathit{waiting}$\; \label{algo:mpicoordinator:dequeue}

                \If{$a$.$\mathit{type}$ = \textit{transmit}}{ \label{line:action-transmit}
                    $\mathit{interferers}$ $\leftarrow$ empty set of node identifiers\;
                    \ForEach{$t$ $\in$ $\mathit{transmits}$ \KwWhere $t$ $\neq$ $a$}{ \label{algo:mpicoordinator:foreachinterfere}
                        \If{$a$.$\mathit{end}$ $\geq$ $t$.$\mathit{start}$ \KwAnd $a$.$\mathit{start}$ $\leq$ $t$.$\mathit{end}$}{ \label{algo:mpicoordinator:transmitintersects}
                            \tcp{Transmissions intersect.}
                            \KwAppend $t$.$\mathit{source}$ \KwTo $\mathit{interferers}$\;
                        }
                    }

                    \ForEach{$l$ $\in$ $\mathit{waiting}$ \KwWhere $l$.$\mathit{type}$ = \textit{listen}}{
                        \If{$a$.$\mathit{start}$ $\geq$ $l$.$\mathit{start}$ \KwAnd $a$.$\mathit{end}$ $\leq$ $l$.$\mathit{end}$}{
                            $p$ $\leftarrow$ $P_p$($l$.$\mathit{source}$, $a$.$\mathit{source}$, $\mathit{interferers}$, |$a$.$\mathit{packet}$|, $a$.$\mathit{end}$)\; \label{mpi:coordinator:pep}
                            $shouldreceive$ $\leftarrow$ randomly choose based on $p$\;
                            \If{$shouldreceive$}{ \label{algo:mpicoordinator:shouldreceive}
                                \KwRemove $l$ \KwFrom $\mathit{waiting}$\; \label{algo:mpicoordinator:removelisten}
                                \KwSend $a$.$\mathit{end}$ \KwTo $l$.$\mathit{source}$\;
                                \KwSend $a$.$\mathit{packet}$ \KwTo $l$.$\mathit{source}$\;
                            }
                        }
                    }
                }

                \ElseIf{$a$.$\mathit{type}$ = \textit{listen}}{ \label{line:action-listen}
                    \KwSend $a$.$\mathit{end}$ \KwTo $a$.$\mathit{source}$\; \label{algo:mpicoordinator:unblock2}
                    \KwSend \KwNull \KwTo $a$.$\mathit{source}$\; \label{algo:mpicoordinator:sendnull}
                }
                \tcp{\textit{sleep}/\textit{inform} actions are handled implicitly.}
            }
        }
    }
    \caption{The Coordinator procedure.}
    \label{algo:mpicoordinator}
\end{algorithm}