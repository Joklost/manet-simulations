\section{Correctness}
\todo[inline]{introduction}

\subsection{Invariants}\label{sec:coordinator-invariants}

With a set of $N$ unique node identifiers where $\mathit{nodes} = \{ 1, 2, 3, \ldots, N \}$, the following
invariants hold:

\begin{enumerate}
    \item For all $n \in \mathit{nodes}$, there exists at most one $a \in \mathit{waiting}$ such that
          $a.\mathit{source} = n$ and $a.\mathit{type} = \mathit{listen}$.
          \begin{itemize}
              \item There is at most one action with the $\mathit{listen}$ type from each node in the
                    \textit{waiting} queue.
          \end{itemize}
    \item For all $a, b \in \mathit{waiting}$, if $a.\mathit{source} = b.\mathit{source}$ and $a.\mathit{type}
              = \mathit{listen}$ then $b.\mathit{end} \leq a.\mathit{start}$.
          \begin{itemize}
              \item If a node has a $\mathit{listen}$ action in the \textit{waiting} queue, no other actions
                    may be present after this.
          \end{itemize}
    \item For all $a \in \mathit{discovered}$, if $a.\mathit{type} = \mathit{transmit}$ and there exists $b
              \in \mathit{waiting}$ such that $b.\mathit{type} = \mathit{transmit}$ and $a.\mathit{end} \geq
              b.\mathit{start}$ and $a.\mathit{start} \leq b.\mathit{end}$ then $a \in \mathit{transmits}$.
          \begin{itemize}
              \item Every \textit{transmit} action submitted to the Coordinator that could possibly interfere
                    with any \textit{transmit} action on the \textit{waiting} queue must be present in the
                    \textit{transmits} set.
          \end{itemize}
\end{enumerate}

The first and second invariants are satisfied both by the implementation of the Listen function
(\autoref{algo:hwfuncslisten} in \autoref{sec:hwfuncspseudo}), and by the Coordinator. The Listen function
uses the blocking \KwAwait keyword on \autoref{algo:hwfuncslisten:awaitend} in \autoref{algo:hwfuncslisten},
to wait for a response from the Coordinator. The Coordinator, in turn, sends a response to the source of the
\textit{listen} action only after the action already has been removed from the \textit{waiting} queue at
\autoref{algo:mpicoordinator:dequeue} or \autoref{algo:mpicoordinator:removelisten} in
\autoref{algo:mpicoordinator}. Due to the blocking nature of the \KwAwait keyword, and the fact that a node is
only able to continue execution after a response have been received from the Coordinator, the invariants are
satisfied, as it is impossible for a node to send more actions to the Coordinator, while a \textit{listen}
action from the same node is already on the \textit{waiting} queue.
\smallbreak

The third invariant is satisfied as long as any \textit{transmit} actions that could possibly interfere with
any \textit{transmit} action in the \textit{waiting} queue is present in the \textit{transmits} set. With the
\textit{discovered} set we can track the \textit{transmit} actions that have been removed from the
\textit{waiting} queue, as these could interfere with future \textit{transmit} actions. A \textit{transmit}
action $a$ interferes with another \textit{transmit} action $b$, as long as the time interval for the actions
intersect at some point ($a.\mathit{end} \geq b.\mathit{start}$ and $a.\mathit{start} \leq b.\mathit{end}$).
At \autoref{algo:mpicoordinator:cleantransmits} in \autoref{algo:mpicoordinator} we have the condition that we
only remove $\mathit{transmit}$ actions from the $\mathit{transmits}$ set if the end time of any action in the
set is less than the earliest start time of all actions in \textit{waiting}. Any \textit{transmit} actions
fulfilling this condition is safe to remove, as they are no longer be able to intersect with any
\textit{transmit} actions in the waiting queue. Additionally, if we have a $\mathit{transmit}$ action both in
the \textit{waiting} queue and $\mathit{transmits}$ set, we will not be able to remove it from the
$\mathit{transmits}$ set, before the action has been removed from the \textit{waiting} queue, according to
the condition at \autoref{algo:mpicoordinator:cleantransmits}. If we have an action $t \in \mathit{waiting}$
where $t \in \mathit{transmits}$ then the earliest start time of all actions in the \textit{waiting} queue
is at least $t.\mathit{start}$, which means that we will not be able to remove $t$ from the \textit{transmits}
set, as $t.\mathit{end} \nless t.\mathit{start}$, and we know from the implementation of the hardware
functions that for any action $a$, $a.\mathit{start} \leq a.\mathit{end}$, as shown in
\autoref{sec:hwfuncspseudo}.

\subsection{Methodology}\label{sec:correctnessmethods}
Suppose that at a point in time $t$ in a real execution a \textit{listen} actions ends, and the listening node
receives a packet with some probability $p$ computed, using the packet error probability function $P_p$. We
want to show that in a simulation, with asynchronous virtual-time execution (as implemented in
\autoref{algo:mpicoordinator}), the same listening node would receive the packet, with the same computed
probability. \autoref{tikz:coordinatormsc0} is a visual representation of a real execution for an
arbitrary wireless communication protocol where five nodes communicate through wireless radio broadcasts,
without the Coordinator. Vertically, each node has a timeline of actions they perform throughout the
execution, where each of the actions are represented as a rectangle (\textit{listen}, \textit{transmit}, or
\textit{sleep}) or a line (\textit{inform}). The rectangles and the line represent the start and end time of
the actions, where the line has the same start and end time. Each action has an identifier, e.g., \T{1}{5},
and an action is the 5-tuple Action (\autoref{algo:mpicoordinator:actiontuple} in
\autoref{algo:mpicoordinator}). The \T{1}{5} action is a \textit{transmit} action where \Node{1} broadcasts
some arbitrary data starting at time 2 and ending at time 5: $\T{1}{5} = (\mathit{transmit}, 1, 2, 5,
\mathit{data})$. The horizontal arrows going from the end of a \textit{transmit} action to a \textit{listen}
action represent a packet received by the listening node. If no arrow originates from a \textit{transmit}
action, nothing was received of the transmission. In a real execution, a node stops listening after the packet
has been received. This is presented in the figure by the \L{4}{14} action, where \Node{4} receives a packet
from \Node{5} at time 12, and stops listening.
\medbreak

The following is the result of processing each of the $\mathit{transmit}$ actions from
\autoref{tikz:coordinatormsc0}, and serve as the timeline of the processing of the $\mathit{transmit}$ actions
throughout both the real execution, and the simulation:
%
\begin{description}[leftmargin=2em,style=nextline]
    \item[\T{4}{3}] The packet originating from \Node{4} (\T{4}{3}) was dropped by \Node{3} (\L{3}{6}) due to
          interference from \Node{1} (\T{1}{5}). \Node{2} (\L{2}{5}) started listening too late to receive the
          packet.
    \item[\T{1}{5}] The packet originating from \Node{1} (\T{1}{5}) was received by \Node{2} (\L{2}{5}) at
          time 5, but dropped by \Node{3} (\L{3}{6}) due to interference from \Node{4} (\T{4}{3}).
    \item[\T{1}{10}] The transmission from \Node{1} (\T{1}{10}) was not received by any listening nodes, as
          \Node{5} (\L{5}{9}) stopped listening too soon, while \Node{3} (\L{3}{10}) started listening too
          late.
    \item[\T{2}{10}] The packet originating from \Node{2} (\T{2}{10}) was dropped by \Node{3} (\L{3}{10}) due
          to interference from \Node{1}.
    \item[\T{5}{12}] The packet originating from \Node{5} (\T{5}{12}) was received by \Node{4} (\L{4}{14}) at
          time 12, but dropped by \Node{2} (\L{2}{14}) due to distance.
    \item[\T{3}{16}] The transmission from \Node{3} (\T{3}{16}) was not received by any listening nodes, as
    \Node{1} (\L{1}{18}) started listening too late.
    %The packet originating from \Node{3} (\T{3}{16}) was dropped by \Node{1} (\L{1}{18}) due
    %      to interference from \Node{4} (\T{4}{17}).
    \item[\T{4}{17}] The packet originating from \Node{4} (\T{4}{17}) was dropped by \Node{1} (\L{1}{18}) due
          to interference from \Node{2} (\T{2}{18}), \Node{3} (\T{3}{16}), and \Node{5} (\T{5}{18}).
    \item[\T{2}{18}] The packet originating from \Node{2} (\T{2}{18}) was dropped by \Node{1} (\L{1}{18}) due
          to interference from \Node{4} (\T{4}{17}), and \Node{5} (\T{5}{18}).
    \item[\T{5}{18}] The packet originating from \Node{5} (\T{5}{18}) was dropped by \Node{1} (\L{1}{18}) due
          to interference from \Node{2} (\T{2}{18}), and \Node{4} (\T{4}{17}).
    \item[\T{3}{20}] The packet originating from \Node{3} (\T{3}{20}) was received by \Node{4} (\L{4}{20}).
\end{description}

With the invariants outlined above, we want to prove that the packet would be received with the same
probability in a virtual-time execution, as it would in a real execution. Due to the asynchronous nature
of executing a wireless communication protocol using an \gls{mpi}, it is not as simple as presenting a single
slice of the execution. Instead, to illustrate snapshots of a given point in the execution of the Coordinator
we introduce the concept of a cut: A cut consist of two horizontal lines, \ProcessedLine\ and \SubmittedLine.
Everything above the first (\ProcessedLine) has already been processed by the Coordinator, and everything
below the second (\SubmittedLine) has not yet been submitted to the Coordinator. We use a cut to show the
content of the data structures within the Coordinator, where we see the order of currently queued actions in
the \textit{waiting} queue, as well as the actions in the $\mathit{transmits}$ set.

\begin{figure}[H]
    \centering
    % Diagram
    \CoordinatorFigure{

        \draw[line width=0.45mm, ->] (legend1-5.west) -- (legend1-5.east);
        \fill (legend2-5) node[right=0.28cm] {Received packet};

        \draw[line width=0.45mm, ->] (\Node{1}-5.west) -- (\Node{2}-5.west);
        \draw[line width=0.45mm, ->] (\Node{5}-12.east) -- (\Node{4}-12.east);
        \draw[line width=0.45mm, ->] (\Node{3}-20.west) -- (\Node{4}-20.west);
        \draw[pattern=north west lines, pattern color=gray] (\Node{4}-12.west) rectangle (\Node{4}-14.east);

    }
    \caption{Real execution of an arbitrary protocol with five nodes.}\label{tikz:coordinatormsc0}
\end{figure}

\subsection{Simulated Execution}
\todo[inline]{try to shorten some of the arguments}
Throughout this section we present a simulated, virtual-time, version of the real execution from
\autoref{tikz:coordinatormsc0} using the Coordinator from \autoref{algo:mpicoordinator}. In a simulated
execution, the nodes communicate through the Coordinator, rather than directly with each other.
Throughout the figures in this section, it is assumed that all nodes interact with the Coordinator implicitly.
We utilise cuts to present different different possible scenarios of the Coordinator: No action can be
processed, processing a $\mathit{transmit}$, processing an $\mathit{inform}$ action, processing a
$\mathit{listen}$ action, and finally, processing a $\mathit{sleep}$ action. For each of the different
scenarios we start with a concrete example, basing it on the accompanying figure, and ending with a
generalisation of the scenario. When processing each of the \textit{transmit} actions we follow the
descriptions listed in \autoref{sec:correctnessmethods}.

\subsubsection{Cut 1: Nothing can be processed}
For the first cut we have a scenario where nothing may be processed. Recall that the condition for Part 3 of
the Coordinator procedure is that the waiting queue is only processed if all nodes have at least one action on
the queue. As of this cut, \Node{3} has yet to submit an action to the Coordinator, which means that the
Coordinator is unable to progress from this point until the \L{3}{6} action is submitted at some point, later
in the execution. Note that due to the asynchronous nature of our simulation, and the fact that only
\textit{listen} actions are blocking on the hardware side, it is very possible to have a scenario, where a
node has submitted a large number of actions, as \Node{1} has in the figure, where \Node{1} is currently
waiting for the \L{1}{18} action to be processed, and \Node{3} has yet to submit any actions to the
Coordinator. Additionally, nodes may be doing other work internally, before submitting more actions to the
Coordinator, which is why \Node{4} might not have submitted the \textit{listen} action \L{4}{14} as of this
cut. \Node{1}, \Node{2}, and \Node{5} are not able to submit more actions to the Coordinator, before the
\textit{listen} actions they have submitted have been processed.

\begin{figure}[H]
    \caption{Waiting for \Node{3} to submit an action.}\label{tikz:coordinatormsc1}
    % Diagram
    \CoordinatorFigure{\LineLegend\Processed{(sep0-2.center) -- (sep2-2.center) -- (sep2-1.center) --
            (sep3-1.center) -- (sep4-1.center) -- (sep4-4.center) -- (sep5-4.center)}
        \Submitted{(sep0-18.center) -- (sep1-18.center) -- (sep1-5.center) -- (sep2-5.center) --
            (sep2-1.center) -- (sep3-1.center) -- (sep3-8.center) -- (sep4-8.center) -- (sep4-9.center) --
            (sep5-9.center) -- (sep5-9.center)}} \par

    \begin{minipage}[h]{14.5cm}
        The content of the \textit{waiting} queue and \textit{transmits} set as of this cut is: \smallbreak

        $\mathit{waiting} \leftarrow \Big\langle \T{4}{3}, \I{5}{4}, \T{1}{5}, \L{2}{5}, \S{5}{6}, \S{4}{8},
            \L{5}{9}, \T{1}{10}, \I{1}{12}, \S{1}{15}, \L{1}{18} \Big\rangle$

        $\mathit{transmits} \leftarrow \Big\{ \T{4}{3}, \T{1}{5}, \T{1}{10} \Big\}$ \smallbreak

        The \textit{waiting} queue is a priority queue where actions are ordered by end time, and
        \textit{sleep} and \textit{inform} action must be before \textit{transmit} actions, which must be
        before \textit{listen} actions, if the end time is the same.
    \end{minipage}
\end{figure}

\subsubsection{Cut 2: \textit{transmit} action}
The next cut is a snapshot of the Coordinator directly after \Node{3} has submitted the \L{3}{6} action. With
this action in the \textit{waiting} queue, the Coordinator may begin to process the actions in the
\textit{waiting} queue and \textit{transmits} set. First the Coordinator would check if any \textit{transmit}
actions should be removed from the \textit{transmits} set, but as none of the actions in the set has a start
time earlier than the start time of the earliest action in the \textit{waiting} queue (\L{3}{6} or \T{4}{3},
both with start = 1) nothing can be removed. Next, the Coordinator can begin processing actions in the
\textit{waiting} queue. %\smallbreak

\begin{figure}[H]
    \caption{Processing a \textit{transmit} action.}\label{tikz:coordinatormsc2}
    % Diagram
    \CoordinatorFigure{\LineLegend\Processed{(sep0-2.center) -- (sep2-2.center) -- (sep2-1.center) --
            (sep3-1.center) -- (sep4-1.center) -- (sep4-4.center) -- (sep5-4.center)}
        \Submitted{(sep0-18.center) -- (sep1-18.center) -- (sep1-5.center) -- (sep2-5.center) --
            (sep2-6.center) -- (sep3-6.center) -- (sep3-8.center) -- (sep4-8.center) -- (sep4-9.center) --
            (sep5-9.center) -- (sep5-9.center)}} \par

    \begin{minipage}[h]{14.5cm}
        The content of the \textit{waiting} queue and \textit{transmits} set as of this cut is: \smallbreak
        $\mathit{waiting} \leftarrow \Big\langle \T{4}{3}, \I{5}{4}, \T{1}{5}, \L{2}{5}, \S{5}{6}, \L{3}{6},
            \S{4}{8}, \L{5}{9}, \T{1}{10}, \I{1}{12}, \S{1}{15}, \L{1}{18} \Big\rangle$

        $\mathit{transmits} \leftarrow \Big\{ \T{4}{3}, \T{1}{6}, \T{1}{10} \Big\}$
    \end{minipage}
\end{figure}

The first action in the \textit{waiting} queue is the \textit{transmit} action \T{4}{3}. Recall that when
processing a \textit{transmit} action, the Coordinator iterates through the \textit{transmits} set to find
other \textit{transmit} actions with intersecting time intervals, and that invariant 3 ensures that any
\textit{transmit} action with an intersecting time interval will be in the \textit{transmits} set. In this
case, only the \T{1}{5} action intersects with the \T{4}{3} action, so the source of that action is included
in the \textit{interferers} set. Next, the Coordinator iterates through all \textit{listen} actions in the
\textit{waiting} queue, and only if the time interval for the \textit{transmit} action is fully within the
time interval of any \textit{listen} action, we compute the probability for packet error on a transmission
between the source of the \textit{listen} action ($\L{3}{6}.\mathit{source}$) and the source of the
\textit{transmit} action ($\T{4}{3}.\mathit{source}$). With the probability for packet error, \autoref{eq:pep}
in \autoref{sec:radiomodel}, $p = P_p(\L{3}{6}.\mathit{source}, \T{4}{3}.\mathit{source},
\{\T{1}{5}.\mathit{source} \}, |\T{4}{3}.\mathit{packet}|, \T{4}{3}.\mathit{end})$, we can randomly choose
whether the transmission should be received, or dropped, and either finish processing the \textit{listen}
action by sending the packet to the source of the action, or move on to the next \textit{listen} action. For
the \T{4}{3} action, we assume the packet to be dropped by the listening node. \smallbreak

When processing the \T{4}{3} action we have two cases to consider for this particular scenario: In the first
case, the packet is received by \Node{3} and the \L{3}{6} action is removed from the \textit{waiting} queue.
Should this be the case, the condition for processing actions in the \textit{waiting} queue is no longer
satisfied, and the Coordinator will not be able to process any further actions until the \L{3}{10} action has
been submitted to the Coordinator. In the second case, the packet is dropped and not received by \Node{3}. In
this case, the \L{3}{6} action is still on the \textit{waiting} queue, and the Coordinator can continue
processing actions as the \textit{waiting} queue still contains at least one action from each node.
\medbreak

This holds in general, as every \textit{listen} action will be after any \textit{transmit} action $t$ on the
\textit{waiting} queue, and every \textit{transmit} action that could possibly interfere with $t$ is in the
\textit{transmits} set, as stated by invariant 3. Let us assume that a node $n$ submitted the
\textit{transmit} action $t$, and that $t$ is at the head of the \textit{waiting} queue. When processing $t$,
we want to compute the probability for any node $m$ listening in the time interval of the \textit{transmit}
action $t$ to receive the packet transmitted by $n$. When this is the case, the \textit{listen} action is
relevant to $t$. To do this, we need to make sure that any \textit{listen} action, that is relevant for the
processing of $t$ is on the \textit{waiting} queue, and any \textit{transmit} actions that could possibly
interfere with $t$ is in the \textit{transmits} set, when processing $t$. \smallbreak

First, we know that every \textit{listen} action $l$ that could possibly receive the \textit{transmit} action
$t$ is on the \textit{waiting} queue, as the end time of any $l$ on the \textit{waiting} queue, when
processing $t$, has to be greater than, or equal to, the end time of $t$. Recall that due to the ordering of
the \textit{waiting} queue, we always process any \textit{transmit} action before processing a \textit{listen}
action. In addition to this, when processing $t$, we know that for all \textit{listen} actions $l$, where the
time interval of $t$ is fully within the time interval for $l$, the source node $m$ of $l$ will have only the
$l$ action on the \textit{waiting} queue. We know this as any action $a$ from node $m$ would have an end time
earlier than the start time of $l$, $a.\mathit{end} \leq l.\mathit{start}$, as stated by invariant 1, which
means that the $a$ action would already have been processed, before we process $t$. Additionally, invariant 2
states that if a node has a \textit{listen} action in the \textit{waiting} queue, no other actions may be
present after this. Finally, as we know that the condition for processing actions in the \textit{waiting}
queue is satisfied when we begin processing the \textit{transmit} action $t$, any node $m$ that does not have
a \textit{listen} action that is relevant to $t$, is unable to submit a \textit{listen} action that could
become relevant to $t$, as any future action submitted from $m$ would have a start time greater than, or equal
to, the end time of the previous action submitted from $m$, which we know is greater than the end time of the
action $t$ that is currently being processed. \smallbreak

%Secondly, invariant 3 states that every \textit{transmit} action submitted to the Coordinator (discovered
%actions) that could possibly interfere with any \textit{transmit} action on the \textit{waiting} queue must be
%present in the \textit{transmits} set. This means that when we process the \textit{transmit} action $t$ we can
%be we certain that any discovered \textit{transmit} actions that interfere with $t$, will be in the
%\textit{transmits} set. \todo[inline]{what about non-discovered?} \smallbreak

With these two points, when processing a \textit{transmit} action $t$, we can safely compute the
probability for any \textit{listen} action $l$ relevant to $t$, including the interference from any
interfering \textit{transmit} actions in the \textit{transmits} set.

\subsubsection{Cut 3: \textit{inform} action}
The next cut is a snapshot of the Coordinator directly after the \T{4}{3} action has been processed and
removed from the \textit{waiting} queue. The condition for processing actions remains satisfied, as there is
still at least action from each node in the \textit{waiting} queue. At the head of the \textit{waiting} queue
is the \textit{inform} action \I{5}{4}. Processing an \textit{inform} action is trivial, as no processing is
needed for this action. The action is simply removed from the \textit{waiting} queue, and the Coordinator may
move on to the next action, if the condition is still satisfied. This holds in general for any \textit{inform}
action. %\smallbreak

\begin{figure}[H]
    \caption{Processing an \textit{inform} action.}\label{tikz:coordinatormsc3}
    % Diagram
    \CoordinatorFigure{\LineLegend\Processed{(sep0-2.center) -- (sep2-2.center) -- (sep2-1.center) --
            (sep3-1.center) -- (sep3-3.center) -- (sep4-3.center) -- (sep4-4.center) -- (sep5-4.center)}
        \Submitted{(sep0-18.center) -- (sep1-18.center) -- (sep1-5.center) -- (sep2-5.center) --
            (sep2-6.center) -- (sep3-6.center) -- (sep3-8.center) -- (sep4-8.center) -- (sep4-9.center) --
            (sep5-9.center) -- (sep5-9.center)}} \par

    \begin{minipage}[h]{14.5cm}
        The content of the \textit{waiting} queue and \textit{transmits} set as of this cut is: \smallbreak

        $\mathit{waiting} \leftarrow \Big\langle \I{5}{4}, \T{1}{5}, \L{2}{5}, \S{5}{6}, \L{3}{6}, \S{4}{8},
            \L{5}{9}, \T{1}{10}, \I{1}{12}, \S{1}{15}, \L{1}{18} \Big\rangle$

        $\mathit{transmits} \leftarrow \Big\{ \T{4}{3}, \T{1}{6}, \T{1}{10} \Big\}$
    \end{minipage}
\end{figure}

%Note that the only change in the \textit{waiting} queue, from the previous cut, is the removal of the \T{4}{3}
%action, which still remains in the \textit{transmits} set, as the end time of the \T{4}{3} action is still
%greater than, or equal to, the start time of the \L{3}{6} action, and causes interference for the \T{1}{5}
%action (the next action in the queue). Additionally, no new actions have been submitted between the 2nd and
%3rd cuts, as a new action is received only if the condition for processing actions is no longer satisfied.
%\medbreak

\subsubsection{Cut 4: \textit{listen} action}
This cut shows a snapshot of the Coordinator directly after the \T{3}{16} action has been submitted. With this
action on the \textit{waiting} queue the condition for processing actions has been satisfied, and the
Coordinator may process the first action on the queue, the \L{2}{14} action. For this cut, there is two
interesting points to note. First, when a \textit{listen} action is at the head of the \textit{waiting} queue,
it means that no packet has been received during this transmission. When this is the case, the action is
removed, and \KwNull is sent to the source of the \textit{listen} action. Second, the \L{4}{14} action has
already been processed, and removed from the \textit{waiting} queue, even though the action could not have
been processed before the \T{3}{16} action had been submitted. The \L{4}{14} action was removed from the
\textit{waiting} queue early, as the node had received a packet when processing the \T{5}{12} action.
\medbreak

This holds in general, as no \textit{transmit} action can be fully within the time interval of any
\textit{listen} action, if the \textit{listen} action is at the head of the \textit{waiting} queue. The
\textit{listen} action would have been removed when processing earlier \textit{transmit} actions, as
\textit{transmit} actions are ordered before \textit{listen} actions in the \textit{waiting} queue. Let us
assume that a node $n$ submitted the \textit{listen} action $l$, and that $l$ is at head of the
\textit{waiting} queue. When processing the $l$ action, no actions from node $n$ before, or after, $l$ is
present in the \textit{waiting} queue, as all previous actions have been processed. Additionally, no
\textit{transmit} actions from any other node $m$ can be present on the \textit{waiting} queue, with end time
less than, or equal to, the end time of the $l$ action, as these would already have been processed, due to the
ordering of the priority queue.

\begin{figure}[H]
    \caption{Processing a \textit{listen} action.}\label{tikz:coordinatormsc4}
    % Diagram
    \CoordinatorFigure{\LineLegend\Processed{(sep0-13.center) -- (sep1-13.center) -- (sep1-10.center) --
            (sep2-10.center) -- (sep2-14.center) -- (sep3-14.center) -- (sep3-15.center) -- (sep4-15.center)
            -- (sep4-16.center) -- (sep5-16.center)}\Submitted{(sep0-18.center) -- (sep1-18.center) --
            (sep1-14.center) -- (sep2-14.center) -- (sep2-16.center) -- (sep3-16.center) -- (sep3-20.center)
            -- (sep4-20.center) -- (sep4-18.center) -- (sep5-18.center)}} \par

    \begin{minipage}[h]{14.5cm}
        The content of the \textit{waiting} queue and \textit{transmits} set as of this cut is: \smallbreak

        $\mathit{waiting} \leftarrow \Big\langle \L{2}{14}, \S{1}{15}, \T{3}{16}, \T{4}{17}, \T{5}{18},
            \L{1}{18}, \L{4}{20} \Big\rangle$

        $\mathit{transmits} \leftarrow \Big\{ \T{1}{10}, \T{2}{10}, \T{5}{12}, \T{3}{16}, \T{4}{17}, \T{5}{18}
            \Big\}$
    \end{minipage}
\end{figure}

\subsubsection{Cut 5: \textit{sleep} action}
The final cut is a snapshot of the Coordinator directly after the \T{2}{18} action has been submitted to the
Coordinator. The \T{2}{18} action was submitted by \Node{2} after the \L{2}{14} action had been processed
without the node receiving any data, and the \textit{transmit} action enabled the Coordinator to continue
processing actions from the \textit{waiting} queue. The head of the \textit{waiting} queue is the
\textit{sleep} action \S{1}{15}, which, similarly to an \textit{inform} action, is trivial to process, as no
processing is needed for the Coordinator. Again, similarly to an \textit{inform} action, this holds in general
for any \textit{sleep} action. \smallbreak

After removing the \L{2}{14} action from the \textit{waiting} queue, the Coordinator were able to remove the
\T{1}{10}, \T{2}{10}, and \T{5}{12} actions from the \textit{transmits} set, as the new earliest start time is
now the start time of the \S{1}{15} action.

\begin{figure}[H]
    \caption{Processing a \textit{sleep} action.}\label{tikz:coordinatormsc5}
    % Diagram
    \CoordinatorFigure{\LineLegend\Processed{(sep0-13.center) -- (sep1-13.center) -- (sep1-14.center) --
            (sep2-14.center) -- (sep2-14.center) -- (sep3-14.center) -- (sep3-15.center) -- (sep4-15.center)
            -- (sep4-16.center) -- (sep5-16.center)} \Submitted{(sep0-18.center) -- (sep1-18.center) --
            (sep1-18.center) -- (sep2-18.center) -- (sep2-16.center) -- (sep3-16.center) -- (sep3-20.center)
            -- (sep4-20.center) -- (sep4-18.center) -- (sep5-18.center)}} \par

    \begin{minipage}[h]{14.5cm}
        The content of the \textit{waiting} queue and \textit{transmits} set as of this cut is: \smallbreak

        $waiting \leftarrow \Big\langle \S{1}{15}, \T{3}{16}, \T{4}{17}, \T{5}{18}, \T{2}{18}, \L{1}{18},
            \L{4}{20} \Big\rangle$

        %\T{1}{10}, \T{2}{10}, \T{5}{12}, (removed)
        $transmits \leftarrow \Big\{ \T{3}{16}, \T{4}{17}, \T{2}{18}, \T{5}{18} \Big\}$
    \end{minipage}
\end{figure}

\subsection{Summary}\label{sec:summary}
\todo[inline]{finish with a conclusion, summarise that we have analysed all the cases and generalised the scenarios.}