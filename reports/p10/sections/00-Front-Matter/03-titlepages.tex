\cleardoublepage
{\selectlanguage{english}
    \pdfbookmark[0]{Title Page}{label:titlepage_en}
    \aautitlepage{%
        \englishprojectinfo{
            Message Passing Interface for Massive\\Simulations of Mobile Ad-hoc Networks %title
        }{%
            Distributed Systems %theme
        }{%
            Spring Semester 2019 %project period
        }{%
            ds101f19 % project group
        }{%
            %list of group members
            Charlie Dittfeld Byrdam\smallbreak
            Jonas Kloster Jacobsen
        }{%
            %list of supervisors
            Jiri Srba\smallbreak
            Peter Gjøl Jensen
        }{%
            3 % number of printed copies
        }{%
            \today % date of completion
        }%
    }{%department and address
        \textbf{Software Engineering}\smallbreak
        Aalborg University\smallbreak
        \href{Department of Computer Science}{http://www.cs.aau.dk/}
    }{% the abstract 
        %\todo[inline]{bibliographical remark: no changes yet}
        A \gls{manet} is a decentralised wireless network where nodes communicate directly with each other
        using radio and require \gls{mac} protocols to provide energy efficient communication. This aim of
        this project is to simulate the \gls{mac} protocols and to provide an alternative to real-life
        testing. The goal is to be able to perform repeatable experiments in a controlled topology
        environment. We propose a C++ library for writing and running, simulations of \gls{manet}s, using
        \acrshort{mpi}. With this library, it is possible to write C++ implementations of communication and
        \gls{mac} protocols, for \gls{manet}s, such as ALOHA or LMAC, and perform repeatable experiments,
        where our library emulate the physical radio hardware and simulate radio communication between the
        emulated hardware. We propose a method for modelling link \gls{pathloss} using building footprints
        between nodes, on OpenStreetMap map tiles, modelling using real-life field measurements. Our
        experiments show that we can simulate 100 nodes in about 45 minutes, while using 128 cores, and that
        simulation time scales significantly with an increasing number of nodes.
        
        %that requires no pre-existing
        %infrastructure to function. Nodes communicate in the network using radio, and require networking protocols to
        %provide energy efficient communication. 
        
        %The aim of this project is to simulate the protocols, and to provide
        %an alternative to real-life testing, as scaling a real-life test requires a significant amount of effort and
        %investment of both money and time. The goal is to be able to perform repeatable experiments in a controlled
        %topology environment with up to 1000 devices.

        %We propose a C++ library for writing, and running, simulations of mobile ad-hoc networking protocols, using
        %MPI. With this library, it is possible to write C++ implementations of communication protocols for mobile
        %ad-hoc networks, such as ALOHA or LMAC, and perform repeatable experiments on these using our library, where
        %we emulate the physical hardware, and simulate the radio communication between the emulated hardware with
        %state-of-the-art modelling of link path loss, packet error and collisions.
%
        %We introduce the notion of virtual time, where we are able to skip long periods of inactivity when simulating
        %wireless networking protocols.
        % http://users.ece.cmu.edu/~koopman/essays/abstract.html
    }}

\label{page:titlepage}
\clearpage

\vspace{\baselineskip}\hfill Aalborg University, June, 2019
\vfill\noindent
\begin{minipage}[b]{0.45\textwidth}
    \centering
    \rule{\textwidth}{0.5pt}\smallbreak
    Charlie Dittfeld Byrdam\smallbreak
    {\footnotesize <cbyrda14@student.aau.dk>}
\end{minipage}
\hfill
\vspace{3\baselineskip}
\begin{minipage}[b]{0.45\textwidth}
    \centering
    \rule{\textwidth}{0.5pt}\smallbreak
    Jonas Kloster Jacobsen\smallbreak
    {\footnotesize <jkja14@student.aau.dk>}
\end{minipage}