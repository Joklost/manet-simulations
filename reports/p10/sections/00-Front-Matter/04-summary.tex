\chapter*{Summary}\label{ch:summary}
\pdfbookmark[0]{Summary}{label:summary}

A \acrfull{manet} is a decentralised wireless network where nodes communicate directly with each other using
radio, and require \gls{mac} protocols to provide energy efficient communication. The aim of this project is
to simulate the \gls{mac} protocols, and to provide an alternative to real-life testing. The goal is to be
able to perform repeatable experiments in a controlled topology environment. \medbreak

Our project proposes a \gls{mpi} C++ library for writing, and running, simulations of the network protocol
behind the mesh communication in a \gls{manet}, modelling link \gls{pathloss} to simulate packet loss and
collisions caused by interfering transmitters, where the physical devices, and the communication between
these, are emulated entirely using software. With our library, it is be possible to write a C++ implementation
of communication protocols, such as \gls{lmac}~\cite{paper:lmac_protocol} or Slotted
ALOHA~\cite{Roberts:1975:APS:1024916.1024920}, using a simple interface header file resembling a traditional
hardware interface, and perform simulations with these, where each physical device is emulated by different
CPUs on the MCC compute cluster at AAU~\cite{website:mccaau}. \medbreak

The primary contribution of this thesis is the Coordinator that facilitate wireless communication between the
emulated physical devices. The Coordinator allow us to simulate wireless communication in virtual time, where
the Coordinator is able to skip periods of inactivity, reducing the time required to run real-time
simulations. \medbreak

We present arguments of correctness for the Coordinator and the hardware functions, analysing each of the
different possible cases for when the Coordinator processes an action, starting with a concrete example, and
finishing with a generalisation of each of the scenarios. \medbreak

Additionally, we propose a method for modelling and computing link \gls{pathloss} by using building footprints
between wireless radio transmitters and receivers, with map tiles from OpenStreetMap obtained with the Mapbox
Maps Service API, and show how our method compares with another link modelling method, that uses angles
between two wireless links, that share a common transmitter or receiver, to model the correlation between
them. \medbreak

We have implemented the \gls{lmac}~\cite{paper:lmac_protocol} protocol using our C++ library, to show how our
C++ library can be used to simulate the protocol, and we present the results of our scalability experiments
for the Coordinator and the interface between the protocol and the Coordinator. We do, however, face
significant scalability problems, as we rely on a single centralised Coordinator. Our experiments show that we
can simulate 100 nodes in about 45 minutes, while using 128 cores, and that simulation time scales
significantly with an increasing number of nodes. \medbreak

Finally, we propose extensions to a network topology visualisation tool created by Peter Gjøl Jensen, in which
we are now able to visualise communication and protocol logs, that are generated by running simulations using
the Coordinator and our library.