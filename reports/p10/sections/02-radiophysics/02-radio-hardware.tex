\section{Radio Hardware}\label{sec:hardwarephysics}
In this section, we introduce a series of terms, such as \gls{rssi}, transmission power, \acrshort{dbm}, and
\gls{pathloss}, that we utilise throughout the thesis.

\begin{itemize}
    \item \acrshort{dbm} is a logarithmic scale, measuring power of a wireless radio
          signal~\cite{website:rssi-metageek}.
    \item \gls{rssi} is the perceived signal strength of a link, in
          \acrshort{dbm}~\cite{website:rssi-metageek}.
    \item \gls{pathloss} is the signal loss inflicted by the propagation of a radio signal from transmitter to
          receiver~\cite[p.~10]{paper:linkmodel}.
    \item Transmission power is the actual amount of power a transmitter uses to transmit packets, in
          \acrshort{dbm}. \gls{rssi} can be computed by subtracting \gls{pathloss} from the transmission
          power~\cite{paper:linkmodel}. For the Reachi devices, the transmission power is 26
          \acrshort{dbm}~\cite{paper:linkmodel}.
    \item The \gls{baudrate} is the rate at which information can be transferred as a wireless
          signal~\cite{website:baudrate-mathworks}, and is equivalent to bits per second. For the Reachi
          devices, the \gls{baudrate} is 34800 Hz~\cite{paper:linkmodel}.
\end{itemize}

\autoref{eq:transmission-time} computes the amount of time required to transmit a packet, in microseconds,
based on the \gls{baudrate} $f_s$ and the size of a packet in bytes. For a packet where $|\mathit{packet}| =
20$ bytes would take $\mathit{transmissiontime}(|\mathit{packet}|) = 4597$ microseconds, with \gls{baudrate}
$f_s = 34800$ Hz.
%
\begin{eq}\label{eq:transmission-time}
    \mathit{transmissiontime}(\mathit{packetsize}) = \frac{1000000}{f_s} \cdot \left( \mathit{packetsize} \cdot 8
    \right)
\end{eq}
