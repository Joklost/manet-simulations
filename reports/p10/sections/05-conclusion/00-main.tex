\chapter{Conclusion}\label{ch:conclusion}
In this thesis, we introduce a C++ library for writing and running, simulations of the \gls{mac} protocols
behind the mesh communication in a \gls{manet} using an \gls{mpi}. Our library consists of X parts: A hardware
interface header file, used for writing implementations of \gls{mac} protocols, that emulate the physical part
of a device in a \gls{manet}, a Coordinator, facilitating and coordinating the communication between the
emulated devices, and finally the link model, where we can annotate network topology \gls{gps} logs with links
between nodes, based on a model for link \gls{pathloss}, where we approximate the \gls{pathloss} for a link
using building footprints between the nodes of a link, on OpenStreetMap map tiles. With the annotated
\gls{gps} logs, we can simulate wireless radio communication, through the Coordinator, where we can
simulate packet errors and collisions caused by interfering transmissions or bad links. \smallbreak

% Conclude on building method vs angle method.
In \autoref{sec:reachi-experiments} and \autoref{sec:linkmodel}, we propose a link modelling method for
approximating the \gls{pathloss} on a link by computing the percentage amount of building between the two
nodes of a link, and show that the computed \gls{rssi} values are roughly equivalent to field measurements,
that is more reliable than computing the \gls{pathloss} entirely based on the distance. \smallbreak

% Arguments for correctness
\autoref{sec:correctness} presents our arguments of correctness for the Coordinator and the hardware
functions, analysing the different possible cases for when the Coordinator processes an action, starting with 
a concrete example, and finishing with a generalisation of each of the scenarios. \smallbreak

In \autoref{ch:experiments}, we present the \gls{lmac} protocol and show how our C++ library can be used to
simulate the protocol, and we present the results of our scalability experiments for the Coordinator. We do,
however, face significant scalability problems, as we rely on a single centralised Coordinator, as shown in
\autoref{sec:scalability}. Our experiments show that we can simulate 100 nodes in about 45 minutes, while
using 128 cores, and that simulation time scales significantly with an increasing number of nodes.

% We added extensions to the Visualiser tool, to visualise communication and protocol logs, in order to replay
% a simulation.

\section{Future Work}
The 
\todo[inline]{scalability improvements (coordinator optimisation)}
\todo[inline]{distributed coordinator}
% Scalability improvements / optimisation for the Coordinator
% Distributed "Coordinator"