\section{LMAC}\label{sec:lmacc}
\gls{lmac}~\cite{paper:lmac_protocol}\cite{paper:lmac_verification} is a \gls{tdma} protocol designed to be
lightweight and energy efficient, in order to prolong the lifetime of a network. In the protocol, time is
divided in to frames, that consist of a fixed number of time slots. Each node in a network controls a single
time slot in each frame, and if a node has any data to transmit, the node waits for its time slot to come up,
which means that a node is able to transmit the data without causing collision, or interference for other
nodes. Additionally, whenever a nodes time slot comes up, the node transmits a short synchronisation message
in the beginning of the time slot. For every other time slot, the node listens for the synchronisation message
from other nodes, to maintain synchronisation and keep neighbourhood information up-to-date. The structure of
a synchronisation can be found in \autoref{fig:bytefield:lmac-control-packet}. \medbreak

The \gls{lmac} protocol consists of four phases:

\begin{description}[style=nextline]
    \item[Initialisation] Each node initially starts in the Initialisation phase. In this phase, the node has
          yet to choose a time slot, so instead it listens for synchronisation messages in every time slot.
          When a synchronisation message has been received and a neighbouring node has been detected, the node
          synchronises, and the node knows the current slot number. After a synchronisation message has been
          received, the node switches to the Wait phase at the beginning of the next frame. A single node
          chosen as the gateway node starts the Initialisation phase by picking a time slot and proceeding to
          the active phase.
    \item[Wait] Node in the Wait phase wait a random amount of frames (up to a pre-defined limit) between
          receiving the synchronisation message, and moving to the Discover phase.
    \item[Discover] In the Discover phase, the node collects first order neighbourhood information from
          neighbouring active nodes, by listening for synchronisation messages throughout one frame, and
          recording the occupied time slots. Once a frame worth of neighbourhood information has been
          recorded, the node chooses a random, available, time slot, and proceeds to the Active phase.
    \item[Active] Finally, a node in the Active phase is able to transmit a data message in its chosen time
          slot, while listening in other time slots to accept data from neighbouring nodes. The node still
          uses the synchronisation message to keep neighbourhood information up-to-date, and attempts to
          detect, and report, possible collisions in the network. When a node in the Active phase is informed
          of a collision in its chosen time slot, the node will give up its time slot, and proceed to the Wait
          phase. A collision happens when two or more nodes have chosen the same time slot. Nodes that are
          part of a collision is unable to detect the collision, and they need to be informed by their
          neighbours. When a node has detected a collision, the time slot in question is included in the
          synchronisation message that will be sent from the node in its next time slot, to inform all
          neighbours of the collision.
\end{description}

\begin{figure}[ht]
    \centering

    \begin{bytefield}[bitwidth=\textwidth / 96, bitheight=2cm]{96}
        \bitheader{0, 8, 16, 24, 32, 40, 48, 56, 64, 72, 80, 88, 95}\\
        \bitbox{16}{ID}
        \bitbox{8}{Slot}
        \bitbox{32}{Occupied Slots}
        \bitbox{8}{DtG}
        \bitbox{8}{Colli- sion}
        \bitbox{16}{Destination ID}
        \bitbox{8}{Data Size)}
    \end{bytefield}

    \caption{The synchronisation message structure in \gls{lmac}~\cite[p.~2]{paper:lmac_protocol}.}
    \label{fig:bytefield:lmac-control-packet}
\end{figure}

In \gls{lmac}, each node keeps track of its \textit{hop-distance} to the pre-defined gateway
node~\cite{paper:lmac_protocol} and includes this hop-distance in the synchronisation message as the
\doublequote{DtG}, or distance to gateway, field. When an Active node has a data packet to transmit, the node
looks through its neighbourhood information to find a neighbouring node that is closest to the gateway,
pick this node as the destination for its message, and include the destination in the synchronisation message.
Should multiple neighbours nodes be equally close to the gateway, a destination will randomly be picked
between them. If a destination, and a data size, is included in the synchronisation message, and the
destination is equal to the nodes id, the node will listen for a data message, in the time slot, after having
received the synchronisation message. Additionally, it is only possible for a node to transmit a single data
message per frame, and the maximum size of a data message is 256 bytes.

%https://youtu.be/p_hQJd0pMXk

%https://youtu.be/EbsL2zhTlgc

\begin{figure}[ht]
    \centering
    \begin{subfigure}[b]{0.48\textwidth}
        \centering
        \qrcode[hyperlink]{https://youtu.be/EbsL2zhTlgc}
        \caption{Static grid topology synchronisation.}
        \label{fig:lmac-static-topology-synchronisation-qr}
    \end{subfigure}
    \hfill
    \begin{subfigure}[b]{0.48\textwidth}
        \centering
        \qrcode[hyperlink]{https://youtu.be/p_hQJd0pMXk}
        \caption{Static grid topology routing.}
        \label{fig:lmac-static-topology-routing-qr}
    \end{subfigure}

    \caption{\gls{lmac} static grid topology synchronisation and routing.}
    \label{fig:lmac-visualisation}
\end{figure}

\autoref{fig:lmac-visualisation} contains YouTube links to two visualisations of the \gls{lmac} protocol,
where \autoref{fig:lmac-static-topology-synchronisation-qr} visualises node synchronisation and network
stabilisation, and \autoref{fig:lmac-static-topology-routing-qr} visualises the routing from a data generation
node to the gateway node. Nodes are coloured per their phase, with white nodes being in the Initialisation
phase, red nodes in the Wait phase, blue nodes in the Discover phase, and green nodes in the Active phase.
When the node enters the Active phase, the chosen slot is drawn on the node. \medbreak

In the routing visualisation in \autoref{fig:lmac-static-topology-routing-qr}, the bottom left node is chosen
as the gateway node, and the top right node generates a single data message each frame. An arrow originates
from a node whenever a message is sent from the node, with a green arrow denoting the synchronisation message,
and the red arrow denoting a data message. \medbreak

% \autoref{code:sendmessage} shows a snippet of the \gls{lmac} implementation for the part of the code where the
% protocol constructs and broadcast the synchronisation message, and a data message, if any has been received.

% \begin{cpp}{Construct and send synchronisation and data messages.}{code:sendmessage}
% ...
% /* Create synchronisation signal. */
% ControlPacket ctrl{id, state.chosen_slot,
%                    state.occupied_slots, state.gateway_distance,
%                    state.collision_slot, receiver, data_size};
% state.collision_slot = NO_SLOT;

% /* Send initial synchronisation signal. */
% hardware::sleep(3ms);
% hardware::broadcast(mpilib::serialise(ctrl));

% /* Send packet, if any. */
% if (!state.data_packet.empty()) {
%     hardware::sleep(10ms);
%     hardware::broadcast(state.data_packet);
%     state.data_packet.clear();
% }
% ...
% \end{cpp}

% \autoref{code:recvmessage} shows another snippet of the \gls{lmac} protocol implementation

% \begin{cpp}{Receive messages}{code:recvmessage}
% ...
% /* Listen for synchronisation signal. */
% auto ctrl_data = hardware::receive(20ms);
% if (!ctrl_data.empty()) {
%     auto ctrl = mpilib::deserialise<ControlPacket>(ctrl_data);
%     state.occupied_slots[ctrl.chosen_slot] = true;

%     if (ctrl.gateway_distance + 1 < state.gateway_distance) {
%         state.gateway_distance = ctrl.gateway_distance + 1;
%     }

%     ...

%     if (state.phase == active) {
%         if (ctrl.destination_id == id && ctrl.data_size > 0) {
%             /* Listen for packet. */
%             auto data = hardware::receive(70ms);
%             if (!data.empty()) {
%                 state.data_packet = data;

%                 ...
%             } 
%         }
%     }
% }
% ...    
% \end{cpp}

The complete source code for the C++ implementation can be found on GitHub:

{\small \url{https://github.com/Joklost/manet-simulations/tree/master/src/lmac}}


%\todo[inline]{include some code from the C++ implementation