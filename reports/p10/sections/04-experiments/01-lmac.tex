\section{LMAC}\label{sec:lmacc}
\gls{lmac}~\cite{paper:lmac_protocol}\cite{paper:lmac_verification} is a \gls{tdma} protocol designed to be
lightweight and energy efficient, to prolong the lifetime of a network. In the protocol, time is
divided into frames, that consist of a fixed number of time slots. Each node in a network controls a single
time slot in each frame, and if a node has any data to transmit, the node waits for its time slot to come up,
which means that a node can transmit the data without causing a collision, or interference for other
nodes. Additionally, whenever a nodes time slot comes up, the node transmits a short synchronisation packet
at the beginning of the time slot. For every other time slot, the node listens for the synchronisation packet
from other nodes, to maintain synchronisation and keep neighbourhood information up-to-date. The structure of
a synchronisation can be found in \autoref{fig:bytefield:lmac-control-packet}. \medbreak

The \gls{lmac} protocol consists of four phases:

\begin{description}[style=nextline]
    \item[Initialisation] Each node initially starts in the Initialisation phase. In this phase, the node has
          yet to choose a time slot, so instead, it listens for synchronisation packets in every time slot.
          When a synchronisation packet has been received and a neighbouring node has been detected, the node
          synchronises, and the node knows the current slot number. After a synchronisation packet has been
          received, the node switches to the Wait phase at the beginning of the next frame. A single node
          chosen as the gateway node starts the Initialisation phase by picking a time slot and proceeding to
          the active phase.
    \item[Wait] A node in the Wait phase waits for a random amount of frames (up to a pre-defined limit) between
          receiving the synchronisation packet and moving to the Discover phase.
    \item[Discover] In the Discover phase, the node collects first order neighbourhood information from
          neighbouring active nodes, by listening for synchronisation packets throughout one frame, and
          recording the occupied time slots. Once a frame worth of neighbourhood information has been
          recorded, the node chooses a random, available, time slot, and proceeds to the Active phase.
    \item[Active] Finally, a node in the Active phase can transmit a data packet in its chosen time
          slot, while listening in other time slots to accept data from neighbouring nodes. The node still
          uses the synchronisation packet to keep neighbourhood information up-to-date, and attempts to
          detect, and report, possible collisions in the network. When a node in the Active phase is informed
          of a collision in its chosen time slot, the node will give up its time slot, and proceed to the Wait
          phase. A collision happens when two or more nodes have chosen the same time slot. Nodes that are
          part of a collision is unable to detect the collision, and they need to be informed by their
          neighbours. When a node has detected a collision, the time slot in question is included in the
          synchronisation packet that will be sent from the node in its next time slot, to inform all
          neighbours of the collision.
\end{description}

\begin{figure}[ht]
    \centering

    \begin{bytefield}[bitwidth=\textwidth / 96, bitheight=2cm]{96}
        \bitheader{0, 8, 16, 24, 32, 40, 48, 56, 64, 72, 80, 88, 95}\\
        \bitbox{16}{ID}
        \bitbox{8}{Slot}
        \bitbox{32}{Occupied Slots}
        \bitbox{8}{DtG}
        \bitbox{8}{Colli- sion}
        \bitbox{16}{Destination ID}
        \bitbox{8}{Data Size)}
    \end{bytefield}

    \caption{The synchronisation packet structure in \gls{lmac}~\cite[p.~2]{paper:lmac_protocol}.}
    \label{fig:bytefield:lmac-control-packet}
\end{figure}

In \gls{lmac}, each node keeps track of its \textit{hop-distance} to the pre-defined gateway
node~\cite{paper:lmac_protocol} and includes this hop-distance in the synchronisation packet as the
\doublequote{DtG}, or distance to gateway, field. When an Active node has a data packet to transmit, the node
looks through its neighbourhood information to find a neighbouring node that is closest to the gateway,
pick this node as the destination for its message, and include the destination in the synchronisation packet.
Should multiple neighbours nodes be equally close to the gateway, a destination will randomly be picked
between them. If a destination, and a data size, is included in the synchronisation packet, and the
destination is equal the id of the node, the node will listen for a data packet, in the time slot, after having
received the synchronisation packet. Additionally, it is only possible for a node to transmit a single data
message per frame and the maximum size of a data packet is 256 bytes.

%https://youtu.be/p_hQJd0pMXk

%https://youtu.be/EbsL2zhTlgc

\begin{figure}[ht]
    \centering
    \begin{subfigure}[b]{0.32\textwidth}
        \centering
        \qrcode[hyperlink]{https://youtu.be/EbsL2zhTlgc}
        \caption{Grid topology synchronisation.}
        \label{fig:lmac-static-topology-synchronisation-qr}
    \end{subfigure}
    \hfill
    \begin{subfigure}[b]{0.32\textwidth}
        \centering
        \qrcode[hyperlink]{https://youtu.be/p_hQJd0pMXk}
        \caption{Grid topology routing.}
        \label{fig:lmac-static-topology-routing-qr}
    \end{subfigure}
    \hfill
    \begin{subfigure}[b]{0.32\textwidth}
        \centering
        \qrcode[hyperlink]{https://www.youtube.com/watch?v=-GQ5qWEalQ8}
        \caption{Dynamic topology synchronisation.}
        \label{fig:lmac-dynamic-topology-synchronisation-qr}
    \end{subfigure}
    \caption{\gls{lmac} topology synchronisation and routing.}
    \label{fig:lmac-visualisation}
\end{figure}

\autoref{fig:lmac-visualisation} contains YouTube links to three visualisations of the \gls{lmac} protocol,
where \autoref{fig:lmac-static-topology-synchronisation-qr} visualises node synchronisation and network
stabilisation, \autoref{fig:lmac-static-topology-routing-qr} visualises the routing from a data generation
node to the gateway node, and \autoref{fig:lmac-dynamic-topology-synchronisation-qr} visualises part of the
node synchronisation of a larger network. Nodes are coloured per their phase, with white nodes being in the
Initialisation phase, red nodes in the Wait phase, blue nodes in the Discover phase, and green nodes in the
Active phase. When the node enters the Active phase, the chosen slot is drawn on the node. \medbreak

In the routing visualisation in \autoref{fig:lmac-static-topology-routing-qr}, the bottom left node is chosen
as the gateway node, and the top right node generates a single data packet each frame. An arrow originates
from a node whenever a message is sent from the node, with a green arrow denoting the synchronisation packet,
and the red arrow denoting a data packet. \medbreak

% https://www.youtube.com/watch?v=-GQ5qWEalQ8

\autoref{code:sendmessage} shows a snippet of the \gls{lmac} implementation for the part of the code where a
node in the Active phase constructs and broadcast the synchronisation packet, and a data packet, if any has
been received, in its chosen time slot. If the node has a data packet to send, it computes the receiver with
the lowest hop-distance to the gateway, and adds it to the synchronisation packet, along with the size of the 
data packet.

\begin{cpp}{Construct and send synchronisation and data packets.}{code:sendmessage}
...
/* Create synchronisation signal. */
ControlPacket ctrl{id, state.chosen_slot,
                   state.occupied_slots, state.gateway_distance,
                   state.collision_slot, receiver, data_size};
state.collision_slot = NO_SLOT;
/* Send initial synchronisation signal. */
hardware::sleep(3ms);
hardware::broadcast(mpilib::serialise(ctrl));
/* Send packet, if any. */
if (!state.data_packet.empty()) {
    hardware::sleep(10ms);
    hardware::broadcast(state.data_packet);
    state.data_packet.clear();
}
...
\end{cpp}

\autoref{code:recvmessage} shows another snippet of the \gls{lmac} protocol implementation, where a node in
either the Initialisation, Discover, or Active phase is listening for synchronisation packets. When a node
receives the synchronisation packet, the node updates its local state, and the neighbourhood information, with
any new information, received in the synchronisation packet. If a packet as destined for the node, and the node
is in the Active phase, the node listens for the data packet, and stores it for later transmission.
\smallbreak

The complete source code for the C++ implementation can be found on GitHub:

{\small \url{https://github.com/Joklost/manet-simulations/tree/master/src/lmac}}

\begin{cpp}{Receive synchronisation and data packets.}{code:recvmessage}
...
/* Listen for synchronisation signal. */
auto ctrl_data = hardware::receive(20ms);
if (!ctrl_data.empty()) {
    auto ctrl = mpilib::deserialise<ControlPacket>(ctrl_data);
    state.occupied_slots[ctrl.chosen_slot] = true;
    if (ctrl.gateway_distance + 1 < state.gateway_distance) {
        state.gateway_distance = ctrl.gateway_distance + 1;
    }
    /* Update neighbourhood information. */
    ...
    if (state.phase == active) {
        if (ctrl.destination_id == id && ctrl.data_size > 0) {
            /* Listen for packet. */
            auto data = hardware::receive(70ms);
            if (!data.empty()) {
                state.data_packet = data;
                ...
            } 
        }
    }
}
...    
\end{cpp}
