\section{Scalability}\label{sec:scalability}
We conducted two scalability experiments with the \gls{lmac} protocol from \autoref{sec:lmacc}. The first
experiment is to show how simulations with the centralised Coordinator performs, with a fixed number of nodes,
when scaling the amount of CPUs available for the simulation. In the second experiment, we attempt to scale
the number of nodes, while also scaling the number of CPUs to match the size of the network.

\begin{table}[H]
    \begin{tabular}{|l|c|c|c|}
        \hline
        Cores & 16        & 32        & 64        \\ \hline
        Time  & 41:49 min & 34:52 min & 19:28 min \\ \hline
    \end{tabular}
    \caption{5-minute random walk topology, 60 nodes.}\label{table:lmac-experiment1}
\end{table}

In the first experiment, we simulate the \gls{lmac} protocol with 60 nodes in a generated random walk
topology, over a 5-minute real-time period with approximately 300 links at all times, throughout the log. The
results for this experiment can be seen in \autoref{table:lmac-experiment1}. The experiment demonstrate that
scaling the amount of CPUs will significantly improve the time required to run the simulation, but it also
show that the simulation does not scale very well with a larger number of nodes, as even when running the
simulation with 64 cores available, simulating the experiment takes way longer than the 5-minute \gls{gps} log
we simulate.
% 5-minute simulation, 60 nodes, random walk topology
% 16 cores : 41:49
% 32 cores : 34:52
% 64 cores : 19:28

\begin{table}[H]
    \begin{tabular}{|l|c|c|c|}
        \hline
        Nodes & 100          & 300           & 600           \\ \hline
        Links & $\approx$450 & $\approx$1200 & $\approx$2400 \\ \hline
        Cores & 128          & 320           & 576           \\ \hline
        Time  & 45:43 min    & 6:51:53 hrs   & \dots         \\ \hline
    \end{tabular}
    \caption{5-minute random walk topology.}\label{table:lmac-experiment2}
\end{table}

The second experiment further shows the scalability issues of the Coordinator. In this experiment, we
attempted to simulate the \gls{lmac} protocol while scaling the number of nodes in the network, as well as the
amount of CPUs available. The results for this experiment can be seen in \autoref{table:lmac-experiment2}. A
5-minute real-time simulation with 100 nodes, and approximately 450 links, took over 45 minutes with 128
cores, and a 300 node experiment, with approximately 1200 links, took almost seven hours using 320 core. After
this result, we decided not to continue the experiment. \medbreak

% Where do we think the bottleneck is?

The annotated random-walk \gls{gps} logs can be found on GitHub:

{\footnotesize \url{https://github.com/Joklost/manet-simulations/tree/master/src/coordinator/logs}}

% 5-minute simulation, random walk topology
% 100 nodes : 45:43
% 300 nodes : 6:51:53
% 600 nodes :
% 1000 nodes : 
