
\paragraph{Hardware Functions}\label{sec:hwfuncspseudo} \

The hardware functions listed in this section depend on the following local state. The local state is unique for each node.\smallbreak

clock $\leftarrow$ \KwNow

localtime $\leftarrow$ 0

id $\leftarrow$ unique identifier \smallbreak

In the hardware functions we utilise a special keyword \KwNow, which represents the real-time hardware clocks of a node. It is assumed that all clocks run at the same speed. The clock variable is used to measure the real-time spent by the node between calls to hardware functions. Initially, we store the current time in the clock variable, and use the clock to compute the real-time difference between the calling of hardware functions and add the difference to the localtime variable (e.g., localtime $\leftarrow$ (\KwNow $-$ clock) $+$ localtime).

The unique identifier (id) of a node is meant to function as the address of a node for passing messages between nodes and the Coordinator. The identifier of the Coordinator = $0$, and the identifier of the nodes are in the range $\{ 1, 2, 3, \ldots, N \}$ for $N$ nodes.

%Common for all four hardware functions is that we initially update the localtime variable, and set the clock variable to \KwNow before returning from the function, such that the execution time spent communicating with the Coordinator is not included in the duration of the action.

\begin{algorithm}[ht]
    \DontPrintSemicolon
    \SetKwFunction{FBroadcast}{Broadcast}
    \SetKwProg{Fn}{Function}{}{}
    
    \Fn{\FBroadcast{packet}}{
        localtime $\leftarrow$ (\KwNow $-$ clock) $+$ localtime\;
        duration $\leftarrow$ transmission-time(|packet|)\;
        end $\leftarrow$ localtime $+$ duration\;
        a $\leftarrow$ (transmit, id, localtime, end, packet)\;
        \KwSend a \KwTo Coordinator\;
        localtime $\leftarrow$ end\;
        clock $\leftarrow$ \KwNow\;
    }

    \caption{The \texttt{Broadcast} Function.}
    \label{algo:hwfuncstransmit}
\end{algorithm}

The \texttt{Broadcast} (\autoref{algo:hwfuncstransmit}) function broadcasts a data packet. The packet is sent to the Coordinator using the \gls{mpi}, and the Coordinator takes care of distributing the packet to neighbouring nodes listening for packets. The duration of a transmission is computed based on the \gls{baudrate} (the amount of bits the hardware can transmit per second~\cite{website:baudrate-mathworks}) as well as the size of the packet. For our hardware, we assume a \gls{baudrate} $f_s = 34800$ Hz.

\begin{eq}\label{eq:transmission-time}
    \text{transmission-time}(\text{packetsize}) = \frac{1000000}{f_s} \cdot \left( \text{packetsize} \cdot 8 \right)
\end{eq}

\autoref{eq:transmission-time} computes the amount of time required to transmit a packet, in microseconds, based on the \gls{baudrate} $f_s$ and the size of a packet in bytes. For a packet where $|\text{packet}| = 20$ bytes would take $\text{transmission-time}(|\text{packet}|) = 4597$ microseconds, with \gls{baudrate} $f_s = 34800$ Hz.

After computing the duration, the transmit action is sent to the Coordinator, the localtime variable is set to the end time and the clock is set to \KwNow, before the function returns. \medbreak

\begin{algorithm}[ht]
    \DontPrintSemicolon
    \KwResult{A packet or \KwNull}
    \SetKwFunction{FListen}{Listen}
    \SetKwProg{Fn}{Function}{}{}
    
    \Fn{\FListen{duration}}{
        localtime $\leftarrow$ (\KwNow $-$ clock) $+$ localtime\;
        a $\leftarrow$ (listen, id, localtime, localtime $+$ duration, \KwNull)\;
        \KwSend a \KwTo coordinator\;
        localtime $\leftarrow$ \KwAwait end \KwFrom Coordinator\; \label{algo:hwfuncslisten:awaitend}
        packet $\leftarrow$ \KwAwait packet \KwFrom Coordinator\;
        \tcp{The packet returned from Coordinator may be \KwNull}
        clock $\leftarrow$ \KwNow\;
        \KwRet packet\;
    }
    
    \caption{The \texttt{Listen} Function.}
    \label{algo:hwfuncslisten}
\end{algorithm}

The \texttt{Listen} (\autoref{algo:hwfuncslisten}) functions takes a duration as input and sends a listen action to the Coordinator. After sending its action, the function waits for a response from the Coordinator at \autoref{algo:hwfuncslisten:awaitend}. The \KwAwait keyword is blocking, which means that no other functions can be called from a node while the node is listening for a packet. When the Coordinator processes a listen action, two messages will be sent to the node. The first is the end time, which is assigned to the localtime variable, and the second is the packet received (if any). If no packet has been received, the end time received from the Coordinator will be the same as the end time in the action sent to the Coordinator (localtime $+$ duration), and the packet received will be \KwNull. If a packet has been received, the Coordinator will send the time when the packet was received, with the packet following right after.

At most a single packet may be received on a call to the \texttt{Listen} function, but depending on the number of transmissions in the same time interval, no packet could be received, as multiple transmissions either will provide interference for each other, creating collisions, or no transmissions may have happened in the time interval.

After receiving a response from the Coordinator, the function will set the clock variable to \KwNow, and return either the packet or \KwNull. \medbreak

\begin{algorithm}[ht]
    \DontPrintSemicolon
    \SetKwFunction{FSleep}{Sleep}
    \SetKwProg{Fn}{Function}{}{}
    
    \Fn{\FSleep{duration}}{
        localtime $\leftarrow$ (\KwNow $-$ clock) $+$ localtime\;
        end $\leftarrow$ localtime $+$ duration\;
        a $\leftarrow$ (sleep, id, localtime, end, \KwNull)\;
        \KwSend a \KwTo Coordinator\;
        localtime $\leftarrow$ end\;
        clock $\leftarrow$ \KwNow\;
    }
    
    \caption{The \texttt{Sleep} Function.}
    \label{algo:hwfuncssleep}
\end{algorithm}

The \texttt{Sleep} (\autoref{algo:hwfuncssleep}) takes a duration as input and sends a sleep action to the Coordinator. As no response is expected of the Coordinator, the function sets the localtime and clock variables immediately after sending the action. \medbreak

\begin{algorithm}[ht]
    \DontPrintSemicolon
    \SetKwFunction{FInformLocalTime}{InformLocalTime}
    \SetKwProg{Fn}{Function}{}{}
    
    \Fn{\FInformLocalTime{}}{
        localtime $\leftarrow$ (\KwNow $-$ clock) $+$ localtime\;
        a $\leftarrow$ (inform, id, localtime, localtime, \KwNull)\;
        \KwSend a \KwTo Coordinator\;
        clock $\leftarrow$ \KwNow\;
    }
    
    \caption{The \texttt{InformLocaltime} Function.}
    \label{algo:hwfuncsupdatelocaltime}
\end{algorithm}

The \texttt{InformLocalTime} function is equivalent to the \texttt{Sleep} function in the sense that it behaves like the \texttt{Sleep} function with the duration set to 0. The function is included in the case there none of the other hardware functions are applicable, e.g., in the case where the node is performing longer computations,. Regularly informing the Coordinator of a nodes localtime will allow the Coordinator to continually process actions from other nodes.

\clearpage

\paragraph{Coordinator} \ 

\begin{algorithm}[H]
    \DontPrintSemicolon
    \SetKwFunction{FCoordinator}{Coordinator}
    \SetKwProg{Fn}{procedure}{}{}

    Action = (type, source, start, end, packet)\;
    \;

    \Fn{\FCoordinator{}}{
        waiting $\leftarrow$ priority queue of Action objects, ordered by end, transmit before listen\;
        transmits $\leftarrow$ empty list of Action objects\;
        \;
        
        \Repeat{\textit{protocol terminates}}{
            \tcp{Part 1. Handle incoming messages.}
            a $\leftarrow$ \KwAwait Action \KwFrom any node\; \label{line:action-await}
            \KwEnqueue a \KwTo waiting\;
            \If{a.type = transmit}{ \label{algo:mpicoordinator:message-transmit}
                \KwAppend a \KwTo transmits\;
            }
            
            \;
            \tcp{Part 2. Process waiting actions.}
            \While{each node has at least one Action in waiting}{ \label{algo:mpicoordinator:whileactions}
                a $\leftarrow$ \KwDequeue Action \KwFrom waiting\; \label{algo:mpicoordinator:dequeue}

                \If{a.type = transmit}{ \label{line:action-transmit}
                    interferers $\leftarrow$ empty list of node identifiers\;
                    \ForEach{t $\in$ transmits \KwWhere t $\neq$ a}{
                        \If{a.end $\geq$ t.start \KwAnd a.start $\leq$ t.end}{ \label{algo:mpicoordinator:transmitintersects}
                            \tcp{Transmissions intersect.}
                            \KwAppend t.source \KwTo interferers\;
                        }
                    }

                    \ForEach{l $\in$ waiting \KwWhere l.type = listen}{
                        \If{a.start $\geq$ l.start \KwAnd a.end $\leq$ l.end}{
                            probability $\leftarrow$ $P_p$(l.source, a.source, interferers, |a.packet|)\;
                            should-receive $\leftarrow$ randomly choose based on probability\;
                            \If{should-receive}{ \label{algo:mpicoordinator:shouldreceive}
                                \KwSend a.end \KwTo l.source\; 
                                \KwSend a.packet \KwTo l.source\;
                                \KwRemove l \KwFrom waiting\;
                            }
                        }
                    }
                }

                \ElseIf{a.type = listen}{ \label{line:action-listen}
                    \KwSend a.end \KwTo a.source\;
                    \KwSend \KwNull \KwTo a.source\; \label{algo:mpicoordinator:sendnull}
                }
                \tcp{Sleep/Inform actions are handled implicitly.}
            }
            \;
            \tcp{Part 3. Clean transmits list.}            
            start-time $\leftarrow$ earliest start time of all actions \KwInn waiting\;
            \ForEach{t $\in$ transmits \KwWhere t.end < start-time}{ \label{algo:mpicoordinator:cleantransmits}
                \KwRemove t \KwFrom transmits\;
            }
        }
    }
    \caption{The \texttt{Coordinator} procedure.}
    \label{algo:mpicoordinator}
\end{algorithm}

\clearpage

\todo[inline]{we can never receive an earlier action from a node}

\paragraph{Part 1} \

\begin{itemize}
    \item The \KwAwait keyword is blocking.
    \item Only transmit actions need special handling at Part 1. Any action we received is added to the waiting queue, and at \autoref{algo:mpicoordinator:message-transmit} we add any transmit actions to the transmits list.
    \item After receiving a single message the Coordinator moves to Part 2, where we process all possible messages while the condition at \autoref{algo:mpicoordinator:whileactions} is satisfied.
\end{itemize}

\paragraph{Part 2} \

\begin{itemize}
    \item Sleep and inform actions are handled implicitly as we simply dequeue the actions at \autoref{algo:mpicoordinator:dequeue}. No further processing is needed, as the actions only have to be present on the waiting queue, to satisfy the condition at \autoref{algo:mpicoordinator:whileactions}.
    \item As transmit actions always are processed before listen actions, we can safely remove listen actions from the queue when processing a transmit action, and process the listen action by sending the end time and packet of the transmit action to the source node of the listen action, should the packet be received by the node at \autoref{algo:mpicoordinator:shouldreceive}.
    \item If we dequeue a listen action at \autoref{algo:mpicoordinator:dequeue}, we know that no packets were received for the action, and we can send \KwNull to the source node of the listen action at \autoref{algo:mpicoordinator:sendnull}.
    \item It is assumed that a transmit action will interfere with another transmit action if the time intervals of the two actions intersect at any time (\autoref{algo:mpicoordinator:transmitintersects}).
\end{itemize}

\paragraph{Part 3} \

\begin{itemize}
    \item Any transmit actions in the transmits can be removed when it can no longer interfere with other transmit actions. At \autoref{algo:mpicoordinator:cleantransmits} any transmit actions in the transmits list is removed if the end time is before the earliest start time of all actions in the waiting queue.
\end{itemize}

\clearpage

\begin{eq}\label{eq:noisepower}
    P_{NI,dB}(n_r, m_t, \text{nodes}_t) = 10 \log_{10}\left( 10^{\frac{P_{N,dB}}{10}} + \mathlarger{\sum}\limits_{m \in \text{nodes}_t}  10^{\frac{\text{RSSI}_{\text{dBm}}(n_{r}, m)}{10}} \right)
\end{eq}

\begin{eq}
    \gamma_{dB}(n_r, m_t, \text{nodes}_t) = \text{RSSI}_{\text{dBm}}(n_r, m_t) - P_{NI,dB}(n_r, m_t, \text{nodes}_t)
\end{eq}

\begin{eq}
    P_b(n_r, m_t, \text{nodes}_t) = \frac{1}{2}\text{erfc} \left( \sqrt{ \left( \frac{10^{\frac{\gamma_{dB}(n_r, m_t, \text{nodes}_t)}{10}}}{2} \right)} \right)
\end{eq}

\begin{eq}
    P_p(n_r, m_t, \text{nodes}_t, \text{packetsize}, \text{time}) = 1 - \left( 1 - P_b(n_r, m_t, \text{nodes}_t) \right) ^{\text{packetsize}}
\end{eq}

\clearpage

\begin{figure}
    \centering
% Diagram
\begin{tikzpicture}[every node/.style={font=\normalsize,minimum height=0.5cm,minimum width=0.75cm},]

% Matrix
\node [matrix,very thin,column sep=0.5cm,row sep=0.2cm] (matrix) at (0,0) {

\node(0,0) (t00 left) {}; &  \node(0,0) (\Node{1} 0) {}; &&  \node(0,0) (\Node{2} 0) {}; &&  \node(0,0) (\Node{3} 0) {}; &&  \node(0,0) (\Node{4} 0) {}; &&  \node(0,0) (\Node{5} 0) {}; & \node(0,0) (t00 right) {}; & \node(0,0) (legend 00) {}; \\
\node(0,0) (t01 left) {}; &  \node(0,0) (\Node{1} 1) {}; &&  \node(0,0) (\Node{2} 1) {}; &&  \node(0,0) (\Node{3} 1) {}; &&  \node(0,0) (\Node{4} 1) {}; &&  \node(0,0) (\Node{5} 1) {}; & \node(0,0) (t01 right) {}; & \node(0,0) (legend 01) {}; \\
\node(0,0) (t02 left) {}; &  \node(0,0) (\Node{1} 2) {}; &&  \node(0,0) (\Node{2} 2) {}; &&  \node(0,0) (\Node{3} 2) {}; &&  \node(0,0) (\Node{4} 2) {}; &&  \node(0,0) (\Node{5} 2) {}; & \node(0,0) (t02 right) {}; & \node(0,0) (legend 02) {}; \\
\node(0,0) (t03 left) {}; &  \node(0,0) (\Node{1} 3) {}; &&  \node(0,0) (\Node{2} 3) {}; &&  \node(0,0) (\Node{3} 3) {}; &&  \node(0,0) (\Node{4} 3) {}; &&  \node(0,0) (\Node{5} 3) {}; & \node(0,0) (t03 right) {}; & \node(0,0) (legend 03) {}; \\
\node(0,0) (t04 left) {}; &  \node(0,0) (\Node{1} 4) {}; &&  \node(0,0) (\Node{2} 4) {}; &&  \node(0,0) (\Node{3} 4) {}; &&  \node(0,0) (\Node{4} 4) {}; &&  \node(0,0) (\Node{5} 4) {}; & \node(0,0) (t04 right) {}; & \node(0,0) (legend 04) {}; \\
\node(0,0) (t05 left) {}; &  \node(0,0) (\Node{1} 5) {}; &&  \node(0,0) (\Node{2} 5) {}; &&  \node(0,0) (\Node{3} 5) {}; &&  \node(0,0) (\Node{4} 5) {}; &&  \node(0,0) (\Node{5} 5) {}; & \node(0,0) (t05 right) {}; & \node(0,0) (legend 05) {}; \\
\node(0,0) (t06 left) {}; &  \node(0,0) (\Node{1} 6) {}; &&  \node(0,0) (\Node{2} 6) {}; &&  \node(0,0) (\Node{3} 6) {}; &&  \node(0,0) (\Node{4} 6) {}; &&  \node(0,0) (\Node{5} 6) {}; & \node(0,0) (t06 right) {}; & \node(0,0) (legend 06) {}; \\
\node(0,0) (t07 left) {}; &  \node(0,0) (\Node{1} 7) {}; &&  \node(0,0) (\Node{2} 7) {}; &&  \node(0,0) (\Node{3} 7) {}; &&  \node(0,0) (\Node{4} 7) {}; &&  \node(0,0) (\Node{5} 7) {}; & \node(0,0) (t07 right) {}; & \node(0,0) (legend 07) {}; \\
\node(0,0) (t08 left) {}; &  \node(0,0) (\Node{1} 8) {}; &&  \node(0,0) (\Node{2} 8) {}; &&  \node(0,0) (\Node{3} 8) {}; &&  \node(0,0) (\Node{4} 8) {}; &&  \node(0,0) (\Node{5} 8) {}; & \node(0,0) (t08 right) {}; & \node(0,0) (legend 08) {}; \\
\node(0,0) (t09 left) {}; &  \node(0,0) (\Node{1} 9) {}; &&  \node(0,0) (\Node{2} 9) {}; &&  \node(0,0) (\Node{3} 9) {}; &&  \node(0,0) (\Node{4} 9) {}; &&  \node(0,0) (\Node{5} 9) {}; & \node(0,0) (t09 right) {}; & \node(0,0) (legend 09) {}; \\
\node(0,0) (t10 left) {}; & \node(0,0) (\Node{1} 10) {}; && \node(0,0) (\Node{2} 10) {}; && \node(0,0) (\Node{3} 10) {}; && \node(0,0) (\Node{4} 10) {}; && \node(0,0) (\Node{5} 10) {}; & \node(0,0) (t10 right) {}; & \node(0,0) (legend 10) {}; \\
\node(0,0) (t11 left) {}; & \node(0,0) (\Node{1} 11) {}; && \node(0,0) (\Node{2} 11) {}; && \node(0,0) (\Node{3} 11) {}; && \node(0,0) (\Node{4} 11) {}; && \node(0,0) (\Node{5} 11) {}; & \node(0,0) (t11 right) {}; & \node(0,0) (legend 11) {}; \\
\node(0,0) (t12 left) {}; & \node(0,0) (\Node{1} 12) {}; && \node(0,0) (\Node{2} 12) {}; && \node(0,0) (\Node{3} 12) {}; && \node(0,0) (\Node{4} 12) {}; && \node(0,0) (\Node{5} 12) {}; & \node(0,0) (t12 right) {}; & \node(0,0) (legend 12) {}; \\
\node(0,0) (t13 left) {}; & \node(0,0) (\Node{1} 13) {}; && \node(0,0) (\Node{2} 13) {}; && \node(0,0) (\Node{3} 13) {}; && \node(0,0) (\Node{4} 13) {}; && \node(0,0) (\Node{5} 13) {}; & \node(0,0) (t13 right) {}; & \node(0,0) (legend 13) {}; \\
\node(0,0) (t14 left) {}; & \node(0,0) (\Node{1} 14) {}; && \node(0,0) (\Node{2} 14) {}; && \node(0,0) (\Node{3} 14) {}; && \node(0,0) (\Node{4} 14) {}; && \node(0,0) (\Node{5} 14) {}; & \node(0,0) (t14 right) {}; & \node(0,0) (legend 14) {}; \\
\node(0,0) (t15 left) {}; & \node(0,0) (\Node{1} 15) {}; && \node(0,0) (\Node{2} 15) {}; && \node(0,0) (\Node{3} 15) {}; && \node(0,0) (\Node{4} 15) {}; && \node(0,0) (\Node{5} 15) {}; & \node(0,0) (t15 right) {}; & \node(0,0) (legend 15) {}; \\
\node(0,0) (t16 left) {}; & \node(0,0) (\Node{1} 16) {}; && \node(0,0) (\Node{2} 16) {}; && \node(0,0) (\Node{3} 16) {}; && \node(0,0) (\Node{4} 16) {}; && \node(0,0) (\Node{5} 16) {}; & \node(0,0) (t16 right) {}; & \node(0,0) (legend 16) {}; \\
\node(0,0) (t17 left) {}; & \node(0,0) (\Node{1} 17) {}; && \node(0,0) (\Node{2} 17) {}; && \node(0,0) (\Node{3} 17) {}; && \node(0,0) (\Node{4} 17) {}; && \node(0,0) (\Node{5} 17) {}; & \node(0,0) (t17 right) {}; & \node(0,0) (legend 17) {}; \\
\node(0,0) (t18 left) {}; & \node(0,0) (\Node{1} 18) {}; && \node(0,0) (\Node{2} 18) {}; && \node(0,0) (\Node{3} 18) {}; && \node(0,0) (\Node{4} 18) {}; && \node(0,0) (\Node{5} 18) {}; & \node(0,0) (t18 right) {}; & \node(0,0) (legend 18) {}; \\
\node(0,0) (t19 left) {}; & \node(0,0) (\Node{1} 19) {}; && \node(0,0) (\Node{2} 19) {}; && \node(0,0) (\Node{3} 19) {}; && \node(0,0) (\Node{4} 19) {}; && \node(0,0) (\Node{5} 19) {}; & \node(0,0) (t19 right) {}; & \node(0,0) (legend 19) {}; \\
\node(0,0) (t20 left) {}; & \node(0,0) (\Node{1} 20) {}; && \node(0,0) (\Node{2} 20) {}; && \node(0,0) (\Node{3} 20) {}; && \node(0,0) (\Node{4} 20) {}; && \node(0,0) (\Node{5} 20) {}; & \node(0,0) (t20 right) {}; & \node(0,0) (legend 20) {}; \\
\node(0,0) (ten left) {}; & \node(0,0) (\Node{1} en) {}; && \node(0,0) (\Node{2} en) {}; && \node(0,0) (\Node{3} en) {}; && \node(0,0) (\Node{4} en) {}; && \node(0,0) (\Node{5} en) {}; & \node(0,0) (ten right) {}; & \node(0,0) (legend en) {}; \\%
};

% Agents labels
\fill 
	(\Node{1} 0) node[draw,fill=white] {\Node{1}}%
	(\Node{2} 0) node[draw,fill=white] {\Node{2}}%
	(\Node{3} 0) node[draw,fill=white] {\Node{3}}%
	(\Node{4} 0) node[draw,fill=white] {\Node{4}}%
    (\Node{5} 0) node[draw,fill=white] {\Node{5}}%
    ;

% Vertical lifelines
\draw [dashed] 
    (\Node{1} 0) -- (\Node{1} en)%
    (\Node{2} 0) -- (\Node{2} en)%
    (\Node{3} 0) -- (\Node{3} en)%
    (\Node{4} 0) -- (\Node{4} en)%
    (\Node{5} 0) -- (\Node{5} en)%
    ;

% Legend
\filldraw[fill=green!20] (legend 00.north west) rectangle (legend 00.south east) node[midway,right=0.5 cm] {Transmit ($t$)};
\filldraw[fill=blue!20] (legend 01.north west) rectangle (legend 01.south east) node[midway,right=0.5cm] {Listen ($l$)};
\filldraw[fill=yellow!20] (legend 02.north west) rectangle (legend 02.south east) node[midway,right=0.5cm] {Sleep ($s$)};
\filldraw[fill=gray!20] (legend 03.west) rectangle (legend 03.east) node[midway,right=0.5cm] {Inform ($i$)};

% Horizontal time lines
\draw [dotted] 
    %(t01 left) node[left] {1} -- (t01 right)% omit
    %
    (t02 left) node[left] {2} -- (t02 right)% leave this in
    %
    %(t03 left) node[left] {3} -- (t03 right)% omit
    %(t04 left) node[left] {4} -- (t04 right)% omit
    %(t05 left) node[left] {5} -- (t05 right)% omit
    %
    (t06 left) node[left] {6} -- (t06 right)%
    %(t06 left) -- (t06 right) node[draw,fill=white,right,text width=6.89cm] {%
    %    time$_1$ $\leftarrow$ 6 \\ %
    %    waiting $\leftarrow \{ l_{2,5}, t_{1,6}, l_{3,6}, s_{2,7}, s_{4,8}, l_{5,9} \ldots \}$ \\ %
    %    transmits $\leftarrow \{ t_{4,3}, t_{1,6}, t_{1,10} \}$ %
    %}%
    %
    %(t07 left) node[left] {7} -- (t07 right)% omit
    %(t08 left) node[left] {8} -- (t08 right)% omit
    %(t09 left) node[left] {9} -- (t09 right)% omit
    %
    (t10 left) node[left] {10} -- (t10 right)%
    %(t10 left) -- (t10 right) node[draw,fill=white,right,text width=6.89cm] {%
    %    time$_2$ $\leftarrow$ 10 \\ %
    %    waiting $\leftarrow \{ t_{1,10}, t_{2,10}, l_{3,10}, t_{5,12}, i_{1,12} \ldots \}$ \\ %
    %    transmits $\leftarrow \{ t_{1,10}, t_{2,10}, t_{5,12} \}$ %
    %}%
    %
    %(t11 left) node[left] {11} -- (t11 right)% omit
    %(t12 left) node[left] {12} -- (t12 right)% omit
    %(t13 left) node[left] {13} -- (t13 right)% omit
    %
    (t14 left) node[left] {14} -- (t14 right)%
    %(t14 left) -- (t14 right) node[draw,fill=white,right,text width=6.89cm] {%
    %time$_3$ $\leftarrow$ 14 \\ %
    %waiting $\leftarrow \{ i_{3,13}, l_{4,14}, l_{2,14}, s_{1,14}, t_{3,16} \ldots \}$ \\ %
    %transmits $\leftarrow \{ t_{1,10}, t_{2,10}, t_{5,12}, t_{3,16} \}$ %
    %}%
    %   
    %(t15 left) node[left] {15} -- (t15 right)% omit
    %(t16 left) node[left] {16} -- (t16 right)% omit
    %(t17 left) node[left] {17} -- (t17 right)% omit
    %
    (t18 left) node[left] {18} -- (t18 right)%
    %(t18 left) -- (t18 right) node[draw,fill=white,right,text width=6.89cm] {%
    %time$_4$ $\leftarrow$ 18 \\ %
    %waiting $\leftarrow \{ t_{4,17}, t_{2,18}, t_{5,18}, l_{1,18}, s_{1,20} \ldots \}$ \\ %
    %transmits $\leftarrow \{ t_{3,16}, t_{4,17}, t_{2,18}, t_{5,18}, t_{3,20} \}$ %
    %}%
    %
    %(t19 left) node[left] {19} -- (t19 right)% omit
    %(t20 left) node[left] {20} -- (t20 right)% omit
    ;

\newcommand{\nid}{\@empty}
%Node{1}{
    \renewcommand{\nid}{1}
    \Transmit{\nid}{2}{6}
    \Transmit{\nid}{6}{10}
    \Inform{\nid}{12}
    \Sleep{\nid}{13}{14}
    \Listen{\nid}{14}{18}
    \Sleep{\nid}{18}{20}
%}

%Node{2}{
    \renewcommand{\nid}{2}
    \Listen{\nid}{2}{5}
    \Sleep{\nid}{5}{7}
    \Transmit{\nid}{8}{10}
    \Listen{\nid}{10}{14}
    \Transmit{\nid}{16}{18}
    \Inform{\nid}{20}
%}

%Node{3}{
    \renewcommand{\nid}{3}
    \Listen{\nid}{1}{6}
    \Listen{\nid}{7}{10}
    \Inform{\nid}{13}
    \Transmit{\nid}{14}{16}
    \Transmit{\nid}{18}{20}
%}

%Node{4}{
    \renewcommand{\nid}{4}
    \Transmit{\nid}{1}{3}
    \Sleep{\nid}{4}{8}
    \Listen{\nid}{9}{14}
    \Transmit{\nid}{15}{17}
    \Listen{\nid}{18}{20}
%}

%Node{5}{
    \renewcommand{\nid}{5}
    \Inform{\nid}{3}
    \Listen{\nid}{6}{9}
    \Transmit{\nid}{10}{12}
    \Transmit{\nid}{16}{18}
    \Inform{\nid}{20}
%}

\end{tikzpicture}

\caption{Message Sequence Chart of an execution.}\label{tikz:coordinatormsc}
\end{figure}

\Action{t}{1}{6}: transmission from \Node{1} ending at time 6. \medbreak

It is assumed that the example in \autoref{tikz:coordinatormsc} executes in real-time. For the sake of simplicity, we divide the time in to slots, where an action is received and queued in the same time slot as the action starts (e.g., the local time of the sending node). \smallbreak

The horizontal, dotted, lines at time slots 6, 10, 14, and 18 represents snapshots of the data structures of the Coordinator after a message has been received in part 1, but before any actions are processed in part 2.

%waiting $\leftarrow \{ t_{4,3}, i_{5,3}, l_{2,5}, t_{1,6}, l_{3,6}, s_{2,7}, s_{4,8}, l_{5,9}, t_{1,10}, t_{2,10}, l_{3,10}, t_{5,12}, i_{1,12}, i_{3,13}, l_{4,14}, l_{2,14}, s_{1,14}, t_{3,16}, t_{4,17}, t_{2,18}, t_{5,18}, l_{1,18}, s_{1,20}, i_{2,20}, i_{5,20}, t_{3,20}, l_{4,20} \}$

%waiting $\leftarrow \{t_{4,17}, t_{2,18}, t_{5,18}, l_{1,18}, s_{1,20}, i_{2,20}, i_{5,20}, t_{3,20}, l_{4,20} \}$

%waiting $\leftarrow \{ 
%t_{4,3}
%i_{5,3}

%l_{2,5}
%t_{1,6}
%l_{3,6}
%s_{2,7}
%s_{4,8}
%l_{5,9}
%t_{1,10}
%t_{2,10}
%l_{3,10}
%t_{5,12}
%i_{1,12}
%i_{3,13}
%l_{4,14}
%l_{2,14}
%s_{1,14}
%t_{3,16}
%t_{4,17}
%t_{2,18}
%t_{5,18}
%l_{1,18}
%s_{1,20}
%i_{2,20}
%i_{5,20}
%t_{3,20}
%l_{4,20} 
%\}$

\clearpage

\paragraph{Invariants} \

With a set of $N$ unique node identifiers, $\text{nodes} = \{ 1, 2, 3, \ldots, N \}$:

\begin{enumerate}
    \item $\forall \text{n} \in \text{nodes}.\ \text{exists at most one}\ \text{a} \in \text{waiting such that} \ \text{a.source = n}\ \land\ \text{a.type = listen}$
    %\item $\forall \text{n} \in \text{nodes}.\\ \quad ((\nexists \text{a} \in \text{waiting}.\ \text{a.source = n}\ \land\ \text{a.type = listen})\ \lor\\ \quad (\exists! \text{b} \in \text{waiting}.\ \text{b.source = n}\ \land\ \text{b.type = listen}))$
     
    %\exists^{\leq 1}\text{a} \in \text{waiting}.\ \text{a.source = n}\ \land\ \text{a.type = listen}$
    \begin{itemize}
        \item There is at most one action with the listen type from each node in the waiting queue.
    \end{itemize}
    \item $\forall \text{a}, \forall \text{b} \in \text{waiting}.\ \text{a.source = b.source}\ \land\ \text{a.type = listen} \implies \text{b.end} \leq \text{a.start}$
    \begin{itemize}
        \item If a node has a listen action in the waiting queue, no other actions may be present after this.
    \end{itemize}
    \item $\forall \text{a} \in \text{waiting}.\ \exists \text{b} \in \text{transmits}.\ \text{a.type = transmit} \implies \text{a} = \text{b} $ 
\end{enumerate}

The \texttt{Listen} hardware enforce 1 and 2 at \autoref{algo:hwfuncslisten:awaitend} in \autoref{algo:hwfuncslisten}, as the \KwAwait keyword is blocking while awaiting a response from the Coordinator. \medbreak

\todo[inline]{%
\textbf{TODO}

Show that sleep/inform/listen does not interfere with other actions.\medbreak

For every a on waiting list if a type is transmit then everybody else who can transmit on the same time is also on the transmit list.
Every b that intersects still belong to the transmit list.\medbreak

Suppose at time t in the real execution a listen ends and gets a packet with some probability computed with our distribution. We now want to prove that we return the packet with the same probability in the virtual time.
}

%For each node: if sleep/inform, 

% b.id = a.id 
% b.start >= a.end
%
%
%




\begin{algorithm}[ht]
    \DontPrintSemicolon
    \SetKwFunction{Linkmodel}{Linkmodel}
    \SetKwProg{Fn}{Linkmodel}{}{}

    \Fn{\Linkmodel{l}}{
        packets $\leftarrow$ empty Packet list\;

        \ForEach{t $\in$ l.transmissions}{
            \If{\texttt{ShouldReceive}(l, t, l.transmissions, |t.data|)}{
                \Append t.data \To packets\;
            }
        }

        \Send $|$packets$|$ \To l.source\; \label{line:sendcount}

        \ForEach{p $\in$ packets}{ \label{line:sendpackets}
            \Send p \To l.source\;
        }
    }

    \caption{Pseoducode for the linkmodel}
    \label{algo:linkmodel}
\end{algorithm}
