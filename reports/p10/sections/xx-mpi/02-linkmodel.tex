\chapter{Link Model notes}

\section{Function descriptions}

% \begin{algorithm}[H]
%    \DontPrintSemicolon
%    \SetKwFunction{RSSIAbsDiff}{rssi-abs-diff}
%    \SetKwProg{Fn}{Function}{}{}

%    \Fn{\RSSIAbsDiff{links}}{
%        \ForEach{$l \in $ links}{
%            neighbours $\leftarrow$ findNeighbourhood($l$)\;
%            \ForEach{$neighbour \in neighbours$}{
%                C $\leftarrow r(l, neighbour)$ // The autocorrelation %function\;

%                $\Sigma \leftarrow \sigma^2$C // The covariance matrix\;
%            }
%        }
%    }

%    \caption{Pseoducode for the linkmodel}
%    \label{algo:linkmodel}
% \end{algorithm}

Distance function $d(link)$ returns the distance of the link.\medbreak

\gls{cvpl} is the model for computing the path loss for the distance without any buildings acting as obstruction:
\begin{eq}
    \mathlarger{cvpl(distance) = 25 \cdot log_{10}(distance) + 45}
\end{eq}


\gls{bopl} is the model for computing the path loss for the distance, with buildings as obstruction:
\begin{eq}
    \mathlarger{bopl(distance) = ? \cdot log_{10}(distance) + ?}
\end{eq}

The fading function is the combination of both the \gls{cvpl} and \gls{bopl} function, where $cvp$ is the percentage of distance that is not obstructed by buildings. $bop$ is the percentage with buildings.
\begin{eq}
    fading(l, cvp, bop) = cvpl(d(l) \cdot cvp) + bopl(d(l) \cdot bop)
\end{eq}

\begin{eq}
    \mathlarger{compare(l_1, l_2) =|(l_1.rssi - l_d(d(l_1))) - (l_2.rssi - l_d(d(l_2)))|}
\end{eq}

\begin{eq}
    score(l_1, l_2) = \mathlarger{\sum}\limits_{l_1,\ l_2\ \in\ links} compare(l_1, l_2))^2
\end{eq}
\bigbreak

Optimization problem:
\begin{eq}
    op(l) = \alpha \cdot cvpl(d(l)) \cdot cvp + \beta \cdot bopl(d(l)) \cdot bop \cdot \lambda
\end{eq}


% \begin{eq}
%     \alpha \cdot d(l) + \beta \cdot d(l) + \lambda
% \end{eq}


% \begin{mini}|l|
%     {w,u}{f(w)+ R(w+6x)}{}{}
%     \addConstraint{g(w_k)+h(w_k)}{=0,}{k=0,\ldots,N-1}
%     \addConstraint{l(w_k)}{=5u,\quad}{k=0,\ldots,N-1}
% \end{mini}

% \begin{mini}|l|
%     {l \in\ measurements, l^{\prime} \in links}{score(l, l^{\prime})}{}{}
%     \addConstraint{}{}{}
% \end{mini}

% We define our optimisation problem, 


%%%% normal distribution of rssi minus distance fading in selected buckets
% \begin{tikzpicture}
%     \begin{axis}[
%             no markers, axis lines*=left,
%             enlargelimits=false, clip=false, axis on top,
%             xlabel=Average RSSI, ylabel=Probability density,
%             height=12cm, width=12cm,
%             domain=-40:40,
%             % xtick={4,6.5}, ytick=\empty,
%             grid=major
%         ]

%         \addplot[very thick, solid, cyan!50!black, samples=100] {gauss( 0.6787755836206504, 10.929221904852618)};
%         \addlegendentry{100 - 150};

%         \addplot[very thick, dashed, cyan!50!black, samples=100] {gauss(1.4021184158785516, 8.079011133504437)};
%         \addlegendentry{300 - 350};

%         \addplot[very thick, loosely dashed, cyan!50!black, samples=100] {gauss(1.9580853586687728, 5.444581964307228)};
%         \addlegendentry{500 - 550};

%         \addplot[very thick, dotted, cyan!50!black, samples=100] {gauss(-0.32184116101942706, 4.431214224678553)};
%         \addlegendentry{650 - 700};
%     \end{axis}
% \end{tikzpicture}

%%%% The average raw rssi for selected distance bucket
% \begin{tikzpicture}
%     \begin{axis}[
%         no markers, axis lines*=left,
%         %enlargelimits=false,
%         %xlabel=Average RSSI, ylabel=Probability density,
%         height=12cm, width=12cm,
%         domain=0:700,
%         xtick={2, 4, 6, 8, 10, 12, 14, 16, 18, 20, 22, 24, 26, 28, 30, 32, 34},
%         xticklabel={50, 100, 150, 200, 250, 300, 350, 400, 450, 500, 550, 600, 650, 700}
%         %stack plots=y
%         %% xtick={4,6.5}, ytick=\empty,
%         %grid=major
%         ]
%         \addplot coordinates {(0, -27.83419307295505) (1, -46.92156862745098) (2, -56.93304535637149) (3, -63.736301369863014) (4, -66.5452865064695) (5, -71.28635682158921) (6, -73.09453302961276) (7, -72.90234375) (8, -74.92373923739237) (9, -76.3717791411043) (10, -78.10295857988166) (11, -75.9241773962804) (12, -76.25923546418247) (13, -80.27360515021459) (14, -81.36496350364963) (15, -82.63513513513513) (16, -83.29462875197473) (17, -84.08870967741936) (18, -85.16179337231969) (19, -86.54743083003953) (20, -86.0923076923077) (21, -86.62264150943396) (22, -86.19658119658119) (23, -87.91878172588832) (24, -86.95373665480427) (25, -88.18548387096774) (26, -89.31612903225806) (27, -89.3529411764706) (28, -87.34693877551021) (29, -88.68888888888888) (30, -90.69767441860465) (31, -89.3076923076923) (32, -88.05882352941177) (33, -88.81818181818181) (34, -90.55)};
%         \addlegendentry{from measured values};

%         \addplot coordinates {(0, -19.0) (1, -52.05548236834798) (2, -59.31959641799338) (3, -63.63324587526918) (4, -66.71212547196625) (5, -69.10803434456606) (6, -71.06963425791125) (7, -72.7304778163845) (8, -74.17064690079624) (9, -75.44196437172961) (10, -76.57990143551223) (11, -77.60980684212777) (12, -78.5504260643717) (13, -79.41601268345701) (14, -80.21765799762699) (15, -80.96416238984608) (16, -81.6626258101218) (17, -82.31885947481244) (18, -82.93768004764144) (19, -83.52312439189049) (20, -84.07860931550455) (21, -84.60705239589171) (22, -85.11096473669596) (23, -85.5925231347412) (24, -86.0536269093458) (25, -86.49594314668114) (26, -86.92094308248812) (27, -87.32993162766424) (28, -87.72407153140404) (29, -88.10440330975827) (30, -88.4718618000685) (31, -88.8272900044145) (32, -89.17145073797043) (33, -89.505036487141) (34, -89.82867779781962)};
%         \addlegendentry{computed from distance function};
%     \end{axis}
% \end{tikzpicture}


% measured
%[(0, -27.83419307295505), (1, -46.92156862745098), (2, -56.93304535637149), (3, -63.736301369863014), (4, -66.5452865064695), (5, -71.28635682158921), (6, -73.09453302961276), (7, -72.90234375), (8, -74.92373923739237), (9, -76.3717791411043), (10, -78.10295857988166), (11, -75.9241773962804), (12, -76.25923546418247), (13, -80.27360515021459), (14, -81.36496350364963), (15, -82.63513513513513), (16, -83.29462875197473), (17, -84.08870967741936), (18, -85.16179337231969), (19, -86.54743083003953), (20, -86.0923076923077), (21, -86.62264150943396), (22, -86.19658119658119), (23, -87.91878172588832), (24, -86.95373665480427), (25, -88.18548387096774), (26, -89.31612903225806), (27, -89.3529411764706), (28, -87.34693877551021), (29, -88.68888888888888), (30, -90.69767441860465), (31, -89.3076923076923), (32, -88.05882352941177), (33, -88.81818181818181), (34, -90.55)]


% function
%[(0, -19.0), (1, -52.05548236834798), (2, -59.31959641799338), (3, -63.63324587526918), (4, -66.71212547196625), (5, -69.10803434456606), (6, -71.06963425791125), (7, -72.7304778163845), (8, -74.17064690079624), (9, -75.44196437172961), (10, -76.57990143551223), (11, -77.60980684212777), (12, -78.5504260643717), (13, -79.41601268345701), (14, -80.21765799762699), (15, -80.96416238984608), (16, -81.6626258101218), (17, -82.31885947481244), (18, -82.93768004764144), (19, -83.52312439189049), (20, -84.07860931550455), (21, -84.60705239589171), (22, -85.11096473669596), (23, -85.5925231347412), (24, -86.0536269093458), (25, -86.49594314668114), (26, -86.92094308248812), (27, -87.32993162766424), (28, -87.72407153140404), (29, -88.10440330975827), (30, -88.4718618000685), (31, -88.8272900044145), (32, -89.17145073797043), (33, -89.505036487141), (34, -89.82867779781962)]




%%%%%%%%%%%%%%%%%%%%%%%%%%%%%%% sheiz
%\begin{algorithm}[H]
%    \DontPrintSemicolon
%    \SetKwFunction{FLinkmodel}{Linkmodel}
%    \SetKwProg{Fn}{Function}{}{}
%
%    \Fn{\FLinkmodel{links}}{
%        \ForEach{$l \in $ links}{
%            neighbours $\leftarrow$ findNeighbourhood($l$)\;
%            \ForEach{$neighbour \in neighbours$}{
%                C $\leftarrow r(l, neighbour)$ // The autocorrelation %function\;
%
%                $\Sigma \leftarrow \sigma^2$C // The covariance matrix\;
%            }
%        }
%    }
%
%    \caption{Pseoducode for the linkmodel}
%    \label{algo:linkmodel}
%\end{algorithm}


% \begin{algorithm}[H]
%     \DontPrintSemicolon
%     \SetKwFunction{FLinkmodel}{Linkmodel}
%     \SetKwProg{Fn}{Function}{}{}
% 
%     \Fn{\FLinkmodel{links}}{
%         model $\leftarrow$ An unordered map with id as key and RSSI as % value\;\;
% 
%         \ForEach{$l$ $\in$ links}{
%             fading $\leftarrow$ 0\;
% 
%             \ForEach{$k$ $\in$ links}{
%                 \If{$k$ $\neq$ $l$ \KwAnd nodes($k$) $\cap$ nodes($l$) = % $\emptyset$}{
%                     \KwContinue\;
%                 }
%                 \ElseIf{$l$ $=$ $k$}{
%                     corr $\leftarrow$ 1\;
%                 }
%                 \Else{
%                     angle $\leftarrow \theta$($l$, $k$)\;
%                     corr $\leftarrow$ $r$(angle)\;
%                 }
%                 
%                 
%                 ran $\leftarrow$ Pick a random value from a gaussian % distribution with mean $=$ 0, and standard deviation $=$ 1\;
%                 fading $\leftarrow$ fading + ran $\cdot$ corr\;
% 
%             }
%             %\tcp{distanceFading $\leftarrow l_d$($d$($l$))}
%             model[l.id] = $tx_{power}$ - (fading)\;
%         }
%     }
%     \caption{Pseoducode for the linkmodel}
%     \label{algo:linkmodel}
% \end{algorithm}
% \smallbreak
% 
% $d(link)$ computes the distance of a link.
% 
% $\theta(l, link2)$ computes the angle between two links.
% 
% $r(angle)$ is the autocorrelation function.

%$l_d(distance)$ computes the pathloss for the distance dependent part.
