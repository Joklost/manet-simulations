% see, e.g., http://en.wikibooks.org/wiki/LaTeX/Customizing_LaTeX#New_commands
% for more information on how to create macros

%%%%%%%%%%%%%%%%%%%%%%%%%%%%%%%%%%%%%%%%%%%%%%%%
% Macros for the titlepage
%%%%%%%%%%%%%%%%%%%%%%%%%%%%%%%%%%%%%%%%%%%%%%%%
%Creates the aau titlepage
\newcommand{\aautitlepage}[3]{%
  {
    %set up various length
    \ifx\titlepageleftcolumnwidth\undefined
      \newlength{\titlepageleftcolumnwidth}
      \newlength{\titlepagerightcolumnwidth}
    \fi
    \setlength{\titlepageleftcolumnwidth}{0.5\textwidth-\tabcolsep}
    \setlength{\titlepagerightcolumnwidth}{\textwidth-2\tabcolsep-\titlepageleftcolumnwidth}
    %create title page
    \thispagestyle{empty}
    \noindent%
    \begin{tabular}{@{}ll@{}}
      \parbox{\titlepageleftcolumnwidth}{
        \iflanguage{danish}{%
          \includegraphics[width=\titlepageleftcolumnwidth]{figures/aau_logo_da}
        }{%
          \includegraphics[width=\titlepageleftcolumnwidth]{figures/aau_logo_en}
        }
      } &
      \parbox{\titlepagerightcolumnwidth}{\raggedleft\sf\small
        #2
      }\bigskip\\
       #1 &
      \parbox[t]{\titlepagerightcolumnwidth}{%
      \textbf{Abstract:}\bigskip\par
        \fbox{\parbox{\titlepagerightcolumnwidth-2\fboxsep-2\fboxrule}{%
          #3
        }}
      }\\
    \end{tabular}
    \vfill
    \iflanguage{danish}{%
      \noindent{\footnotesize\emph{Rapportens indhold er frit tilgængeligt, men offentliggørelse (med kildeangivelse) må kun ske efter aftale med forfatterne.}}
    }{%
      \noindent{\footnotesize\emph{The content of this report is freely available, but publication (with reference) may only be pursued due to agreement with the author.}}
    }
    \clearpage
  }
}

\newcommand*{\inlinegraphic}[1]{%
    \raisebox{-.02\baselineskip}{%
        \includegraphics[
        height=\baselineskip,
        width=\baselineskip,
        keepaspectratio,
        ]{#1}%
    }%
}

%Create english project info
\newcommand{\englishprojectinfo}[8]{%
  \parbox[t]{\titlepageleftcolumnwidth}{
    \textbf{Title:}\\ #1\bigskip\par
    \textbf{Theme:}\\ #2\bigskip\par
    \textbf{Project Period:}\\ #3\bigskip\par
    \textbf{Project Group:}\\ #4\bigskip\par
    \textbf{Participant(s):}\\ #5\bigskip\par
    \textbf{Supervisor(s):}\\ #6\bigskip\par
    \textbf{Copies:} #7\bigskip\par
    \textbf{Page Numbers:} \pageref{LastPage}\bigskip\par
    \textbf{Date of Completion:}\\ #8
  }
}

%Create danish project info
\newcommand{\danishprojectinfo}[8]{%
  \parbox[t]{\titlepageleftcolumnwidth}{
    \textbf{Titel:}\\ #1\bigskip\par
    \textbf{Tema:}\\ #2\bigskip\par
    \textbf{Projektperiode:}\\ #3\bigskip\par
    \textbf{Projektgruppe:}\\ #4\bigskip\par
    \textbf{Deltagere:}\\ #5\bigskip\par
    \textbf{Vejleder:}\\ #6\bigskip\par
    \textbf{Oplagstal:} #7\bigskip\par
    \textbf{Sidetal:} \pageref{LastPage}\bigskip\par
    \textbf{Afleveringsdato:}\\ #8
  }
}

%%%%%%%%%%%%%%%%%%%%%%%%%%%%%%%%%%%%%%%%%%%%%%%%
% An example environment
%%%%%%%%%%%%%%%%%%%%%%%%%%%%%%%%%%%%%%%%%%%%%%%%
\theoremheaderfont{\normalfont\bfseries}
\theorembodyfont{\normalfont}
\theoremstyle{break}
\def\theoremframecommand{{\color{aaublue!50}\vrule width 5pt \hspace{5pt}}}
\newshadedtheorem{exa}{Example}[chapter]
\newenvironment{example}[1]{%
		\begin{exa}[#1]
}{%
		\end{exa}
}

% Code listing macros
\newenvironment{java}[3][]
{\VerbatimEnvironment
\begin{listing}[H]
    \caption{#2}\label{#3}
    \begin{minted}[
        linenos=true,
        breaklines=true,
        breakautoindent=true,
        mathescape=true,
        escapeinside=\#\%,
        #1]{Java}}
{\end{minted}\end{listing}}

% Code listing macros
\newenvironment{cpp}[3][]
{\VerbatimEnvironment
\begin{listing}[ht]
    \caption{#2}\label{#3}
    \begin{minted}[
        linenos=true,
        breaklines=true,
        breakautoindent=true,
        mathescape=true,
        escapeinside=\#\%,
        #1]{cpp}}
{\end{minted}\end{listing}}

\newenvironment{csharp}[3][]
{\VerbatimEnvironment
\begin{listing}[H]
    \caption{#2}\label{#3}
    \begin{minted}[
        linenos=true,
        breaklines=true,
        breakautoindent=true,
        mathescape=true,
        escapeinside=\#\%,
        #1]{csharp}}
{\end{minted}\end{listing}}

\newenvironment{xml}[3][]
{\VerbatimEnvironment
\begin{listing}[!htb]
    \caption{#2}\label{#3}
    \begin{minted}[
        linenos=true,
        breaklines=true,
        breakautoindent=true,
        mathescape=true,
        escapeinside=\#\%,
        #1]{XML}}
{\end{minted}\end{listing}}

\NewEnviron{eq}{%
    \begin{equation}
        \scalebox{1.2}{$\BODY$}
    \end{equation}
}

% this is needed, or the \renewcommand for the \listingautorefname will fail
\newcommand{\listingautorefname}{Code} 

\addto\extrasenglish{
    \renewcommand{\chapterautorefname}{Chapter}
    \renewcommand{\listingautorefname}{Code}
    \renewcommand{\sectionautorefname}{Section}
    \renewcommand{\subsectionautorefname}{Section}
    \renewcommand{\subsubsectionautorefname}{Section}
    \renewcommand{\algorithmautorefname}{Algorithm}
}

\newcommand{\seqpacket}[2]{Packet~[Type:~\texttt{#1}, Data:~\texttt{#2}]}

\newcommand{\tableref}[1]{\autoref{table:#1}}
\newcommand{\figureref}[1]{\autoref{figure:#1}}
\newcommand{\coderef}[1]{\autoref{code:#1}}
\newcommand{\secref}[1]{\autoref{sec:#1}}
\newcommand{\chapref}[1]{\autoref{chap:#1}}
\newcommand{\appendixref}[1]{\autoref{app:#1}}
\renewcommand{\lineref}[1]{Line \ref{line:#1}}

\newcommand{\doublequote}[1]{``{#1}''}
\newcommand{\singlequote}[1]{`{#1}'}

\newcommand{\vect}[1]{\ensuremath{\overrightarrow{#1}}}
\newcommand{\argmin}{\operatornamewithlimits{arg\,min}}

\newcommand{\citationneeded}[1][]{[citation needed]}

%\makeatletter
%\newcounter{datastructures}
%\let\c@algocf\c@datastructures

%\newenvironment{datastructures}[1][htb]
%  {\renewcommand{\algorithmcfname}{Data Structures}% Update algorithm name
%   \let\c@algocf\c@datastructures% Update algorithm counter
%   \begin{algorithm}[#1]%
%  }{\end{algorithm}}
%\makeatother
\renewcommand{\labelenumi}{\Roman{enumi}}

\SetKw{KwAwait}{await}
\SetKw{KwFrom}{from}
\SetKw{KwSend}{send}
\SetKw{KwAppend}{append}
\SetKw{KwClear}{clear}
\SetKw{KwRemove}{remove}
\SetKw{KwAnd}{and}
\SetKw{KwOr}{or}
\SetKw{KwTo}{to}
\SetKw{KwTransmit}{transmit}
\SetKw{KwListen}{listen}
\SetKw{KwContinue}{continue}
\SetKw{KwNow}{now}
\SetKw{KwPeek}{peek}
\SetKw{KwEnqueue}{enqueue}
\SetKw{KwDequeue}{dequeue}
\SetKw{KwHead}{head}
\SetKw{KwBreak}{break}
\SetKw{KwNull}{null}
\SetKw{KwWhere}{where}
\SetKw{KwInn}{in}
\SetKw{KwFind}{find}

\newcommand{\Node}[1]{Node #1}

\newcommand{\Transmit}[3]
{
    \filldraw[fill=green!20] (\Node{#1} #2.west) rectangle (\Node{#1} #3.east) node[midway] {$t_{#1,#3}$};
}

\newcommand{\Listen}[3]
{
    \filldraw[fill=blue!20] (\Node{#1} #2.west) rectangle (\Node{#1} #3.east) node[midway] {$l_{#1,#3}$};
}

\newcommand{\Sleep}[3]
{
    \filldraw[fill=yellow!20] (\Node{#1} #2.west) rectangle (\Node{#1} #3.east) node[midway] {$s_{#1,#3}$};
}

\newcommand{\Inform}[2]
{
    \filldraw[fill=gray!20] (\Node{#1} #2.west) rectangle (\Node{#1} #2.east) node[midway,above,fill=white] {$i_{#1,#2}$};
}