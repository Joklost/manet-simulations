% see, e.g., http://en.wikibooks.org/wiki/LaTeX/Customizing_LaTeX#New_commands
% for more information on how to create macros

%%%%%%%%%%%%%%%%%%%%%%%%%%%%%%%%%%%%%%%%%%%%%%%%
% Macros for the titlepage
%%%%%%%%%%%%%%%%%%%%%%%%%%%%%%%%%%%%%%%%%%%%%%%%
%Creates the aau titlepage
\newcommand{\aautitlepage}[3]{%
    {
            %set up various length
            \ifx\titlepageleftcolumnwidth\undefined
                \newlength{\titlepageleftcolumnwidth}
                \newlength{\titlepagerightcolumnwidth}
            \fi
            \setlength{\titlepageleftcolumnwidth}{0.5\textwidth-\tabcolsep}
            \setlength{\titlepagerightcolumnwidth}{\textwidth-2\tabcolsep-\titlepageleftcolumnwidth}
            %create title page
            \thispagestyle{empty}
            \noindent%
            \begin{tabular}{@{}ll@{}}
                \parbox{\titlepageleftcolumnwidth}{
                    \iflanguage{danish}{%
                        \includegraphics[width=\titlepageleftcolumnwidth]{figures/aau_logo_da}
                    }{%
                        \includegraphics[width=\titlepageleftcolumnwidth]{figures/aau_logo_en}
                    }
                }  &
                \parbox{\titlepagerightcolumnwidth}{\raggedleft\sf\small
                    #2
                }\bigskip \\
                #1 &
                \parbox[t]{\titlepagerightcolumnwidth}{%
                    \textbf{Abstract:}\bigskip\par
                    \fbox{\parbox{\titlepagerightcolumnwidth-2\fboxsep-2\fboxrule}{%
                            #3
                        }}
                }         \\
            \end{tabular}
            \vfill
            \iflanguage{danish}{%
                \noindent{\footnotesize\emph{Rapportens indhold er frit tilgængeligt, men offentliggørelse (med kildeangivelse) må kun ske efter aftale med forfatterne.}}
            }{%
                \noindent{\footnotesize\emph{The content of this thesis is freely available, but publication (with reference) may only be pursued due to agreement with the author.}}
            }
            \clearpage
        }
}

\newcommand*{\inlinegraphic}[1]{%
    \raisebox{-.02\baselineskip}{%
        \includegraphics[
            height=\baselineskip,
            width=\baselineskip,
            keepaspectratio,
        ]{#1}%
    }%
}

%Create english project info
\newcommand{\englishprojectinfo}[8]{%
    \parbox[t]{\titlepageleftcolumnwidth}{
        \textbf{Title:}\\ #1\bigskip\par
        \textbf{Theme:}\\ #2\bigskip\par
        \textbf{Project Period:}\\ #3\bigskip\par
        \textbf{Project Group:}\\ #4\bigskip\par
        \textbf{Participant(s):}\\ #5\bigskip\par
        \textbf{Supervisor(s):}\\ #6\bigskip\par
        \textbf{Copies:} #7\bigskip\par
        \textbf{Page Numbers:} \pageref{LastPage}\bigskip\par
        \textbf{Date of Completion:}\\ #8
    }
}

%Create danish project info
\newcommand{\danishprojectinfo}[8]{%
    \parbox[t]{\titlepageleftcolumnwidth}{
        \textbf{Titel:}\\ #1\bigskip\par
        \textbf{Tema:}\\ #2\bigskip\par
        \textbf{Projektperiode:}\\ #3\bigskip\par
        \textbf{Projektgruppe:}\\ #4\bigskip\par
        \textbf{Deltagere:}\\ #5\bigskip\par
        \textbf{Vejleder:}\\ #6\bigskip\par
        \textbf{Oplagstal:} #7\bigskip\par
        \textbf{Sidetal:} \pageref{LastPage}\bigskip\par
        \textbf{Afleveringsdato:}\\ #8
    }
}

%%%%%%%%%%%%%%%%%%%%%%%%%%%%%%%%%%%%%%%%%%%%%%%%
% An example environment
%%%%%%%%%%%%%%%%%%%%%%%%%%%%%%%%%%%%%%%%%%%%%%%%
\theoremheaderfont{\normalfont\bfseries}
\theorembodyfont{\normalfont}
\theoremstyle{break}
\def\theoremframecommand{{\color{aaublue!50}\vrule width 5pt \hspace{5pt}}}
\newshadedtheorem{exa}{Example}[chapter]
\newenvironment{example}[1]{%
    \begin{exa}[#1]
        }{%
    \end{exa}
}

% Code listing macros
\newenvironment{java}[3][]
{\VerbatimEnvironment
    \begin{listing}[H]
        \caption{#2}\label{#3}
        \begin{minted}[
        linenos=true,
        breaklines=true,
        breakautoindent=true,
        mathescape=true,
        escapeinside=\#\%,
        #1]{Java}}
{\end{minted}\end{listing}}

% Code listing macros
\newenvironment{cpp}[3][]
{\VerbatimEnvironment
    \begin{listing}[H]
        \caption{#2}\label{#3}
        \begin{minted}[
        linenos=true,
        breaklines=true,
        breakautoindent=true,
        mathescape=true,
        escapeinside=\#\%,
        #1]{cpp}}
{\end{minted}\end{listing}}

\newenvironment{csharp}[3][]
{\VerbatimEnvironment
    \begin{listing}[H]
        \caption{#2}\label{#3}
        \begin{minted}[
        linenos=true,
        breaklines=true,
        breakautoindent=true,
        mathescape=true,
        escapeinside=\#\%,
        #1]{csharp}}
{\end{minted}\end{listing}}

\newenvironment{xml}[3][]
{\VerbatimEnvironment
    \begin{listing}[!htb]
        \caption{#2}\label{#3}
        \begin{minted}[
        linenos=true,
        breaklines=true,
        breakautoindent=true,
        mathescape=true,
        escapeinside=\#\%,
        #1]{XML}}
{\end{minted}\end{listing}}

\NewEnviron{eq}{%
    \begin{equation}
        \scalebox{1}{$\BODY$}
    \end{equation}
}

% this is needed, or the \renewcommand for the \listingautorefname will fail
\newcommand{\listingautorefname}{Code}

\addto\extrasenglish{
    \renewcommand{\chapterautorefname}{Chapter}
    \renewcommand{\listingautorefname}{Code}
    \renewcommand{\sectionautorefname}{Section}
    \renewcommand{\subsectionautorefname}{Section}
    \renewcommand{\subsubsectionautorefname}{Section}
    \renewcommand{\algorithmautorefname}{Algorithm}
}

\newcommand{\doublequote}[1]{``{#1}''}
\newcommand{\singlequote}[1]{`{#1}'}

\newcommand{\vect}[1]{\ensuremath{\overrightarrow{#1}}}
\newcommand{\argmin}{\operatornamewithlimits{arg\,min}}

\newcommand{\citationneeded}[1][]{[citation needed]}

%\renewcommand{\labelenumi}{\Roman{enumi}}

\SetKw{KwAwait}{await}
\SetKw{KwFrom}{from}
\SetKw{KwSend}{send}
\SetKw{KwAppend}{append}
\SetKw{KwClear}{clear}
\SetKw{KwRemove}{remove}
\SetKw{KwAnd}{and}
\SetKw{KwOr}{or}
\SetKw{KwTo}{to}
\SetKw{KwTransmit}{transmit}
\SetKw{KwListen}{listen}
\SetKw{KwContinue}{continue}
\SetKw{KwNow}{now}
\SetKw{KwPeek}{peek}
\SetKw{KwEnqueue}{enqueue}
\SetKw{KwDequeue}{dequeue}
\SetKw{KwHead}{head}
\SetKw{KwBreak}{break}
\SetKw{KwNull}{\textit{null}}
\SetKw{KwWhere}{where}
\SetKw{KwInn}{in}
\SetKw{KwIs}{is}
\SetKw{KwFind}{find}
\SetKw{KwReturn}{return}

\newcommand{\Node}[1]{Node #1}

\newcommand{\LineWidth}{0.8mm}
\newcommand{\ProcessedLine}{\tikz[baseline=-0.6ex]{\draw[line width=\LineWidth,dotted] (0,0) -- (1,0);}}
\newcommand{\SubmittedLine}{\tikz[baseline=-0.6ex]{\draw[line width=\LineWidth] (0,0) -- (1,0);}}
\newcommand{\Processed}[1]{\draw[line width=\LineWidth,dotted] #1;}
\newcommand{\Submitted}[1]{\draw[line width=\LineWidth] #1;}

\newcommand{\LineLegend}{%
    \Processed{%
        (legend1-5.north west) -- (legend1-5.west) -- %
        (legend1-5.center) -- (legend1-5.south) -- %
        (legend1-5.south east) -- (legend1-5.north east) -- %
        (legend2-5.north) %
        node[midway,right=0.5cm] {Processed}%
    }

    \Submitted{%
        (legend1-6.north west) -- (legend1-6.west) -- %
        (legend1-6.center) -- (legend1-6.south) -- %
        (legend1-6.south east) -- (legend1-6.north east) -- %
        (legend2-6.north) %
        node[midway,right=0.5cm] {Submitted}%
    }
}

\newcommand{\NodeMatrix}[2]{%
    \let\matrixcontent\empty
    \foreach \j in {0,...,#1}{%
            \foreach \i in {0,...,#2} {%
                    \begingroup\edef\x{\endgroup\noexpand\gappto\noexpand\matrixcontent{|(\j-\i)| \&}}\x
                }%
            \gappto\matrixcontent{\\}%
        }

    %% Matrix
    \matrix (matrix) at (0,0) [matrix of nodes, ampersand replacement=\&,very thin,column sep=0.1cm,row sep=0.3cm] {%
        \matrixcontent%
    };
}

\newcommand{\ExecutionMatrix}[1]{%
    \let\matrixcontent\empty
    \foreach \j in {0,...,#1,end}{%
            \foreach \i in {tleft,sep0,\Node{1},sep1,\Node{2},sep2,\Node{3},sep3,\Node{4},sep4,\Node{5},sep5,tright,legend1,legend2} {%
                    \begingroup\edef\x{\endgroup\noexpand\gappto\noexpand\matrixcontent{|(\i-\j)| \&}}\x
                }%
            \gappto\matrixcontent{\\}%
        }

    %% Matrix
    \matrix (matrix) at (0,0) [matrix of nodes, ampersand replacement=\&,very thin,column sep=0.1cm,row sep=0.2cm] {%
        \matrixcontent%
    };

    % Node labels
    \fill
    (\Node{1}-0) node[draw,fill=white] {\Node{1}}%
    (\Node{2}-0) node[draw,fill=white] {\Node{2}}%
    (\Node{3}-0) node[draw,fill=white] {\Node{3}}%
    (\Node{4}-0) node[draw,fill=white] {\Node{4}}%
    (\Node{5}-0) node[draw,fill=white] {\Node{5}}%
    ;

    % Vertical lifelines
    \draw [dashed]
    (\Node{1}-0) -- (\Node{1}-end)%
    (\Node{2}-0) -- (\Node{2}-end)%
    (\Node{3}-0) -- (\Node{3}-end)%
    (\Node{4}-0) -- (\Node{4}-end)%
    (\Node{5}-0) -- (\Node{5}-end)%
    ;

    % Legend
    \filldraw[fill=green!20] (legend1-0.north west) rectangle (legend1-0.south east) node[midway] {$t$};
    \fill (legend2-0) node[right=0.28cm] {Transmit};
    \filldraw[fill=blue!20] (legend1-1.north west) rectangle (legend1-1.south east) node[midway] {$l$};
    \fill (legend2-1) node[right=0.28cm] {Listen};
    \filldraw[fill=white!20] (legend1-2.north west) rectangle (legend1-2.south east) node[midway] {$s$};
    \fill (legend2-2) node[right=0.28cm] {Sleep};
    \filldraw[fill=gray!20] (legend1-3.south west) rectangle (legend1-3.south east) node[above,midway] {$i$};
    \fill (legend2-3) node[right=0.28cm] {Inform};
}

\newcommand{\Action}[3]{\ensuremath{#1^{#2}_{#3}}}
\newcommand{\T}[2]{\Action{t}{#1}{#2}}
\renewcommand{\L}[2]{\Action{l}{#1}{#2}}
\renewcommand{\S}[2]{\Action{s}{#1}{#2}}
\newcommand{\I}[2]{\Action{i}{#1}{#2}}

\newcommand{\nid}{\@empty}

\newcommand{\Transmit}[3]{%
    \filldraw[fill=green!20] (\Node{#1}-#2.west) rectangle (\Node{#1}-#3.east) node[midway] {\Action{t}{#1}{#3}};%
}

\newcommand{\Listen}[3]{%
    \filldraw[fill=blue!20] (\Node{#1}-#2.west) rectangle (\Node{#1}-#3.east) node[midway] {\Action{l}{#1}{#3}};
}

\newcommand{\Sleep}[3]{%
    \filldraw[fill=white!20] (\Node{#1}-#2.west) rectangle (\Node{#1}-#3.east) node[midway] {\Action{s}{#1}{#3}};
}

\newcommand{\Inform}[2]{%
    \filldraw[fill=gray!20] (\Node{#1}-#2.west) rectangle (\Node{#1}-#2.east) node[midway,above,fill=white] {\Action{i}{#1}{#2}};
}

\newcommand{\CoordinatorFigure}[1]{%
    \begin{tikzpicture}[every node/.style={font=\normalsize,minimum height=0.5cm,minimum width=0.75cm},]
        \ExecutionMatrix{20}

        % Horizontal time lines
        \draw [dotted]
        %(t01 left) node[left] {1} -- (t01 right)% omit
        %
        (tleft-2) node[left] {2} -- (tright-2)% leave this in
        %
        %(t03 left) node[left] {3} -- (t03 right)% omit
        %(t04 left) node[left] {4} -- (t04 right)% omit
        %(t05 left) node[left] {5} -- (t05 right)% omit
        %
        (tleft-6) node[left] {6} -- (tright-6)% leave this in
        %
        %(t07 left) node[left] {7} -- (t07 right)% omit
        %(t08 left) node[left] {8} -- (t08 right)% omit
        %(t09 left) node[left] {9} -- (t09 right)% omit
        %
        (tleft-10) node[left] {10} -- (tright-10)% leave this in        
        %
        %(t11 left) node[left] {11} -- (t11 right)% omit
        %(t12 left) node[left] {12} -- (t12 right)% omit
        %(t13 left) node[left] {13} -- (t13 right)% omit
        %
        (tleft-14) node[left] {14} -- (tright-14)% leave this in        
        %   
        %(t15 left) node[left] {15} -- (t15 right)% omit
        %(t16 left) node[left] {16} -- (t16 right)% omit
        %(t17 left) node[left] {17} -- (t17 right)% omit
        %
        (tleft-18) node[left] {18} -- (tright-18)% leave this in              
        %
        %(t19 left) node[left] {19} -- (t19 right)% omit
        %(t20 left) node[left] {20} -- (t20 right)% omit
        ;

        \renewcommand{\nid}{\@empty}
        %Node{1}{
        \renewcommand{\nid}{1}
        \Transmit{\nid}{2}{5}
        \Transmit{\nid}{6}{10}

        \Inform{\nid}{12}
        \Sleep{\nid}{13}{15}
        \Listen{\nid}{15}{18}
        \Sleep{\nid}{18}{20}
        %}

        %Node{2}{
        \renewcommand{\nid}{2}
        \Listen{\nid}{2}{5}
        \Sleep{\nid}{5}{7}

        \Transmit{\nid}{8}{10}
        \Listen{\nid}{10}{14}
        \Transmit{\nid}{16}{18}
        \Inform{\nid}{20}
        %}

        %Node{3}{
        \renewcommand{\nid}{3}
        \Listen{\nid}{1}{6}

        \Listen{\nid}{7}{10}
        \Inform{\nid}{13}
        \Transmit{\nid}{14}{16}
        \Transmit{\nid}{18}{20}
        %}

        %Node{4}{
        \renewcommand{\nid}{4}
        \Transmit{\nid}{1}{3}
        \Sleep{\nid}{4}{8}

        \Listen{\nid}{9}{14}
        \Transmit{\nid}{15}{17}
        \Listen{\nid}{18}{20}
        %}

        %Node{5}{
        \renewcommand{\nid}{5}
        \Inform{\nid}{4}
        \Sleep{\nid}{5}{6}
        \Listen{\nid}{6}{9}

        \Transmit{\nid}{10}{12}
        \Transmit{\nid}{16}{18}
        \Inform{\nid}{20}
        %}

        #1
    \end{tikzpicture}
}

\newcommand{\bibtodo}[1]{\todo[inline]{bibliographical remark: #1}}

\tikzset{
    between/.style args={#1 and #2}{
            at = ($(#1)!0.5!(#2)$)
        }
}

\tikzset{
    between base/.style args={#1 and #2}{
            between=#1.base and #2.base
        }
}

\pgfplotsset{
    compat=1.15,
    /pgf/declare function={
            bopl(\x) = (67 * (ln(\x) / ln(57)) + 11.5);
            cvpl(\x) = (48.5 * (ln(\x) / ln(77)) + 37.5);
            ld(\x) = (55 * log10(\x) - 18.8);
            gauss(\x, \a) = 1/(\a*sqrt(2*pi))*exp(-((x-\x)^2)/(2*\a^2));
        },
}

% % #1 = mean, #2 = standard deviation
% \pgfmathdeclarefunction{gauss}{2}{%
%   \pgfmathparse{1/(#2*sqrt(2*pi))*exp(-((x-#1)^2)/(2*#2^2))}%
% }

\tikzset{
    >=stealth',
    punktchain/.style={
            rectangle,
            % fill=black!10,
            draw=black, very thick,
            text width=10em,
            minimum height=3em,
            text centered,
            on chain},
    line/.style={draw, thick, <-},
    element/.style={
            tape,
            top color=white,
            bottom color=blue!50!black!60!,
            minimum width=8em,
            draw=blue!40!black!90, very thick,
            text width=10em,
            minimum height=3.5em,
            text centered,
            on chain},
    every join/.style={->, thick,shorten >=1pt},
    decoration={brace},
    tuborg/.style={decorate},
    tubnode/.style={midway, right=2pt},
}

\DefineVerbatimEnvironment%
{verbatimprog}%
{Verbatim}%
{fontsize=\footnotesize}%